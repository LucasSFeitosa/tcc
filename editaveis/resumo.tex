\begin{resumo}
% O resumo deve ressaltar o objetivo, o método, os resultados e as conclusões
% do documento. A ordem e a extensão
% destes itens dependem do tipo de resumo (informativo ou indicativo) e do
% tratamento que cada item recebe no documento original. O resumo deve ser
% precedido da referência do documento, com exceção do resumo inserido no
% próprio documento. (\ldots) As palavras-chave devem figurar logo abaixo do
% resumo, antecedidas da expressão Palavras-chave:, separadas entre si por
% ponto e finalizadas também por ponto. O texto pode conter no mínimo 150 e
% no máximo 500 palavras, é aconselhável que sejam utilizadas 200 palavras.
% E não se separa o texto do resumo em parágrafos.

A implantação de aplicações web de grande escala apresentam vários desafios, com
isso, a implantação automatizada de software vem se tornando uma necessidade,
principalmente pelos desafios de instalação e configuração das ferramentas web
mais modernas, podendo facilitar a instalação e configuração de aplicações para
desenvolvedores e para usuários. Este trabalho trata da implantação de software
com seus procedimentos e ferramentas, também a implantação  automatizada de
aplicações em sistema Debian GNU/Linux, para isso foi feito a colaboração da
construção da ferramenta Shak, que propõe a instalação de aplicações web com
apenas uma instrução, o que possibilita a instalaçaõ e configuração de ferramentas
web disponíveis nos servidores oficiais do Debian, utilizando o uso de protocolos
seguros como HTTPS e implantação em hosts virtuais.
 \vspace{\onelineskip}

 \noindent
 \textbf{Palavras-chaves}: Implantação de software, Debian.
\end{resumo}
