\chapter{Considerações preliminares}
\label{cap-conclusoes}

No decorrer deste trabalho, alguns resultados foram alcançados. Primeiramente, o
pesquisador estabeleceu um bom conhecimento sobre implantação automatizada
de software e tecnologias que permites tal implementação, como por exemplo o chef
e o puppet, sendo possível automatizar todo o processo de implantação de um software,
desde o planejamento até a execução também foi conhecido ferramentas que são
utilizadas para implantação de aplicações de forma automatizada, como por exemplo
o JuJu e o Bitnami, que ajudam usuários que não possuem conhecimento a
instalarem aplicações com facilidade.

Além disso, também foram adquiridos os conhecimentos sobre DevOps, que é um conjunto
de práticas e princípios para a implantação ágil de software, que agregou bastante
ao trabalho, pois também trata de automação de implantação de software, resolvendo
um problema maior, que é a falta de integração entre a equipe de desenvolvimento
e a equipe de implantação. E também hospedagem virtual, que é uma forma de manter
várias aplicações no mesmo servidor com um único endereço IP e vários domínios diferentes.

Em relação a construção de solução, foram realizadas melhorias na ferramenta Shak,
provendo três novas aplicações, sendo elas Owncloud, Wordpress e servidor de e-mail,
todas sendo implantadas automaticamente, levando em consideração o uso de protocolos
seguros como HTTP junto ao SSL e SMTP junto ao SSL. Logo também foi adicionado a
ferramenta Shak suporte a criação de certificados autoassinados, permitindo que
as aplicações possam usar certificados para configurar tais protocolos. Ainda foi
adicionado ao Shak o suporte a forçar que sempre as aplicações web utilizem
o protocolo HTTPS, evitando assim implantações inseguras, independente da ferramenta.

Também foi configurado o suporte a múltiplas instâncias nas aplicações disponíveis
no shak, o Wordpress já previa tal suporte, bastando configurá-lo, porém no Owncloud
foi necessário o envio de um patch para o mantenedor do pacote Debian, visto que
é uma funcionalidade nova no Owncloud, permitindo assim a criação de múltiplas
instâncias do Owncloud, em ambos os casos foi utilizado a técnica de hospedagem
virtual, disponibilizada pelo servidor web Nginx.

Todos esses resultados parciais são importantes para o objetivo de pesquisa, que vão
desde compreender como é o processo de implantação de um software, até os cuidados
a se tomar ao implantar uma aplicação, baseado nas considerações, serão apresentado
na próxima seção um Cronograma de atividades para o TCC2.


\section{Cronograma}
%

Afim de apresentar uma continuação para este trabalho, a tabela \ref{tab:atividades_futuras},
apresenta as atividades a serem realizadas e suas respectivas datas.
O objetivo inicial na próxima etapa do projeto é continuar a evolução do shak,
propondo uma forma de descentralizar as receitas chef da aplicação central do shak,
evitando que as receitas chef que instalam as aplicações precisem estar junto ao
código fonte do shak. Além disso, evoluir a documentação do Shak, para que novas
pessoas que se interessarem possam evoluir a ferramenta e propor novas aplicações,
além disso escreverem suas próprias aplicações.

Outro objetivo é melhorar a a forma que são gerados os certificados autoassinados,
para prover uma melhor segurança a infraestrutura do Shak, e facilitando a gerência
desses certificados, de forma que caso seja possível que o usuário possa inserir um certificado
reconhecido por alguma entidade certificadora.

Além disso prover ao Shak uma forma de realizar testes nas aplicações inseridas,
com casos de teste e prover uma qualidade maior das aplicações. Também será levantado
uma lista dos bugs existentes no Shak, para que possam ser solucionados durante
a fase seguinte, provendo a manutenção e evolução do software. Uma das evoluções
propostas será o suporte a implantação de aplicações em caminhos diversos no mesmo
domínio, sendo possível, por exemplo: www.exemplo.com.br/blog para o Wordpress e
www.exemplo.com.br/nuvem para o Owncloud. As atividades serão listadas na tabela
a seguir:

\begin{table}[h]
\centering
\resizebox{\textwidth}{!}{\begin{tabular}{|l|l|l|}
\hline
\rowcolor[HTML]{EFEFEF}
{\textbf{Atividade}} & {\textbf{Data de Início}} & {\textbf{Data de Término}} \\ \hline
Levantamento de bugs e soluções de bugs no Shak & 01/03/2016 & 15/03/2016 \\
Descentralização de receitas & 15/03/2016 & 30/03/2016 \\
Suporte a novas apliacções & 01/04/2016 & 15/04/2016 \\
Levantamento de bugs e soluções de bugs no Shak & 15/04/2016 & 30/04/2016 \\
Evolução da gerência de certificados  & 01/05/2016 & 15/05/2016 \\
Levantamento de bugs e soluções de bugs no Shak & 15/05/2016 & 30/05/2016 \\
Finalizar documento & 01/05/2016 & 30/06/2016 \\
\hline
\end{tabular}}
\caption{Atividades a serem realizadas no TCC2}
\label{tab:atividades_futuras}
\end{table}
