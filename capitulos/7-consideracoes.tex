\chapter{Conclusão}
\label{cap-conclusoes}

No decorrer deste trabalho, alguns resultados foram alcançados. Primeiramente, o
pesquisador estabeleceu o conhecimento sobre implantação automatizada
de software e tecnologias que permites tal implementação, como por exemplo o Chef
e o Puppet, sendo possível automatizar todo o processo de implantação de um software,
desde o planejamento até a execução também foi conhecido ferramentas que são
utilizadas para implantação de aplicações de forma automatizada, como por exemplo
o JuJu e o Bitnami, que ajudam usuários que não possuem conhecimento a
instalarem aplicações com facilidade.

Além disso, também foram adquiridos os conhecimentos sobre DevOps, que é um conjunto
de práticas e princípios para a implantação ágil de software, que agregou bastante
ao trabalho, pois também trata de automação de implantação de software, resolvendo
um problema maior, que é a falta de integração entre a equipe de desenvolvimento
e a equipe de implantação. E também hospedagem virtual, que é uma forma de manter
várias aplicações no mesmo servidor com um único endereço IP e vários domínios diferentes.

Em relação a construção de solução, foram realizadas melhorias na ferramenta Shak,
provendo três novas aplicações, sendo elas Owncloud, Wordpress e servidor de e-mail,
todas sendo implantadas automaticamente, levando em consideração o uso de protocolos
seguros como HTTP junto ao SSL e SMTP junto ao SSL. Logo também foi adicionado a
ferramenta Shak suporte a criação de certificados auto-assinados, permitindo que
as aplicações possam usar certificados para configurar tais protocolos. Ainda foi
adicionado ao Shak o suporte a as aplicações web sempre utilizarem
o protocolo HTTPS, evitando assim implantações inseguras, independente da ferramenta.

Também foi configurado o suporte a múltiplas instâncias nas aplicações disponíveis
no shak, o Wordpress já previa tal suporte, bastando configurá-lo, porém no Owncloud
foi necessário o envio de uma correção para o mantenedor do pacote Debian, visto que
é uma funcionalidade nova no Owncloud, permitindo assim a criação de múltiplas
instâncias do Owncloud, em ambos os casos foi utilizado a técnica de hospedagem
virtual, disponibilizada pelo servidor web Nginx.

Todos esses resultados parciais são importantes para o objetivo de pesquisa, que vão
desde compreender como é o processo de implantação de um software, até os cuidados
a se tomar ao implantar uma aplicação, baseado nas considerações, serão apresentados
na próxima seção um Cronograma de atividades para o TCC2.


\section{Limitações}
%

\section{Trabalhos Futuros}
%

