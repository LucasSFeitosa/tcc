\chapter{Conclusão}
\label{cap-conclusoes}

A gerência de configuração de software é uma importante área da engenharia de software,
visto que o software precisa estar implantado e disponível para uso, para que ofereça
valor ao usuário final. Uma atividade importante neste cenário é a implantação
automatizada de software, que facilita a instalação de ferramentas, principalmente
para usuários que não possuam conhecimento técnico para realizar instalações complexas.

Observou-se que outros trabalhos já utilizam de conceitos de implantação automatizada,
porém nenhuma delas voltados a aplicações em sistemas Debian GNU/Linux, que contém
pacotes de algumas aplicações web livres bastante conhecidas, como wordpress e
owncloud. Porém, foi visto que a apenas a instalação do pacote não é suficiente,
existem ainda, algumas atividades a serem feitas, e nem sempre o usuário detém
o conhecimento necessário para configurar as aplicações, com isso, dado o objetivo
deste trabalho, e as motivações, foi feito evolução da ferramenta Shak, para
que usuários possam instalar aplicações com apenas um clique num sistema Debian
GNU/Linux.

No decorrer deste trabalho, alguns resultados foram alcançados. Primeiramente, o
pesquisador estabeleceu o conhecimento sobre implantação automatizada
de software e tecnologias que permitam tal implementação, como por exemplo o Chef,
sendo possível automatizar todo o processo de implantação de um software,
desde o planejamento até a execução. Também foi conhecido ferramentas que são
utilizadas para implantação de aplicações de forma automatizada, como por exemplo
o JuJu e o Bitnami, que ajudam usuários que não possuem conhecimento a
instalarem aplicações com facilidade.

Além disso, também foram adquiridos os conhecimentos sobre \textit{DevOps}, que é um conjunto
de práticas e princípios para a implantação ágil de software, que foram bastante
úteis ao trabalho, pois também trata de automação de implantação de software. Também
foi adquirido conhecimentos sobre hospedagem virtual, que é uma forma de manter
várias aplicações no mesmo servidor com um único endereço IP e vários domínios diferentes.

Em relação a construção de solução, foram realizadas melhorias na ferramenta Shak,
provendo quatro novas aplicações, sendo elas Owncloud, Wordpress, MoinMoin e
servidor de e-mail, todas sendo implantadas automaticamente e
levando em consideração o uso de protocolos seguros como \textit{HTTP} junto ao SSL e SMTP
junto ao SSL. Logo também foi adicionado a
ferramenta Shak suporte a criação de certificados autoassinados, permitindo que
as aplicações possam usar certificados para configurar tais protocolos. Ainda foi
adicionado ao Shak o suporte a as aplicações web sempre utilizarem
o protocolo HTTPS, evitando assim implantações inseguras, independente da ferramenta.

Também foi configurado o suporte a múltiplas instâncias nas aplicações disponíveis
no shak, o Wordpress e MoinMoin já previa tal suporte, bastando configurá-lo, porém no Owncloud
foi necessário o envio de uma correção para o mantenedor do pacote Debian, visto que
é uma funcionalidade nova no Owncloud, permitindo assim a criação de múltiplas
instâncias do Owncloud, em ambos os casos foi utilizado a técnica de hospedagem
virtual, disponibilizada pelo servidor web Nginx. Além disso, foi proposto um protótipo
da interface web para a ferramenta, que seria a forma dos usuários acessarem a
ferramenta sem precisar utilizar um terminal.

Por fim, todos esses resultados são importantes para o objetivo deste trabalho, que vão
desde compreender como é o processo de implantação de um software, a evolução
da ferramenta Shak, e até os cuidados
a se tomar ao implantar uma aplicação, sendo assim, contribuindo para o resultado
deste trabalho.

\section{Trabalhos Futuros}
%
Algumas melhorias na ferramenta Shak não puderam ser implementadas  no contexto
deste trabalho, destaca-se o suporte a servidores autoassinados utilizando
let's encrypt, que trariam ganhos para os usuários finais, visto que os certificados
autoassinados que utilizam a ferramenta openssl não são reconhecidos automaticamente
pelos navegadores, precisando ainda, de uma autorização do usuário. Já
com let's encrypt esse problema não aconteceria, visto que os certificados
gerados a partir desta tecnologia é reconhecido pelos navegadores mais modernos.

Outra melhoria seria o suporte a criação de máquinas virtuais de forma automatizada, visto
que ainda é necessário que o usuário instale a ferramenta no servidor destino, uma
solução viável é que o usuário possa instalar a partir de seu computador pessoal,
 apontando apenas o endereço do servidor destino, tanto a partir do terminal, como
a partir da interface web, facilitando ainda mais o uso da ferramenta para seus usuários, ou
até mesmo utilizar a ferramenta Shak como um serviço, sendo assim, disponibilizando uma
versão do Shak na nuvem, para que qualquer usuário com conexão a internet e um servidor
disponível pudesse instalar aplicações web em que o Shak dê suporte.
