\chapter{Conclusão}
\label{cap-conclusoes}

A gerência de configuração de software é uma importante área da engenharia de software 
visto que o software precisa estar implantado e disponível para uso, para que ofereça
valor ao usuário final. Uma atividade importante nesse cenário é a implantação
automatizada de software, que facilita a instalação de ferramentas, principalmente
para usuários que não possuam conhecimento técnico para realizar instalações complexas.

Observou-se que outros trabalhos já utilizam os conceitos de implantação automatizada,
porém nenhuma delas voltados a aplicações em sistemas Debian GNU/Linux, que possibilita
algumas vantagens como suporte à segurança.
 
Também, foi visto que apenas a instalação dos pacotes não é suficiente. Existem outras atividades a serem feitas e nem sempre o usuário detém
o conhecimento necessário para configurar as aplicações. Com isso, dado o objetivo
deste trabalho e as motivações, foi feito a evolução da ferramenta Shak, para
que usuários possam instalar aplicações com ``apenas um clique'' num sistema Debian
GNU/Linux.

No decorrer deste trabalho, alguns resultados foram alcançados. Primeiramente,
estabeleceu-se o conhecimento sobre implantação automatizada
de software e tecnologias que permitam tal implementação. Além disso, 
também foi discutido o conceito de \textit{DevOps}. 
 
Em relação à evolução da ferramenta, foram realizadas melhorias na ferramenta Shak
provendo quatro novas aplicações, sendo elas: Owncloud, Wordpress, MoinMoin e
servidor de e-mail, além de iniciar a implantação das aplicações Noosfero e Roundcube. Isso
com todas as aplicações com suporte a múltiplas instâncias, e utilizando protocolos seguros.

Por fim, todos esses resultados são importantes para o objetivo deste trabalho, que vão
desde como implantar aplicações web de forma automatizada em sistemas Debian/GNU Linux, 
até compreender como é o processo de implantação de um software. Além da evolução
da ferramenta Shak, e até os cuidados a se tomar ao implantar uma aplicação.
 
\section{Trabalhos Futuros}
%
Algumas melhorias na ferramenta Shak não puderam ser implementadas no contexto
deste trabalho, destaca-se o suporte a servidores autoassinados utilizando
Let's encrypt, que trariam ganhos para os usuários finais, visto que os certificados
autoassinados que utilizam a ferramenta Openssl não são reconhecidos automaticamente
pelos navegadores. 

Outra melhoria seria o suporte a criação de máquinas virtuais de forma automatizada, visto
que é necessário que o usuário instale a ferramenta no servidor destino. Uma sugestão
seria integrar com ferramentas que disponibilizam servidores na nuvem, como a 
DigitalOcean \footnote{www.digitalocean.com/}. 
 
Também como trabalhos futuros, finalizar as aplicações Noosfero e Roundcube, além
de suportar novas aplicações no Shak.
