\chapter{Visualização de Software}
\label{cap-proposta}


\section{Considerações Iniciais}

O objetivo da vizualização de software é trazer uma abstração que permita ao
usuário entender conceitos e explorar um conjunto de relações para compreender
melhor o software, suas estruturas e funções.

\section{Visualização da Informação}

A visualização de informação “é uma área emergente da ciência que estuda
formas de apresentar dados visualmente de tal modo que relações entre os mesmos
sejam melhor compreendidas ou novas informações possam ser descobertas.”
(Nascimento e Ferreira, 2005, p. 1263).

O termo visualização da informação foi introduzido por Card, Mackinlay e
Shneiderman (1999) que conceitua como a utilização de representações visuais
interativas apoiadas por computador de dados abstratos para ampliar a cognição.

Autores como Freitas et al. (2001) definem a visualização da Informação como
uma área da ciência com finalidade de estudar as principais formas de
representações gráficas para apresentar a informação de modo a contribuir para
uma melhor percepção e entendimento delas.

Para Ware (2000), a visualização da Informação oferece cinco vantagens quando
empregada de forma eficiente:

\begin{itemize}
  \item Compreensão: a visualização permite a compreensão degrande quantidade de informação;
  \item Percepção: a visualização revela propriedades do dado que não podem ser antecipadas;
  \item Ccontrole de qualidade: a visualização permite o controle de qualidade dos dados,
   porque os problemas se tornam imediatamente aparentes;
  \item Foco no contexto: a visualização facilita a compreensão de um aspecto dentro do contexto
  geral dos dados em que esse encontra;
  \item Interpretação: a visualização apoia a formaçãode hipóteses que propiciam futuras investigações.
\end{itemize}

-> TODO procurar  formas de visualização.

Como a visão é o sentido dominante humano (WADE, 2001) e a maioria do cérebro
humano lida com processamento e análise de imagens visuais (BURKHARD, 2004),
a visualização de informação serve para maximizar a capacidade humana de cognição
através da representação visual das informações.

\section{Tecnicas de visualização}

Dados puramente coletados podem não trazer a informação desejada, por isso esses
dados precisam ser processados para que possa gerar alguma informação que tenha
valor a determinado contexto.

No contexto de engenharia de software podemos extrar vários dados de código
fonte como: quantidade de classes, quantidade de métodos por classe,
complexidade ciclomática cobertura de código, quantidade de Parâmetros por
método, dentre várias outras.

Mas ter apenas esses dados não garante a interpretação ou entendimento correto
do que cada dado quer dizer, para isso esses dados devem de  ser
iterpretados e mostrados de uma forma em que o usuário consiga visualizar e
interagir com a informação de forma rápida e eficiente fornecendo um feedback
imediato.

Segundo Valiati (2008), os mecanismos de visualização contribuem para melhor
entender a representação dos dados. Na visão de Yamaguchi (2010), através de recursos
interativos como, por exemplo, a filtragem, o usuário pode eliminar os elementos da
visualização que não interessam no momento da analise.

Para Grinstein e Ward (2002) as técnicas de visualização suportam três categorias
de tarefas:

\begin{itemize}
  \item Exploração de dados: o usuário não deve ter necessariamente um
  conhecimento a priori sobre os dados, nem metas de exploração precisa. O
  usuário procura uma estrutura significativa, padrões ou tendências e,
  consequentemente, para a formulação de uma hipótese relevante.
  \item Confirmação de uma hipótese: o usuário procura padrões ou estrutura de
  dados (o objetivo do usuário é verificar uma hipótese). Podem ser necessárias
  ferramentas analíticas para confirmar ou refutar a hipótese.
  \item Produção da apresentação: o usuário tem uma hipótese validada e seu
  objetivo é comunicar o conhecimento para outras partes. Com foco de refinar
  a visualização para otimização e a apresentação.
\end{itemize}

->TODO Falar sobre as técnicas muitas duvidas ainda =/

\section{Ferramentas de visualização}

Ferramentas de visualização de software uitilizam técnicas gráficas para tornar
o software visível através da exibição de programas, artefatos de programas e
comportamento do programa. A idéia essencial é que as representações visuais
podem tornar o processo de compreensão mais fácil (BALL e Erick).

-> Fazer aquela tabela das ferramentas que eu tinha montado e falar um pouco sobre
as ferramentas e no que elas apoiam.


\section{Perspectiva do usuário}

No contexto de engenharia de software e a visualização de métricas de código
queremos que o usuário possa tirar conclusões e facilitar na interpretação desse
 conjunto de dados, e com isso ajudar no apoio de decisões.

\section{Considerações Finais}
%
