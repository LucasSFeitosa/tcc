\chapter{Introdução}
\label{cap-introducao}

Segundo \cite{giles2010psychology}, a Internet emergiu no contexto da Guerra Fria na década
de 60, em um projeto exército norte-americano. Os motivos de sua criação eram:
criar um sistema de informação e comunicação em rede, que
sobrevivesse a um ataque nuclear e dinamizar a troca de informações entre os centros de
produção científica. Os militares pensaram que um único centro de computação
centralizando toda informação era mais vulnerável a um ataque nuclear do que vários
pontos conectados em rede, pois assim a informação estaria espalhada por inúmeros
centros computacionais pelo país.

Desde a sua origem, a internet foi projetada para ser uma rede
descentralizada e flexível \cite{galaxia}. A grande maioria dos protocolos 
utilizados na rede não dependem de pontos centrais para funcionar. 
De fato, é de conhecimento geral
que, por exemplo, usuários com contas de e-mails em provedores diferentes podem
trocar mensagens entre si sem qualquer problema. Considerando que no
passado, a maioria dos softwares eram como uma ilha isolada,
hoje, os softwares são disponibilizados através da Web ou conectado através da Internet,
como por exemplo: as redes sociais e mercados de compras online \cite{6265084}.

Os novos sistemas voltados para internet estão mais complexos, considerando que,
no passado, a maioria dos softwares eram
como uma ilha isolada, hoje espera-se que os softwares estejam disponíveis 
através da Web ou conectado através da internet \cite{byhand},
essa alta sofisticação implicam numa alta complexidade de
desenvolvimento e manutenção, que faz com que a
implantação de software e a manutenção de serviços exija conhecimento técnico
especializado. Para organizações, isto implica num maior
custo para manter sistemas e aplicações. Para indivíduos, isto quase sempre
impossibilita a manutenção de serviços próprios e leva à procura por serviços
gratuitos.

O mesmo vale para outras aplicações, em especial para aquelas aplicações
que são utilizadas na web, como por exemplo: um blog, uma rede social ou site de
notícias. Eles podem estar hospedados em qualquer provedor de internet, e mesmo os
usuários de outros provedores, também terão acesso a elas.

Existe atualmente uma tendência de centralização da internet,\cite{vaz} mostra 
através de uma série de dados empíricos que a internet está centralizada, 
A imensa maioria dos usuários individuais utilizam contas de e-mail em algum
poucos grandes provedores \cite{shak2015}. O mesmo acontece com utilitários como armazenamento
de documentos na nuvem, documentos como fotos, vídeos, arquivos, são geralmente
armazenados em ferramentas conhecidas, como Google Drive ou Dropbox, estes serviços 
são geralmente serviços gratuitos, mas os termos com os quais os usuários concordam, 
muitas vezes, permitem aos provedores fazerem uso de informações pessoais para usos comerciais.

De acordo com \cite{shak2015} essa excessiva centralização traz os seguintes
problemas para a sociedade:

\textbf{Riscos à privacidade}

A quantidade de informações pessoais fornecidas a serviços online
centralizados por uma quantidade imensa de pessoas faz com que estes serviços
representem um risco muito grande à privacidade dos seus usuários. Diversos
destes provedores mencionam explicitamente em seus termos de serviço que
informações pessoais serão usadas para fins de marketing como exibição de
propaganda, e não há forma de saber se são feitos outros usos dessas
informações.

Mesmo em mente que os serviços oferecidos pelos provedores são
confiáveis, eventuais problemas de segurança em seus sistemas podem deixar as
informações pessoais de seus usuários vulneráveis. Se essas informações estivessem dispersas em
diversos provedores de serviços, uma eventual invasão afetaria uma quantidade
muito menor de pessoas.

\textbf{Subordinação dos usuários aos provedores}

Os grandes provedores de serviços acumulam uma quantidade excessiva de poder
sobre a sociedade. Uma vez que a imensa maioria dos usuários da internet usa
os seus serviços, estas empresas têm a capacidade de usar dados dos usuários para
seus interesses comerciais.

\textbf{Pontos centrais de falha}

Provedores centralizados também se tornam pontos centrais de falha, e uma
eventual indisponibilidade de seus serviços podem consequências econômicas
gravíssimas. Na medida em que mais e mais pessoas e negócios dependem do seu
bom andamento de suas atividades, os riscos à economia mundial se tornam cada
vez maiores.

Mesmo que toda a sociedade tivesse plena consciência destes problemas causados
pela centralização dos serviços digitais, a maioria dos usuários ainda
continuaria utilizando-os, visto que para trocar os serviços utilizados, seria
necessário que existisse uma alternativa à centralização da internet através
de uma solução que elimina barreira técnica para que usuários interessados
possam ter servidores próprios com aplicações sem a necessidade de conhecimento
técnico especializado .

A grande dificuldade de prover essa alternativa é que os sistemas voltado para a
Internet consistem em muitas partes que são complicadas de gerenciar, alguns
requisitos como disponibilidade e desempenho são requisitos indispensáveis, além de ser
necessário gerenciar as aplicações em seus devidos servidores, gerenciamento de
banco de dados, e configurações de recuperação de dados, além de administrar uma
infinidade de bibliotecas de terceiros e serviços online \cite{6265084}.

Além disso, também existe o impeditivo dos usuários terem que hospedar seus próprios serviços,
porém, nos últimos anos este fator vem perdendo relevância em função de
dois fenômenos. O primeiro deles é o barateamento do acesso a servidores virtuais
privados decorrente os avanços da computação em nuvem. Por outro lado, a
disponibilidade de servidores físicos de dimensão reduzida e baixo consumo de
energia, em conjunto com o barateamento de conexões de banda larga à internet \cite{shak2015}.

Esses impeditivos atingem de fato o usuário final, e a decisão
acaba sendo de utilizar serviços gratuitos das grandes empresas, em troca de acesso
às suas informações pessoais, cedidas de forma consciente ou não. Porém, de acordo
com \cite{terra2006comunicaccao} cada vez mais, caminhamos
para uma mídia que atenda às necessidades do indivíduo. O usuário será o responsável
por aquilo que deseja consumir na rede, essa mídia denomina-se U-Media (You-Media).
São inerentes colaboração à U-Media: contribuição e comunidades, participação
e customização, além da descentralização dos serviços.

Este trabalho relata a evolução da ferramenta Shak, que é uma ferramenta para implantação
automatizada de aplicações web em sistema Debian GNU/Linux, que tem como objetivo fornecer
uma alternativa à centralização da internet através de uma solução que elimina
barreira técnica para que usuários interessados possam ter servidores próprios
com aplicações sem a necessidade de conhecimento
técnico especializado. Este trabalho também relata uma pesquisa para
descobrir as ferramentas que são utilizadas para automatizar implantação de aplicações
como trabalhos relacionados, e também as evoluções feitas na ferramenta de acordo
com os objetivos definidos, os desafios a serem superados, e a validação dos
resultados alcançados.

O trabalho está organizado na seguinte forma:  O Capítulo \ref{cap-referencial}
trás o referencial teórico necessário para apoio o desenvolvimento da solução,
como a gerência de configuração de software, o processo de implantação de software
 e métodos e técnicas para implantação automatizada de software, já o Capítulo 
\ref{cap-metodologia}
trás a definição e preparação dos estudos, que envolve os trabalhos relacionados, 
envolve também como será construído a solução e também como a solução será validada. 
Já no Capítulo \ref{cap-resultados} contém os resultados alcançados, e por fim as
considerações finais no Capítulo \ref{cap-conclusoes}.

\section{Problema}

O que problema que envolve todo este trabalho é:

\begin{center}
  \textit{
  Como implantar aplicações web em sistema Debian GNU/Linux de forma automatizada
  e segura?
}
\end{center}

\section{Objetivo}

O objetivo deste trabalho consiste na contribuição da construção de uma ferramenta
que possa automatizar instalação e configuração de aplicações web em sistemas
Debian GNU/LINUX, a partir de pacotes que sejam distribuídos oficialmente pelo
Debian, mostrando os aspectos mais importantes que devem ser tratados durante
todo o processo de configuração e instalação de um software, facilitando assim, que
tanto usuários como desenvolvedores possam implantar aplicações com apenas uma
instrução.

\section{Contribuições}

As contribuições deste trabalho são:

\begin{description}
  \item [Contribuições Tecnológicas]\
\end{description}
    \begin{enumerate}
      \item \textbf{CT1} Implantação de sistemas Debian GNU/LINUX,
        \begin{enumerate}
          \item Implantação de aplicações web com certificados digitais autoassinados.
          \item Implantação de múltiplas aplicações web em um mesmo servidor utilizando
          hospedagem virtual.
          \item Implantação de automatizada de aplicações web utilizando pacotes
          distribuídos oficialmente pelo Debian.
        \end{enumerate}
    \end{enumerate}

\begin{description}
  \item [Contribuições Científicas]\
\end{description}
    \begin{enumerate}
      \item \textbf{CC1} Gestão de configuração de software,
        \begin{enumerate}
          \item Estudo teórico sobre implantação automatizada de software.
          \item Aplicação prática do modelo de implantação de software sugerido
pela OMG.
        \end{enumerate}
    \end{enumerate}


\section{Motivação}
\label{sec:motivacao}
A motivação deste trabalho é a continuidade do projeto iniciado pelo Google Summer
of Code 2015, Google Summer of Code é um programa global que oferece aos
alunos uma oportunidade para desenvolverem projetos de código aberto durante o período
de verão americano, de maio até agosto \cite{gsoc2015}.

O projeto aprovado para o Google Summer of Code 2015 foi "Automated Configuration
of Packaged Web Applications", com a organização Debian Project e com a mentoria de
Antonio Terceiro. A ideia inicial do projeto era uma ferramenta que
pudesse ajudar os usuários que não possuem conhecimento técnico, poderem
instalar e configurar ferramentas a partir de pacotes Debian, e tudo isso com
apenas 1 comando (ou um clique). Dentro desse projeto a minha participação
inicialmente seria adicionar o máximo de aplicações possíveis na ferramenta,
e sendo aplicações que fossem bastante conhecidas por usuários da internet. Com o
 fim do programa, viu-se a possibilidade de dar continuidade ao projeto, ajudando
na construção da ferramenta Shak, para a implantação automatizada de aplicações.

Outra motivação foi a oportunidade de contribuir com software livre, visto que é
um projeto junto a organização Debian Project, o que poderia ser de grande proveito
para a formação acadêmica, no qual seria possível contribuir com outras
aplicações aumentando o conhecimento como engenheiro de software, interagir com
pessoas com bastante conhecimento na área de desenvolvimento de software e também
crescer como profissional.
