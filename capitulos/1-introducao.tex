\chapter{Introdução}
\label{cap-introducao}

\section{Justificativa e Motivação}

\section{Objetivos}

\section{Organização do trabalho}

\section{Contribuições}
%
\begin{description}
  \item [Contribuições Tecnológicas]\
\end{description}
    \begin{enumerate}
      \item \textbf{CT1} - Evolução da plataforma livre Mezuro de monitoramento de código-fonte:
        \begin{enumerate}
          \item Evolução da arquitetura do Mezuro para melhorar sua modularização e flexibilidade.
          \item Evolução da configuração de métricas do Mezuro
          \item Evolução de mecanismos de visualização de software do Mezuro
        \end{enumerate}
      \item \textbf{CT2} - Criação de um ambiente de \emph{Data Warehousing} para monitoramento dos cenários de decisão no contexto de vulnerabilidade de software:
            \begin{enumerate}
              \item Criação de modelo dimensional que, diferente de um modelo relacional utilizado para modelagem de banco de dados, permitiu melhor performance para processamento analítico dos dados.
              \item Implementação de mecanismos de extração, transformação e carregamento da base de dados a partir dos resultados das ferramentas de análise estática.
              \item Geração Cubo de dados para definição das agregações para a manipulação dos dados do \emph{Data Warehouse}.
          \item Configuração dos mecanismos de visualização do cubo e geração de relatórios.
            \end{enumerate}
    \end{enumerate}
\begin{description}
  \item [Contribuições Científicas]\
\end{description}
     \begin{enumerate}
      \item \textbf{CC1} - Catalogar definições teóricas a respeito dos principais conceitos relacionados à vulnerabilidades de software.
      \item \textbf{CC2} - Estudo teórico sobre a relação de vulnerabilidades de software com o \emph{design}
      \item \textbf{CC3} - Definição de cenários a partir de estudos teóricos para melhorar a interpretação e tomada de decisão sobre métricas estáticas de código-fonte.
     \end{enumerate}
