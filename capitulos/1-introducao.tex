\chapter{Introdução}
\label{cap-introducao}

A Internet emergiu no contexto da Guerra Fria na década
de 60, em um projeto do exército norte-americano. Os motivos de sua criação eram:
criar um sistema de informação e comunicação em rede, que
sobrevivesse a um ataque nuclear e dinamizar a troca de informações entre os centros de
produção científica. Os militares pensaram que um único centro de computação
centralizando toda informação era mais vulnerável a um ataque nuclear do que vários
pontos conectados em rede, pois assim, a informação estaria espalhada por inúmeros
centros computacionais pelo país \cite{giles2010psychology}.

Desde a sua origem a Internet foi projetada para ser uma rede
descentralizada e flexível \cite{galaxia}. A maioria dos protocolos
utilizados na rede não dependem de pontos centrais para funcionar. 
Considerando que no
passado a maioria dos softwares eram como uma ilha isolada, hoje os softwares 
estão mais complexos pois se caracterizam por um conjunto de componentes, que 
são disponibilizados através da web ou conectado através da Internet,
como por exemplo: as redes sociais e mercados de compras online \cite{byhand}.

Essa grande sofisticação implica numa alta complexidade de
desenvolvimento e manutenção, que faz com que a
implantação de software e a manutenção de serviços exija conhecimento técnico
especializado. Para as organizações isto implica num maior
custo para manter sistemas e aplicações. Para os indivíduos isto quase sempre
impossibilita a manutenção de serviços próprios, e os leva à procura por serviços
gratuitos \cite{shak2015}.

Existe atualmente uma tendência de centralização da Internet. \citeonline{vaz}
mostra
através de uma série de dados empíricos que a Internet está centralizada. Isso
implica que os serviços que são utilizados pelos usuários, na maioria
das vezes, são serviços que são providos por grandes empresas. 
Um exemplo disso é que os usuários que utilizam contas de e-mail preferem usar algum
dos grandes provedores de e-mail, como Gmail\footnote{http://gmail.com/}. 

Isso também acontece com aplicações responsáveis por armazenar dados dos usuários,
com armazenamento de documentos na nuvem. Alguns exemplos são 
documentos como: fotos, vídeos e arquivos. Eles são geralmente armazenados em 
ferramentas conhecidas, como Google Drive 
\footnote{https://www.google.com/intl/pt-BR/drive/} ou Dropbox \footnote{https://www.dropbox.com/pt\_BR/}. 

Estes serviços não custam nada ao usuário, mas quando o usuário aceita os
termos de uso, é possível que algumas empresas utilizem das informações pessoais dos
usuários para usos comerciais. Essa excessiva centralização traz problemas para os usuários que utilizam deste tipo de ferramentas, tais como: riscos a privacidade, subordinação dos usuários
aos provedores e pontos centrais de falhas \cite{shak2015}. 

Para solucionar este problema, seria necessário a existência de uma alternativa à 
centralização da Internet, através de uma solução que elimina barreira técnica, e 
para que usuários interessados possam ter servidores próprios com aplicações web, 
sem a necessidade de conhecimento técnico especializado.

A dificuldade de prover essa alternativa é que os sistemas voltado para a
Internet consistem em muitas partes que são complicadas de gerenciar. Alguns
requisitos como: disponibilidade e desempenho são indispensáveis, além de ser
necessário gerenciar as aplicações em seus devidos servidores, com o gerenciamento de
banco de dados e rotinas de recuperação de dados, além de administrar uma
infinidade de bibliotecas de terceiros e serviços online \cite{6265084}.

Além disso, existe o impeditivo dos usuários terem que hospedar seus 
próprios serviços. Porém, ultimamente este fator vem perdendo relevância em função de
dois fenômenos. O primeiro deles é o barateamento do acesso a servidores virtuais
privados, decorrente dos avanços da computação em nuvem. Por outro lado, a
disponibilidade de servidores físicos de dimensão reduzida e baixo consumo de
energia, em conjunto com o barateamento de conexões de banda larga à Internet \cite{shak2015}.

Esses impeditivos atingem de fato o usuário final, e a decisão
acaba sendo de utilizar os serviços gratuitos das grandes empresas. Porém, 
cada vez mais caminhamos para uma mídia que atenda às necessidades do indivíduo. 
O usuário será o responsável por aquilo que deseja consumir na rede, essa mídia 
denomina-se You-Media \textit{(U-Media)}. São inerentes colaboração à U-Media: contribuição 
e comunidades, participação e customização, além da descentralização dos 
serviços \cite{terra2006comunicaccao}. 

Este trabalho relata a evolução da ferramenta Shak, que é uma ferramenta para implantação
automatizada de aplicações web em sistema Debian GNU/Linux, que tem como objetivo fornecer
uma alternativa à centralização da Internet. Essa alternativa é dada através de 
uma solução que elimina barreira técnica, para que usuários interessados possam 
ter servidores próprios com aplicações web, sem a necessidade de conhecimento técnico 
especializado. 

O problema que envolve este trabalho é:

\begin{center}
  \textit{
  Como implantar aplicações web em sistema Debian GNU/Linux de forma automatizada
  e segura?
}
\end{center}

\section{Objetivo}

O objetivo deste trabalho consiste na contribuição em uma ferramenta
que possa automatizar instalação e configuração de aplicações web em sistemas
Debian GNU/Linux, a partir de pacotes que sejam distribuídos oficialmente pelo
Debian. Mostrando também os aspectos mais importantes que devem ser tratados durante
todo o processo de configuração e instalação de um software, facilitando assim, que
tanto usuários e desenvolvedores possam implantar aplicações com apenas uma
instrução.

As contribuições deste trabalho são:

\begin{description}
  \item [Contribuições Tecnológicas]\
\end{description}
    \begin{enumerate}
      \item \textbf{CT1} Implantação de sistemas Debian GNU/LINUX,
        \begin{enumerate}
          \item Implantação de aplicações web com certificados digitais autoassinados.
          \item Implantação de aplicações web com suporte a múltiplas instâncias no mesmo servidor.
          hospedagem virtual.
          \item Implantação de automatizada de aplicações web utilizando pacotes
          distribuídos oficialmente pelo Debian.
        \end{enumerate}
    \end{enumerate}

\begin{description}
  \item [Contribuições Científicas]\
\end{description}
    \begin{enumerate}
      \item \textbf{CC1} Gestão de configuração de software,
        \begin{enumerate}
          \item Estudo teórico sobre implantação automatizada de software.
        \end{enumerate}
    \end{enumerate}


\section{Organização do trabalho}
\label{sec:organizacao}

O trabalho está organizado na seguinte forma: O Capítulo \ref{cap-referencial}
traz o referencial teórico necessário para apoio o desenvolvimento da solução,
como a gerência de configuração de software, o processo de implantação de software
 e métodos e técnicas para implantação automatizada de software. O Capítulo
\ref{cap-metodologia}
fala sobre a definição e preparação dos estudos, que envolve os trabalhos relacionados,
envolve também como será construída a solução e também como a solução será validada.
Já no Capítulo \ref{cap-resultados} contém os resultados alcançados, e por fim as
considerações finais no Capítulo \ref{cap-conclusoes}.

