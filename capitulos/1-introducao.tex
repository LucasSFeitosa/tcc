\chapter{Introdução}
\label{cap-introducao}

A forma como as equipes de desenvolvimento entregam software estão mudando, as
necessidades do mercado estão mudando continuamente, a tecnologia está mudando
rapidamente, ao longo dos anos as organizações adotaram muitos processos ágeis para
otimizações em seu desenvolvimento de software \cite{7173368}. As práticas ágeis
têm aumentado a capacidade do software ser desenvolvido continuamente, contudo,
a implantação do software ainda é uma atividade que pode ser complexa, principalmente
pelo fato das aplicações voltadas para internet estarem evoluindo, se tornando
aplicações mais complexas que envolvem vários componentes, integrando até mesmo
aplicações em arquiteturas diferentes em um único pacote \cite{6265084}, essas
arquiteturas diferentes envolvem aplicações em diferentes linguagens de programação
, diferentes dependências e configurações, e isso impacta diretamente na implantação
 de uma aplicação web.

Sistemas voltado para a Internet consistem em muitas partes que são complicadas
de gerenciar, alguns requisitos como disponibilidade e desempenho são requisitos
indispensáveis, é também necessário gerenciar as aplicações em seus devidos
servidores, gerenciamento de banco de dados, e configurações de recuperação de dados
, além de administrar uma infinidade de bibliotecas de terceiros e serviços online.
 Considerando que no passado, a maioria dos softwares eram como uma ilha isolada,
hoje, os softwares são disponibilizados através da Web ou conectado através da Internet,
como por exemplo as redes sociais e mercados de compras online \cite{6265084}.

Em uma implantação de software, executar todas essas tarefas de forma manual pode ser
uma atividade complicada, porém felizmente, é possível controlar e conquistar
esta complexidade, utilizando sistemas de configuração de ferramentas e de
gerenciamento de implantação. Que vem para facilitar a instalação e configuração de ferramentas,
de forma que automatize todas as tarefas de uma implantação de software.

A implantação de software é um processo que vai desde a sua aquisição até o momento
em que ela é disponibilizada para uso \cite{OMG06}, um exemplo de aquisição de software
é quando um usuário busca um software desejado a partir dos pacotes disponíveis em uma
distribuição linux. As distribuições linux que optam por disponibilizar
pacotes mantém uma infraestrutura de servidores como fonte, sendo possível assim
instalar uma aplicação desejada. Como visto, aplicações web podem conter tarefas
que são complicadas dentro da sua implantação, existem diferentes perfis de
usuários que buscam instalar aplicações \cite{araujo2011apprecommender}, um seria
um usuário que detém um certo conhecimento na área de computação, que poderia
cuidar dessas tarefas mais complicadas, e usuários regulares de computadores,
que estão interessados apenas no uso das aplicações.

Um problema desse contexto é que ao instalar um pacote, são feitos todos os
procedimentos necessários para a instalação de uma aplicação, porém, existem
ainda casos em que o usuário precise fazer algumas configurações específicas após
a instalação desses pacotes, alguns exemplos são a configuração do banco de dados
da aplicação e customização a a partir de arquivos de configuração, e quando se
trata de segurança, ainda é necessário gerenciar certificados de segurança para
poder utilizar protocolos seguros como HTTPS. Para usuários regulares essas
atividades não são comuns, e o uso de alguma ferramenta que automatize todo
esse processo de implantação pode ser uma solução a este problema, cuidando das
tarefas mais complexas. Já existem diferentes ferramentas para a solução desse
problema, que disponibilizam a aplicação pronta para uso ao usuário, essas ferramentas
cuidam de todo o processo de implantação de software, como por exemplo \cite{bitnami}
e \cite{sandstormio} que são ferramentas que disponibilizam a aplicação ao
usuário com apenas um clique.

Este trabalho relata a evolução de uma ferramenta para implantação automatizada
de aplicações web em sistema Debian GNU/Linux, que vão desde uma pesquisa para
descobrir as ferramentas que são utilizadas para automatizar implantação de aplicações
como trabalhos relacionados, até a contribuição da construção de uma solução, de
acordo com os objetivos definidos, os desafios superados, e a validação dos
resultados alcançados.

O trabalho está organizado na seguinte forma:  O Capítulo \ref{cap-referencial}
trás uma brave introdução em alguns aspectos necessários para a construção da solução,
como implantação de software, métodos e técnicas, já o Capítulo \ref{cap-metodolgia}
trás a metodologia do trabalho, que envolve os trabalhos relacionados, envolve também
como será construído a solução e também como a solução será validada. Já no Capítulo
\ref{cap-resultados} contém os resultados alcançados até o momento, e por fim as
considerações parciais no Capítulo \ref{cap-conclusoes}.

\section{Problema}

O que problema que envolve todo este trabalho é:

\begin{center}
  \textit{
  Como implantar aplicações web em sistema Debian GNU/Linux de forma automatizada e
  e segura?
}
\end{center}

\section{Objetivo}

O objetivo deste trabalho consiste na contribuição da construção de uma ferramenta
que possa automatizar instalação e configuração de aplicações web em sistemas
Debian GNU/LINUX, a partir de pacotes que sejam distribuídos oficialmente pelo
Debian, mostrando os aspectos mais importantes que devem ser tratados durante
todo o processo de configuração e instalação de um software.

%------------------------------------------------------------------------------%
%\section{Contribuições}
%
%As contribuições parciais deste trabalho são:

%\begin{description}
%  \item [Contribuições Tecnológicas]\
%\end{description}
%    \begin{enumerate}
%      \item \textbf{CT1} - Implantação de sistemas Debian GNU/LINUX,
%        \begin{enumerate}
%          \item Gerência de certificados digitais autoassinados.
%          \item Implantação de multiplas aplicações em um mesmo servidor utilizando
%          hospedagem virtual.
%          \item Implantação de automatizada de aplicações web utilizando pacotes
%          distribuídos oficialmente pelo Debian.
%        \end{enumerate}
%    \end{enumerate}
%\begin{description}

%------------------------------------------------------------------------------%
\section{Motivação}

A motivação deste trabalho é a continuidade do projeto iniciado pelo Google Summer
of Code 2015, Google Summer of Code é um programa global que oferece aos
alunos uma oportunidade para desenvolverem projetos de código aberto durante o período
de verão americano, de maio até agosto \cite{gsoc2015}.

O projeto aprovado para o google summer of code 2015 foi "Automated configuration
of packaged web applications", com a organização Debian Project e com a mentoria de
Antonio Terceiro. A ideia inicial do projeto era uma ferramenta que
pudesse ajudar os usuários que não possuem conhecimento técnico, poderem
instalar e configurar ferramentas a partir de pacotes debian, e tudo isso com
apenas 1 comando (ou um clique). Dentro desse projeto a minha participação
inicialmente seria adicionar o máximo de aplicações possíveis na ferramenta,
e sendo aplicações que fossem bastante conhecidas por usuários da internet. Com o
 fim do programa, viu-se a possibilidade de dar continuidade ao projeto, ajudando
na construção da ferramenta Shak, para a implantação automatizada de aplicações.

Outra motivação foi a oportunidade de contribuir com software livre, visto que é
um projeto junto a organização Debian Project, o que poderia ser de grande proveito
para a formação acadêmica, no qual seria possível contribuir com outras
aplicações aumentando o conhecimento como engenheiro de software, interagir com
pessoas com bastante conhecimento na área de desenvolvimento de software e também
crescer como profissional.
