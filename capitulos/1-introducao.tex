\chapter{Introdução}
\label{cap-introducao}

%------------------------------------------------------------------------------%
\section{Problema e Hipótese}

%

Como interpretar o que certos valores de métricas querem dizer em um
determinado contexto?

Quais são as técnicas de visualização de software e como aplicá-las no contexto
de métricas simples e compostas?

Como vizualizar um conjunto de métricas específico?

Como trazer  o acompanhamento de métricas de código ao ciclo de desenvolvimento
de software livre?

(hipotese)
Dada a utilização de métricas de software para avaliação da qualidade do código
fonte, o entendimento dessas métricas são de extrema importância para o
acompanhamento do desenvolvimento do software sendo que a visualização dessas
métricas são de difícil interpretação, principalmente quando combinamos várias
métricas diferentes. Logo é necessário uma forma de visualizar essas métricas


%------------------------------------------------------------------------------%
\section{Justificativa e Motivação}

%

Começar aqui talvez a falar sobre qualidade de software?

%------------------------------------------------------------------------------%
\section{Objetivos}

%

O obejtivo deste trabalho consiste em um estudo teórico e prático sobre
vizualização de métricas de software, além de uma proposta de visualização dos
resulatdo de métricas extraídas de código fonte, no qual o usuário sem
conhecimento sobre métricas possa entender o que tais resultados querem dizer,
podendo assim acompanhar melhor a qualidade do software produzido.
%

Além disso, tem-se como objetivo oferecer uma proposta de visualização de
métricas nos seguintes cenários:

\begin{itemize}
  \item  \textbf{Visuzlização de métricas compostas:} Evolução da visualização
  de métricas  de código na plataforma Mezuro, suportando a visualização de
  diferentes métricas de código e suas combinações.
  \item  \textbf{Biblioteca para auxílio na visualização de métricas:}
  Biblioteca na linguagem ruby para auxílio ao desenvolvimento de aplicações
  que buscam a vizualização de métricas de código fonte.
\end{itemize}

%------------------------------------------------------------------------------%
\section{Organização do trabalho}

%

Esta monografia está dividida em mais outros x capítulos. A seguir, o leitor
encontrará o Capítulo \ref{cap-proposta} onde são definidos a proposta e
a meotodologia do trabalho, além de ferramentas e dados que serão utilizados
como base para pesquisa. No Capítulo \ref{cap-visualizacao}
são introduzidos os principais conceitos de visualização de software
....... \ref{cap-metricas}  serão apresentados o conceito
de metricas de software, sua utilização dentro do ciclo de desenvolvimento de
software, avaliação da qualiadde do software com apoio a métricas, o que devo falar
de metricas?........

No fim desta monografia existem X apêndices que complementam e detalham os
aspectos do presente trabalho. No  \ref{Att:mezuro} apresenta os detalhes
técnicos sobre o Mezuro.

%------------------------------------------------------------------------------%
\section{Contribuições}
%
\begin{description}
  \item [Contribuições Tecnológicas]\
\end{description}
    \begin{enumerate}
      \item \textbf{CT1} - Evolução da visualização de métricas na
      plataforma livre Mezuro:
        \begin{enumerate}
          \item Evolução de mecanismos de vizualização de métricas de código.
          \item Evolução da configuração de métricas do Mezuro.
        \end{enumerate}
      \item \textbf{CT2} - Criação de uma biblioteca para auxiliar o desenvolvimento
      de aplicações com conetexto de vizualização de software:
            \begin{enumerate}
              \item  Criação de uma API em ruby para auxiliar o desenvolvimento
              de aplicações que queiram utilizar a visualização de métricas, no qual
              o desenvolvedor se preocupará apenas em popular a API já colhendo
              como resulatdo a visualização e interpretação das métricas.
            \end{enumerate}
    \end{enumerate}
\begin{description}
  \item [Contribuições Científicas]\
\end{description}
     \begin{enumerate}
      \item \textbf{CC1} - Estudo teórico das técnicas de visualização de software.
      \item \textbf{CC2} - Estudo teórico das técnicas de visualização de métricas de código
      e suas combinações.
      \item \textbf{CC3} - Definição de cenários a partir de estudos teóricos
       com uma forma de combinar métricas de código fonte para atingir resultados
       expressivos dentro do desenvolvimento de software.
     \end{enumerate}
