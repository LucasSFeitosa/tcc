\chapter{Introdução}
\label{cap-introducao}


%isso é bom para introdução
A forma como as empresas entregar software vai
por meio de uma onda de mudanças como o ambiente,
as empresas funcionam, está mudando - as necessidades do mercado
estão mudando continuamente, a tecnologia está mudando
rapidamente, existe mais pressão para se adaptar ao mercado
precisa e entregar rapidamente. As empresas não podem
já não ter recursos para fazer uma torre de menagem do cliente à espera de
6 meses ou 1 ano para uma liberação para vir e, em seguida,
solicitar feedback de cliente sobre como software
se comporta. Os clientes esperam que o engajamento contínuo
de modo que eles podem fornecer feedback contínuo. Dentro
a fim de enfrentar os desafios de hoje, as empresas
precisa ser magra e ágil em todas as fases de
software ciclo de vida de desenvolvimento. Ao longo dos anos
organizações adotaram muitos processos
otimizações em seu desenvolvimento de software (ágeis
transformação) práticas. \cite{7173368}


%fala de ágeis e implantação, bom para introdução
Hoje, o software é desenvolvido em rápida mudança e os mercados imprevisíveis,
com os requisitos do cliente complexas e em mudança
com a pressão adicional de tempo de colocação no mercado de mais curto. Para
resolver esta
situação, práticas ágeis têm aumentado a capacidade para software
empresas de desenvolvimento para lidar com as necessidades dos clientes complexos
 e mudando, e mudando as necessidades do mercado [23]. Contudo,
enquanto métodos e práticas ágeis são atraentes para muitos software
empresas de desenvolvimento, existem poucas empresas (por exemplo, Facebook
[21,40], Atlassian [41], a IBM [4], Adobe e Tesla [10], e Microsoft
[26] que tiveram sucesso na implementação de práticas ágeis para um
medida em que software é continuamente implantado para seus clientes
[42]). Ao mesmo tempo, uma série de estudos recentes [16,47]
revelaram que os profissionais de desenvolvimento de software ágil tem
mostrado um interesse crescente no sentido de compreender o princípio de
desenvolvimento ágil de software '' Entregar [ndo] de software que trabalha
freqüentemente '' [7].
Para ser capaz de se mover em direção, e implementar um Continuous
Implantação (CD) processo de software, vários passos precisam ser
tomado [33]. Enquanto implantação contínua de software pode criar
novas oportunidades de negócios, implantação contínua também apresenta
novos desafios. Uma série de blogs organizacionais dos praticantes
[4,22,41,48] manifestaram os desafios que enfrentaram quando
adotar e utilizar o CD. Estes desafios incluem a facilidade de
implantação de bugs de software, estabelecendo uma cultura de toda a organização,
a necessidade de adoção de princípios 'magras', e a colaboração entre equipes.
Além disso, Humble e Farley [27] afirma que um '' objeção intuitiva para
implantação contínua é que é muito arriscado '', e que
é um fato conhecido que mais lançamentos levar a um menor risco de qualquer
lançamento. Assim, implantação contínua reduz os riscos de cada
versão do software.
%aqui pode ir na introdução
A instalação e configuração de um software é um assunto sério, esta aumenta
vez mais nos afeta principalmente devido desenvolvedores à proliferação e
complexidade do sistemas voltados para a internet. Felizmente, nós podemos
controlar e conquistar este comcomplexidade, adotando IT-sistema de figuração
ferramentas de gerenciamento de implantação. Comece com a complexidade das novas
aplicações na internet, são aplicações de louças foram monolítico contraptions
que podem ser instalados numa ambiente padrão básico. para em aplicações de
prémios e do usuário final, a ambiente seria a operar sis; para sistemas embarcados, seria
o hardware subjacente. Tudo o que desenvolvedores tinha que fazer era o teste de software
procedimentos de implantação e gerenciar configuração do nosso software com uma ver
 sistema de controle de Sion, para que pudéssemos proporcionar uma linha de base
 conhecida para o  para a instalação. \cite{6265084}

%introdução do trabalho
Sistemas voltado para a Internet consistem em muitas partes que não podem mais con-
trolar como um bloco monolítico de software. Disponibilidade e desempenho requisitos
mentos impulsionar a adoção de aplicação, servidores, soluções de balanceamento de
 carga, rela sistemas de gerenciamento de banco de dados adicionais, e configurações de
recuperação de desastres, enquanto interoperability requisitos e complexo normas
nos fazer usar uma infinidade de bibliotecas de terceiros e serviços online. Em
seguida, considere a onipresença da Sistemas voltados para a Internet. Considerando
que, no passado, a maioria dos softwares felizmente correu como uma ilha isolada;
hoje, nós expect todos os nossos pedidos para ser Adesãosível através da Web ou
conectado através da Internet. Sistemas ranging de au-molhando do nosso flowerpot
tomation para motores a jato de um avião telefonar regularmente para casa para
fazer upload de seu instruções mais recentes status e recebem. Todos os apps de
smartphone que se preze exigem conectividade com a Internet. Compa-sas, grandes e
pequenos, oferecem o seu duto dutos e serviços através da Web e cada vez mais usam
redes sociais para o mercado e até mesmo criar seu OFERTA Ings. O ecologicamente
 incontestado simplicidade de software monolítica tinha ido a maneira do dodo. Finalmente, percebemos que a complexidade e
alimentação onipresença na infra mais barato custos ture oferecidos pela solução
baseada em nuvem ções e virtualização. A criação de um datacenter usado para ser apenas algo
uma grande empresa pode pagar. Agora, armado com um cartão de crédito, uma startup pode
usar um provedor de nuvem para configurar separado desenvolvimento, teste e produção Serv
ers, um sistema de gerenciamento de banco de dados, um subsistema de armazenamento, um clus- computação
ter, monitoramento e recuperação de desastres em uma semana. \cite{6265084}


%problemas que podem acontecer, também bom para introdução.
Entrega de software sofre como resultado de muitos pós-
questões de desenvolvimento: problemas de gerenciamento de configuração,
falta de testes em um clone de produção e o ambiente
insuficiente colaboração entre as equipes de desenvolvimento e
a equipe de implantação (operações) são os principais problemas que
rejeição software causa nesta fase [2]. Um exemplo é a
falha completa do software na produção
ambiente devido a pressupostos construído em teste anterior
- ambientes que são diferentes em relação às características de
o ambiente de produção. Outro exemplo de tais
problemas que derivam desses fatores é adiantado pela
operações da equipe para perceber que não pode apoiar uma versão do
desenvolveu um software devido à incompatibilidade do software
arquitetura com sua infra-estrutura disponível. O resultado final:
falha de entrega.\cite{akerele2013system}



%------------------------------------------------------------------------------%
\section{Problema}
%

Como implantar multiplas aplicações a partir de um pacote único de
forma centralizada e segura?

%------------------------------------------------------------------------------%
\section{Justificativa e Motivação}
\label{sec:motivacao}


%------------------------------------------------------------------------------%
\section{Objetivos}

%

O obejtivo deste trabalho consiste em automatizar instalação e configuração de
múltiplas instâncias de uma aplicação web a partir de um pacote único, mostrando os aspectos
mais importantes que devem ser tratados durante todo o processo de configuração
e instalação de um software via pacotes debian.

%Além disso, tem-se como objetivo oferecer uma proposta de política de
%empacotamento para aplicações web no Debian no qual busca:
%
%\begin{itemize}
%  \item  \textbf{Politica de empacotamento para aplicações web:} Manual com
%  práticas de empacotamento para aplicações web, com intuito de padronizar a
%  forma em que as aplicações são empacotadas.
%  \item  \textbf{Implantação segura de aplicações web:} Boas práticas para
%  configuração de aplicações web que utilizam pacotes debian.
%\end{itemize}


%------------------------------------------------------------------------------%
\section{Organização do trabalho}

%

Esta monografia está dividida em mais outros x capítulos. A seguir, o leitor
encontrará o Capítulo xxxxx onde são definidos a proposta e
a meotodologia do trabalho, além de ferramentas e dados que serão utilizados
como base para pesquisa. No Capítulo xxxxxx
são introduzidos os principais conceitos de  ....

TODO: Pensar na estrutua de capítulos para a monografia

No fim desta monografia existem X apêndices que complementam e detalham os
aspectos do presente trabalho. No  xxxx apresenta os detalhes
técnicos sobre empacotamento....

%------------------------------------------------------------------------------%
\section{Contribuições}
%
\begin{description}
  \item [Contribuições Tecnológicas]\
\end{description}
    \begin{enumerate}
      \item \textbf{CT1} - Implantação segura de software.
        \begin{enumerate}
          \item Proposta de implatação segura de software utilizando pacotes
          Debian.
          \item Proposta de implantação de multiplas aplicações em um mesmo
          servidor.
          \ item Proposta de técnicas para implantação de multiplas instâncias
           de uma aplicação a partir de um único pacote Debian.
        \end{enumerate}
      \item \textbf{CT2} - Política de empacotamento de aplicações web no Debian.
            \begin{enumerate}
              \item  Proposta de um manual com práticas para empacotamento de
              aplicações web no Debian.
            \end{enumerate}
    \end{enumerate}
\begin{description}
  \item [Contribuições Científicas]\
\end{description}
     \begin{enumerate}
      \item \textbf{CC1} - Estudo teórico das técnicas de implantação segura de
      software.
      \item \textbf{CC2} - Estudo teórico da utilização de certificados de
      segurança (ssl).
      \item \textbf{CC3} - Estudo teórico sobre implantação e configuração de
      multiplas aplicações utilizando virtual hosting.
     \end{enumerate}

%------------------------------------------------------------------------------%
\section{Relato google summer of code}

O projeto aprovado para o google summer of code 2015 foi Automated configuration
of packaged web applications, com a organização Debian Project e com a mentoria de
Antonio Terceiro. A idéia inicial do projeto era uma ferramenta que
pudesse ajudar os usuários que não possuem conhecimento técnico, poderem
instalar e configurar ferramentas a partir de pacotes debian, e tudo isso com
apenas 1 comando (ou um clique). Dentro desse projeto a minha participação
inicialmente seria adicionar o máximo de aplicações possíveis na ferramenta,
e sendo aplicações que fossem bastante conhecidas por usuários da internet.

Um exemplo desse contexto é a configuração e instalação de uma ferramenta de
blog bastante conhecida chamada wordpress,
a instalação do wordpress via pacote é bem simples, bastando um apt-get install wordpress,
porém a fase de configuração é um pouco mais complicada, o usuário precisa
possuir conhecimentos básicos de linux para conseguir fazer a configuração mais
simples do wordpress.A tarefa se torna mais complexa quando é necessário fazer
uma configuração mais complexa, que envolva certificados de segurança por exemplo,
outra coisa que pode tornar mais difícil a vida do usuário é quando ele precisa
configurar um servidor web como apache ou nginx.

Logo existe um limite das tarefas que um pacote debian deve fazer e as tarefas
que o usuário do pacote deve fazer, essas tarefas podem ser automatizadas e assim
facilitando a vida dos usuários debian a instalar e configurar suas aplicações.

Com isso, meu mentor Antonio Terceiro havia iniciado, antes do google summer of
code, o desenvolvimento de uma ferramenta com o intuito de automatizar a
configuração e a instalação de aplicações providas por pacotes debian, a ferramenta
se chama shak(self hosting aplication kit) e ela foi o ponta pé inicial do projeto.

A partir disso eu dei início ao projeto do google summer of code, com o foco
principalmente em adicionar o máximo possível de aplicações já empacotadas pelo
debian, elas serviriam de piloto para a ferramenta, para que assim pudéssemos
também evoluí-la.

A ferramenta é composta por código ruby, chef (falar um pouco mais sobre), shell,
servidor web nginx, pacotes debian e uma aplicação web para os usuários. A decisão
técnica para escolha das ferramentas que compõem o shak já tinha sido feita antes do google summer of
code começar, porém durante o programa tivemos várias discussões sobre, o que me
fez concordar com a escolha delas.

Com o início do programa e o início das atividades, logo pude adicionar o wordpress
como minha primeira aplicação disponível pelo shak, porém não foi algo rápido, dentro
dessa entrega tivemos várias discurssões interessantes. A primeira delas foi
o uso de algumas boas práticas para a configuração de virtual hosting com  o
servidor web nginx a fim de evitar possiveis falhas de segurança.

Além disso também chegamos a conclusão de que seria interessante que o usuário pudesse
criar várias "instâncias" de wordpress no mesmo servidor, por exemplo: o usuário
poderia muito bem querer em um mesmo servidor um blog chamado \url{www.pessoalblog.com}
e \url{www.trabalhoblog.net} ou até mesmo um caminho personalizado como \url{www.empresa.com/blog}
, e isso trouxe a necessidade de entender como é possível
a configuração de multiplas "instâncias" de uma aplicação utilizando um pacote
único no mesmo servidor, já que o modo mais comum de fazer isso seria duplicando
 todos os arquivos da aplicação no servidor, coisa que sem dúvida não é a forma certa.

Com a primeira aplicação pronta partimos para a segunda aplicação, owncloud, que
é uma aplicação bem conhecida para compartilhamento de arquivos na nuvem. A grande
diferença entre o owncloud e o wordpress é que o wordpress suporte multiplas
instâncias com um pacote único, já o wordpress não. Com isso deu-se a necessidade de
buscar uma solução para isso, e após estudar o código fonte do owncloud pude
ver uma forma de adicionar essa nova funcionalidade. Com isso enviei um
patch ao mantenedor do pacote para adicionar o suporte para multiplas instâncias
com o owncloud.

Ao finalizar as duas aplicações vimos ainda que era necessário adicionar mais
uma camada de segurança além do servidor web, que seria forçar as aplicações
a usarem HTPPS, para facilitar esse tipo de configuração foi necessário também
automatizar a geração de certificados de segurança(ssl) para que pudéssemos gerar
um certificado por servidor e assim fazer com que o servidor web use esses certificados
autoassinados.

Por fim, como as duas aplicações precisavam de configuração de servidore de email
a pŕoximo passo foi configurar um servidor de email para servir esse serviço. A
ferramenta escolhida para ser o servidor de IMAP foi o dovecot, o agente de
transferência de emails escolhido foi o postfix, também foram tomados alguns
 cuidados de segurança para que o servidor de email também utilizassem os certificados
 de segurança utilizando o protocolo SMTPS (SMTP + TLS) com isso forçando o servidor
 de emails a aceitar apenas autenticaçãi via TLS, assim também utilizando o
 protocolo IMAPS ao invés de apenas IMAP.

Por fim a última atividade do google summer of code foi a prototipagem da aplicação
web do shak, além disso algumas evoluções no shak foram feitas, as principais
contribuições foram:

\begin{itemize}
  \item  \textbf{Evolução da ferramenta shak:}
  \subitem Evolução da ferramenta shak adicionando suporte ao php5.
  \subitem Suporte as aplicações utilizando arquivos de configuração nginx.
  \subitem Evolução do código em geral e testes unitários.
  \subitem Documentação para configurar ambiente de desenvolvimento.

  \item  \textbf{Novas Aplicações}
  \subitem Wordpress
  \subitem Owncloud
  \subitem Servidor de email

  \item  \textbf{Segurança na implantação das aplicações}
  \subitem Forçar aplicações web a sempre utilizarem o protocolo HTTPS.
  \subitem Gerar certificados de segurança autoassinados automaticamente.
  \subitem Forçar aplicação de email a usarem os protocolos SMPTS e IMAPS.
\end{itemize}
