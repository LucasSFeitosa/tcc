\chapter{Introdução}
\label{cap-introducao}

A internet tornou-se nos últimos 10 anos a ferramenta mais poderosa de comunição
mundial, todos os dias milhões de pessoas se conectam e trocam informações,
milhares de dados são trafegados e várias aplicações são utilizadas todo o tempo.

Com isso a necessidade das aplicações estarem disponíveis na web e online vinte e
quatro horas por dia é praticamente obrigatória, se você busca usuários para sua
aplicação sem dúvida você precisa disponibilizar na web para que possa atigingir
qualquer usuário no mundo todo.

Um dos problemas que mais preocupa a equipe de desenvolvimento de software é a
segurança, a partir do momento em que você insere sua aplicação na web ela está
vulnerável a ataques maliciosos, os motivos pelos quais sua aplicação pode receber
um ataque pode ser variado, e todo cuidado deve ser tomado.

A área de segurança da informação é uma área dentro da engenharia de softwre que
a cada dia cresce cada vez mais, a importância de manter as informações dos usuários
segura é diretamente proporcional a qualidade do serviço prestado, já que ninguém
vai querer utilizar um serviço inseguro. Com isso as empresas investem cada vez
mais na segurança dos seus softwares, buscando minimizar as vulnerabilidades que
possam ser alvos de ataques.

Uma fase importante do ciclo de vida de um software é a fase de implatanção, que
é aonde o software é colocado em produção, ou seja, disponibilizado para os
todos os usuários que possuem acesso a internet (no caso de aplicações web), antes
dessa etapa o software é testado em ambiente de testes ou ambiente de desenvolvimento,
ou seja, nos computadores dos programadores ou em um ambiente controlado, logo é
importante tomar cuidados com a segurança, no processo de implantação do
software em produção, evitando brechas para ataques maliciosos quando o software
for disponibilizado para uso.

Neste trabalho será abordado os aspectos para implantação segura de aplicações
voltadas para web, quais são as boas práticas, as técnicas, as ferramentas e
procedimentos para tornar a implantação mais segura e automática, trazendo assim
benefícios tanto para os engenheiros de software que desenvolveram o sistema
quanto para os seus possíveis usuários.

%------------------------------------------------------------------------------%
\section{Problema}
%

Como implantar multiplas aplicações a partir de um pacote único de
forma centralizada e segura?

%------------------------------------------------------------------------------%
\section{Justificativa e Motivação}


%------------------------------------------------------------------------------%
\section{Objetivos}

%

O obejtivo deste trabalho consiste em um estudo prático sobre implantação segura
de multiplas aplicações, mostrando os aspectos mais importantes que devem ser
tratados durante todo o processo de configuração e instalação de um software via
pacotes debian.

%

Além disso, tem-se como objetivo oferecer uma proposta de política de
empacotamento para aplicações web no Debian no qual busca:

\begin{itemize}
  \item  \textbf{Politica de empacotamento para aplicações web:} Manual com
  práticas de empacotamento para aplicações web, com intuito de padronizar a
  forma em que as aplicações são empacotadas.
  \item  \textbf{Implantação segura de aplicações web:} Boas práticas para
  configuração de aplicações web que utilizam pacotes debian.
\end{itemize}

%------------------------------------------------------------------------------%
\section{Organização do trabalho}

%

Esta monografia está dividida em mais outros x capítulos. A seguir, o leitor
encontrará o Capítulo xxxxx onde são definidos a proposta e
a meotodologia do trabalho, além de ferramentas e dados que serão utilizados
como base para pesquisa. No Capítulo xxxxxx
são introduzidos os principais conceitos de  ....

TODO: Pensar na estrutua de capítulos para a monografia

No fim desta monografia existem X apêndices que complementam e detalham os
aspectos do presente trabalho. No  xxxx apresenta os detalhes
técnicos sobre empacotamento....

%------------------------------------------------------------------------------%
\section{Contribuições}
%
\begin{description}
  \item [Contribuições Tecnológicas]\
\end{description}
    \begin{enumerate}
      \item \textbf{CT1} - Implantação segura de software.
        \begin{enumerate}
          \item Proposta de implatação segura de software utilizando pacotes
          Debian.
          \item Proposta de implantação de multiplas aplicações em um mesmo
          servidor.
          \ item Proposta de técnicas para implantação de multiplas instâncias
           de uma aplicação a partir de um único pacote Debian.
        \end{enumerate}
      \item \textbf{CT2} - Política de empacotamento de aplicações web no Debian.
            \begin{enumerate}
              \item  Proposta de um manual com práticas para empacotamento de
              aplicações web no Debian.
            \end{enumerate}
    \end{enumerate}
\begin{description}
  \item [Contribuições Científicas]\
\end{description}
     \begin{enumerate}
      \item \textbf{CC1} - Estudo teórico das técnicas de implantação segura de
      software.
      \item \textbf{CC2} - Estudo teórico da utilização de certificados de
      segurança (ssl).
      \item \textbf{CC3} - Estudo teórico sobre implantação e configuração de
      multiplas aplicações utilizando virtual hosting.
     \end{enumerate}

%------------------------------------------------------------------------------%
\section{Relato google summer of code}

O projeto aprovado para o google summer of code 2015 foi Automated configuration
of packaged web applications, com a organização Debian Project e com a mentoria de
Antonio Terceiro. A idéia inicial do projeto era uma ferramenta que
pudesse ajudar os usuários que não possuem conhecimento técnico, poderem
instalar e configurar ferramentas a partir de pacotes debian, e tudo isso com
apenas 1 comando (ou um clique). Dentro desse projeto a minha participação
inicialmente seria adicionar o máximo de aplicações possíveis na ferramenta,
e sendo aplicações que fossem bastante conhecidas por usuários da internet.

Um exemplo desse contexto é a configuração e instalação de uma ferramenta de
blog bastante conhecida chamada wordpress,
a instalação do wordpress via pacote é bem simples, bastando um apt-get install wordpress,
porém a fase de configuração é um pouco mais complicada, o usuário precisa
possuir conhecimentos básicos de linux para conseguir fazer a configuração mais
simples do wordpress.A tarefa se torna mais complexa quando é necessário fazer
uma configuração mais complexa, que envolva certificados de segurança por exemplo,
outra coisa que pode tornar mais difícil a vida do usuário é quando ele precisa
configurar um servidor web como apache ou nginx.

Logo existe um limite das tarefas que um pacote debian deve fazer e as tarefas
que o usuário do pacote deve fazer, essas tarefas podem ser automatizadas e assim
facilitando a vida dos usuários debian a instalar e configurar suas aplicações.

Com isso, meu mentor Antonio Terceiro havia iniciado, antes do google summer of
code, o desenvolvimento de uma ferramenta com o intuito de automatizar a
configuração e a instalação de aplicações providas por pacotes debian, a ferramenta
se chama shak(self hosting aplication kit) e ela foi o ponta pé inicial do projeto.

A partir disso eu dei início ao projeto do google summer of code, com o foco
principalmente em adicionar o máximo possível de aplicações já empacotadas pelo
debian, elas serviriam de piloto para a ferramenta, para que assim pudéssemos
também evoluí-la.

A ferramenta é composta por código ruby, chef (falar um pouco mais sobre), shell,
servidor web nginx, pacotes debian e uma aplicação web para os usuários. A decisão
técnica para escolha das ferramentas que compõem o shak já tinha sido feita antes do google summer of
code começar, porém durante o programa tivemos várias discussões sobre, o que me
fez concordar com a escolha delas.

Com o início do programa e o início das atividades, logo pude adicionar o wordpress
como minha primeira aplicação disponível pelo shak, porém não foi algo rápido, dentro
dessa entrega tivemos várias discurssões interessantes. A primeira delas foi
o uso de algumas boas práticas para a configuração de virtual hosting com  o
servidor web nginx a fim de evitar possiveis falhas de segurança.

Além disso também chegamos a conclusão de que seria interessante que o usuário pudesse
criar várias "instâncias" de wordpress no mesmo servidor, por exemplo: o usuário
poderia muito bem querer em um mesmo servidor um blog chamado \url{www.pessoalblog.com}
e \url{www.trabalhoblog.net} ou até mesmo um caminho personalizado como \url{www.empresa.com/blog}
, e isso trouxe a necessidade de entender como é possível
a configuração de multiplas "instâncias" de uma aplicação utilizando um pacote
único no mesmo servidor, já que o modo mais comum de fazer isso seria duplicando
 todos os arquivos da aplicação no servidor, coisa que sem dúvida não é a forma certa.

Com a primeira aplicação pronta partimos para a segunda aplicação, owncloud, que
é uma aplicação bem conhecida para compartilhamento de arquivos na nuvem. A grande
diferença entre o owncloud e o wordpress é que o wordpress suporte multiplas
instâncias com um pacote único, já o wordpress não. Com isso deu-se a necessidade de
buscar uma solução para isso, e após estudar o código fonte do owncloud pude
ver uma forma de adicionar essa nova funcionalidade. Com isso enviei um
patch ao mantenedor do pacote para adicionar o suporte para multiplas instâncias
com o owncloud.

Ao finalizar as duas aplicações vimos ainda que era necessário adicionar mais
uma camada de segurança além do servidor web, que seria forçar as aplicações
a usarem HTPPS, para facilitar esse tipo de configuração foi necessário também
automatizar a geração de certificados de segurança(ssl) para que pudéssemos gerar
um certificado por servidor e assim fazer com que o servidor web use esses certificados
auto-assinados.

Por fim, como as duas aplicações precisavam de configuração de servidore de email
a pŕoximo passo foi configurar um servidor de email para servir esse serviço. A
ferramenta escolhida para ser o servidor de IMAP foi o dovecot, o agente de
transferência de emails escolhido foi o postfix, também foram tomados alguns
 cuidados de segurança para que o servidor de email também utilizassem os certificados
 de segurança utilizando o protocolo SMTPS (SMTP + TLS) com isso forçando o servidor
 de emails a aceitar apenas autenticaçãi via TLS, assim também utilizando o
 protocolo IMAPS ao invés de apenas IMAP.

Por fim a última atividade do google summer of code foi a prototipagem da aplicação
web do shak, além disso algumas evoluções no shak foram feitas, as principais
contribuições foram:

\begin{itemize}
  \item  \textbf{Evolução da ferramenta shak:}
  \subitem Evolução da ferramenta shak adicionando suporte ao php5.
  \subitem Suporte as aplicações utilizando arquivos de configuração nginx.
  \subitem Evolução do código em geral e testes unitários.
  \subitem Documentação para configurar ambiente de desenvolvimento.

  \item  \textbf{Novas Aplicações}
  \subitem Wordpress
  \subitem Owncloud
  \subitem Servidor de email

  \item  \textbf{Segurança na implantação das aplicações}
  \subitem Forçar aplicações web a sempre utilizarem o protocolo HTTPS.
  \subitem Gerar certificados de segurança auto-assinados automaticamente.
  \subitem Forçar aplicação de email a usarem os protocolos SMPTS e IMAPS.
\end{itemize}


%
