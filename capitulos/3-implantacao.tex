\chapter{Implantação de Software}
\label{cap-implantacao}

%nesse capitulo eu pretendo falar sobre implantação de software em geral, trazer
%alguns conceitos, falar sobre as técnicas(citar empacotamento!!!) e as ferramentas
%que permitem automatizar a implantação de um software (citar chef, puppet,etc).
%exemplo de citação
%\citeonline{arthur&carlos2014}

A importância da atividade de implantação dentro do ciclo do desenvolvimento
de software vem ganhando forças, principalmente pela necessidade do alinhamento
entre o time de desenvolvimento com o time de operações(conhecido como sysadmin).
Essa necessidade vem em relação a ligação entre as duas equipes no que tange a
processos de implantação de software, ferramentas para automatizar implatanções
e responsabilidades dentro do ciclo de desenvolvimento. Com isso surgiu o termo devops, que
traz as práticas de desenvolvimento ágil a implatanção e une a equipe de desenvolvimento
com a equipe de opreções a fim de acelerar as entregas do software aumentando
a sua qualidade, reduzindo as lacunas entre essas duas equipes.

De acordo com \citeonline{deployment1998} as aplicações de software não são mais
sistemas autônomos, são cada vez mais o resultado da integração de coleções de
componentes, e nem sempre existe a garantia de que cada componente será implantado
corretamente. A equipe de devops devem, portanto, encontrar uma maneira de lidar
com uma maior incerteza no ambiente no qual seus sistemas vão operar.
Por exemplo, antes que eles possam garantir uma instalação bem-sucedida, o
s produtores devem ser capazes de determinar quais os componentes estão disponíveis
 em um dado local, bem como a configuração desses componentes e ser capaz de
antecipar ou reagir a alterações de componentes que não estão sob seu controle.

Essas questões  devem ser levadas em consideração e a atividade de implatação de
software deve receber uma atenção especial, visto que os softwares estão ficando
cada vez mais complexos, podendo ter várias ferramentas integradas em diferentes
linguagens de programação e diferentes configurações. Neste capítulo falaremos
sobre (Gerência de configuração de software ou Processo de Implantação de Software)
para entender melhor como funciona o processo de implantação, depois vamos ver as
ferramentas que buscam facilitar  e resolver os problemas da implantação de software,
e as características típicas que essas ferramentas trazem para tornar a implatanção
mais rápida e com qualidade, por fim faleramos sobre....(se for falar de mais coisas)

\section{Gerência de configuração de software ou Processo de Implantação de Software}

\section{Métodos e ferramentas para implantação de software}

%empacotamento, instalação via script, tar.gz, exe, jar, etc.

\citeonline{deployment1998} diz que recentemente, um número de novas tecnologias
começaram a emergir para resolver o problema de implantação. As características
típicas oferecidas por estas tecnologias incluem sistemas para automatizar a
implantação a partir de configurações, pacotes, gerenciamento de rede
e instalação de recursos, com  propósito de entrega das atualizações de forma
automática.


.................... citações do trabalho do léo.... chef... etc

\citeonline{deployment1998} diz que  apesar da proliferação de tecnologias de
implantação, não temos nenhuma compreensão clara da questões relacionadas com a
implantação em escala de Internet, nem uma maneira de caracterizar a adequação
dessas tecnologias para abordar essas questões, com isso o objetivo deste trabalho
é trazer os aspectos da implantação segura de múltiplas instâncias de aplicações
propondo um modelo de implantação caracterizando o uso de ferramentas e tecnologias
para apoiar as atividades de implantação.

%
