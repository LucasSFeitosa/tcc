\chapter{Referencial Teórico}
\label{cap-referencial}

\section{Implantação de Software}

Software só oferece valor para os clientes quando eles estão implantados em produção
e assim possam proporcionar as funcionalidades necessárias ao usuário usuário final,
por isso a importância da fase de implantação de software, pois é nela em que a equipe
de desenvolvimento disponibilizará o software ou uma atualização com novas funcionalidades
ao usuário. Sendo assim, a Implantação de software é um conjunto de atividade cruciais
para todos os fornecedores de software, isto começa desde um pedido de um novo requisito
de software até todas as medidas necessárias para essa nova versão está disponível
para o cliente \cite{5741269}. A implantação de software é composta com várias
atividades que são essenciais para disponibilizar um produto a alguém, como por exemplo:
instalaçaõ de dependências, arquivos de configuração e instalação da própria aplicação.

De acordo com \cite{deployment1998} as aplicações de software não são mais
sistemas autônomos, são cada vez mais o resultado da integração de coleções de
componentes, o que pode tornar a implantação de software um processo complicado, e
nem sempre existir a garantia de que cada componente será implantado corretamente.
Portanto, a equipe de desenvolvimento deve encontrar uma maneira de lidar
com uma maior incerteza no ambiente no qual seus sistemas vão operar, garantindo
que a implantação do software seja feita da forma correta, evitando o acontecimento
de erros e imprevistos.

Uma grande aliada das equipes de desenvolvimento é a implantação automatizada de
software, que  refere-se à prática de implantar o software para os usuários finais
automaticamente, evitando qualquer tipo de execução de esforço manual, por isso
a prática de implantação automatizada facilita na rápida entrega de software, tanto
para implantações de aplicações em servidores na nuvem, como implantação de software
para usuários em seus computadores.\cite{7284592}.

Essas questões  devem ser levadas em consideração e a atividade de implatação de
software deve receber uma atenção especial, visto que os softwares estão ficando
cada vez mais complexos, podendo ter várias ferramentas integradas em diferentes
linguagens de programação e diferentes configurações, com todas essas variáveis compondo
uma implantação automatizada de software. Este capítulo possui uma subseção sobre
processo de Implantação de Software, para entender melhor como funciona o processo de implantação,
depois uma subseção falando sobre DevOps, que busca trazer um conceito de implantação
automatizada de software junto com métodos ágeis e por fim, uma subseção sobre
ferramentas e técnicas que buscam facilitar e resolver os problemas da
implantação automatizada de software, com as características típicas que essas
ferramentas trazem para tornar a implatanção mais rápida e com qualidade.

\subsection{Processo de Implantação de Software}

A implantação de um software é o processo que vai desde a aquisição desse software
até a sua execução \cite{leo2014}. O processo implatanção consiste em diversas
atividades que devem ser executadas desde o planejamento da implatanção até a
disponibilidade para uso.

A OMG (Object Management Group) é uma organização internacional sem fins lucrativos
que aprova padrões abertos para tecnologias e eles definem uma especificação de
implantação e configuração de aplicações distribuidas baseadas em componentes \cite{omg2006},
a OMG também diz que o processo de implantação é um processo que se inicia desde
a aquisição de um componente até o momento em que esse componente está em plena
execução, pronto pra uso.

Segundo \cite{omg2006} os principais termos definidos dentro de uma implantação
de software são:

\begin{itemize}
  \item  \textbf{Implantador:} Pessoa ou equipe responsável pelo processo de
  implantação  de um sistema.
  \item  \textbf{Ambiente alvo:} São o servidor ou conjunto de servidores em
  que os componentes são implantados.
  \item  \textbf{Nó:} É um recurso computacional onde se implanta um componente,
  por exemplo uma máquina virtual dentro do servidor que vai hospedar um serviço
  de banco de dados. Os nós fazem parte do ambiente alvo.
  \item  \textbf{Pacote:} Artefato executável que contém o código binário do componente
  . É por meio de um pacote que um serviço pode ser instalado e executado dentro
  de um sistema operacional, e que são característicos dependendo da distribuição
  do sistema operacional, por exemplo: .deb para debian e .rpm para redhat, ou
  pacotes independentes de sistema operacional como: jar.
\end{itemize}

Além disso, é definido as fases que compõem o processo de implantação e de acordo
com \cite{omg2006} as fases são:

\begin{itemize}
  \item  \textbf{Planejamento:} O planejamento da implantação é uma fase
  para identificar os componentes necessários na implantação e como cada um será
  distribuido entre os nós do ambiente alvo.
  \item  \textbf{Preparação:} São os procedimentos necessários para preparar o
  ambiente alvo para que um determinado componente possa ser executado, isso envolve
  configuração do sistema operacional, instalação e configuração de dependências
  necessárias (por exemplo um servidor web como apache ou nginx) e a transferência
  do componente para o servidor onde ele será executado.
  \item  \textbf{Instalação:} O implantador transfere o componente para a infraestrutura
  alvo, é quando instalamos um dado componente em um servidor, um exemplo disso
  é a instalação de uma aplicação via pacotes (.deb ou .rmp).
  \item  \textbf{Configuração:} Edição de arquivos de configuração para alterar
  determinado comportamento de uma aplicação, o implantador aplica configurações
  específicas a aplicação a partir de arquivos de configuração.
  \item  \textbf{Inicialização:} É quando a aplicação é iniciada e entra em execução,
  pronto para receber chamadas de seus clientes.
\end{itemize}

Os termos e fases formam uma estrutura básica que um processo de implantação de software
deve conter, podendo haver modificações de acordo com a necessidade da equipe de implantação,
cada fase é necessária para que possa ter uma implantação consistente, ou seja, as
atividades dentro de uma implantação de software estarão organizadas, evitando assim
possíveis erros durante a implantação, essa organização também podende ajudar o
time de implantação a encontrar os problemas ocorridos dentro de uma implantação
de software, caso ocorram.

Dependendo do tamanho da aplicação essas fases podem se tornar tarefas complicadas,
com isso existe a necessidade de automatizar o processo de implantação, assim
diminuindo a possibilidade de erros comparado a uma tarefa manual e consequentemente
tornando o trabalho mais ágil, dando adeus aos longos manuais de instalação.
\cite{humble2010} diz que o objetivo de um processo de implantação automatizado é
proporcionar um processo de implantação reprodutível, confiável e fácil de ser
executado.

Para que a implantação automatizada de softwre seja possível, é necessário o uso
de ferramentas para poder automatizar todo o processo de implantação,
mas é extremamente importante compreender cada fase do processo para que possa ser
feito um bom planejamento e execução da implantação.

\subsection{DevOps}
\label{subsec:devops}
Como resultado ao longo dos anos o que aconteceu é que o desenvolvimento de software
equipes são capazes de entregar a um ritmo muito mais rápido do que o ritmo em
que as equipes de operações pode absorver o constrói, principalmente a adoção
de métodos ágeis de desenvolvimento de software \cite{7173368}, como podemos
ver na figura \ref{fig:devops}, as equipes desenvolvem de forma acelerada, enquanto
a equipe de operações consegue implatar de forma sequêncial. Sendo assim DevOps
nasceu a partir dessa necessidade de implantar software no mesmo ritmo em que
as equipes são capazes de entregar novas funcionalidades.

\begin{figure}[h]
  \centering
  \caption{Fluxo de desenvolvimento de software em relação a implantação}
  \includegraphics[width=1.0\textwidth]
      {figuras/devops.eps}
    Fonte: \cite{7173368}
\label{fig:devops}
\end{figure}

DevOps pode ser visto como um conjunto de práticas e princípios para a entrega de software, onde a
chave é o foco e a velocidade de entrega e atuomação, para pode aruxiliar na capacidade
de reagir a mudanças \cite{7173368}. A automação da implantação de software vem
principalmente pela necessidade do alinhamento entre o time de desenvolvimento
com o time de operações. Essa necessidade vem em relação a ligação entre as duas
equipes em relação ao processo de implantação de software, ferramentas para
automatizar implatações de software, as atividades de implantação e responsabilidades
dentro do ciclo de desenvolvimento, o que antes não era visto, o que acontecia é
que o time de desenvolvimento terminavam sua funcionalidades e seus testes e
entregavam a uma equipe de implantação. DevOps tenta unir esses dois mundos,
fazendo com que as duas equipes compartilhem atividades e responsabilidades \cite{6265084}.

De acordo com \cite{httermann2012devops} a comunidade DevOps defende a comunicação
entre a equipe de operações e a equipe de desenvolvimento como um meio de assegurar
que o os desenvolvedores entendam os problemas associados com as operações. Um dos
principais benefıcios disso é a capacidade de quantificar aspectos do desenvolvimento,
isto é, levar a melhoria do desenvolvimento do produto, devido a um foco mais nıtido
em agilidade das entregas.

Em \cite{7173368} é abordado as práticas devops, que podem ser aplicadas diversas
 vezes dentro de um ciclo de desenvolvimento de software, as praticas DevOps podem ser resumidas em:

 \begin{itemize}
   \item \textbf{Planejamento Contınuo:} O planejamento deve ser sempre contínuo
   e evoluindo junto ao planejamento da equipe de desenvolvimento, com tarefas
   sendo priorizadas o tempo todo e sempre alinhadas de acordo com as decisões tomadas
   em conjunto com a equipe de desenvolvimento.
   \item \textbf{Integração Contínua:} Compartilhar as alterações feitas com toda equipe
   evitando que as novas modificações fiquem apenas nas maõs do time de desenvolvimento.
   \item \textbf{Implantação Contínua:} O coração do DevOps, recomenda-se automatizar
   todo o processo de implantação, removendo quaisquer tipo de etapas manuais com auxílio
   de uma ferramenta que possa automatizar a instalação, sendo assim a implantação
   de cada nova versão de software passa a ser de forma mais rápida e eficiente.
   \item \textbf{Testes Contínuos:} Automatiza também os testes da sua aplicação.
   \item \textbf{Monitoração Contínua:} Monitorar as novas alterações
   que foram implantadas a fim de aumentar a capacidade de reagir a quaisquer surpresa
   em tempo hábil.
 \end{itemize}

 Os benefícios do uso do DeVops são vários, como por exemplo: economia de tempo,
 implantação de software passa a ser um processo natural, economia de custos, aumentar
 a organização e eficiencia do desenvolvimento de software como um todo \cite{7173368},
 assim podendo resolver um problema que talvez ainda não é conhecido pela equipe
 de desenvolvimento, pois a necessidade também depende de cada equipe, sem
 esquecer o principal foco do DevOps que é a automação.

Para a entender melhor a prática de implantação automatizada, que é o coração
do DevOps, é necessário possuir um conhecimento prévio sobre ferramentas para
automatizar a implantação de software, conhecendo algumas ferramentas utilizadas
no mercado para automazar implantação e suas maneiras de atuação, além das técnicas
de implantação utilizadas, como por exemplo o empacotmento de software.

\subsection{Métodos e ferramentas para implantação automatizada de software}
\label{subsec:metodoseferramentas}

%empacotamento, instalação via script, tar.gz, exe, jar, etc.

Existem várias maneiras de se automatizar um processo de implantação software,
de acordo com \cite{leo2014}, podem-se utilizar linguagens de script de propósito
geral como (Python, shell script), ferramentas gerais voltados para o processo
de implantação ou sistemas de middleware especializados em determinados tipos de artefatos implantáveis.
Um processo de implantação automatizado depende bastante da integração de diferentes papéis
em uma organização, foi dito também na seção \ref{subsec:devops} que é importante a
integração entre desenvolvedores e operadores, uma vez que o desenvolvimento desses
scripts ou utilização das ferramentas precisam de participação de ambos os perfis.

Existem várias soluções técnicas  para melhorar a implantação de software,
\cite{5741269} no entanto, apesar da sua importância para as empresas de software, as atividades
de implantação de software têm recebido pouca atenção. Em \cite{deployment1998}
é dito que recentemente, um número de novas tecnologias começaram a emergir para
resolver o problema de implantação. As características típicas oferecidas por
estas tecnologias incluem sistemas para automatizar a implantação a partir de
configurações, pacotes, gerenciamento de rede e instalação de recursos, com
propósito de entrega das atualizações de forma automática.

Assim, um grande facilitador da implantação de software são as ferramentas que auxiliam na
automação das atividades de implantação, pois dependendo do software, a implantação
pode se tornar um processo com muitas atividades ou bastante longo, e a automação
vêm para facilitar a implantação, evitando erros de instalações manuais e facilitando a replicação em
vários ambientes diferentes.

De acordo com  \cite{6265084} tais ferramentas permitem-nos controlar e automatizar
a configuração de todos os elementos que compõem um sistema, são eles: hosts, software a serem instalados,
usuários do sistema, serviços em execução, arquivos de configuração, tarefas agendadas,
configuração de rede, armazenamento de arquivos, monitoramento e segurança. Existem
ferramentas populares como Puppet e Chef, tendo como sua principal função principal
de automatizar a instalação e configuração de um sistema. Com essas ferramentas é
possível escrever algumas regras que expressam como um software deve ser configurado,
e assima ferramenta irá configurar o sistema conforme as especificações.

Essas ferramentas funcionam de forma declarativa, especificando a instalação de um sistema de
através de regras \cite{6265084}. Por exemplo, é possível especificar regras para a configuração do sistema
a partir de uma infra-estrutura básica, no chef essa estrutura é conhecida como livro
de receitas, em puppter essa estrutura é conhecida como manifesto, esses recursos
são basicamente arquivos, no qual o implantador define os passos que serão executados,
cada um podendo executar: um aplicativo, um serviço, papéis de usuários do sistema,
aruivos de configuração e etc.

Cada passo da configuração desejada pode depender de outros passos, ou seja, para
instalar uma aplicação é preciso instalar um compilador ou interpetrador da linguagem,
para rodar uma aplicação web é necessárioa instalaçaõ de um servidor web. Para
lidar com estas dependências essas fermamentas descrevem o estado em que a aplicação
deve estar, ou seja, todos os pacotes, arquivos de configuração diretórios, usuários,
sendo possível definir dependência entre as tarefas a serem executadas.

Com essas ferramentas a implantaçaõ de um software dentro de uma infraestrutura
não precisa ser um esforço manual, com elas podemos automatizar toda a implantação
de um software, desde a preparação do ambiente até a inicialização da aplicação,
sendo assim possível reduzir o tempo gasto na implantação, sem qualquer esforço
adicional, o que integra os conceitos de DevOps com implantação de software automatizada
de software.

Além das ferramentas de automação, é importante também entender o empacotamento de
programas, tanto chef como puppet utilizam a instalação de pacotes como recurso para
poder instalar softwares,ou seja, é possível buscar os softwares desejados a partir
dos pacotes disponíveis em uma distribuição linux. As distribuições linux que
optam por disponibilizar pacotes mantém uma infraestrutura de servidores como fonte
de distribuição de programas \cite{araujo2011apprecommender}, tais servidores são
chamados de repositórios, e esses pacotes são gerenciados com softwares conhecidos
como sistemas gerenciadores de pacotes, que possuem a responsabilidade de buscar
os softwares a serem instalados que estão disponíveis nos respectivos repositórios.

Existem pacotes que dependem do sistema operacional, como por exemplo pacotes .deb
para a distribuição Debian e pacotes .rpm para a distribuição Fedora, e
pacotes independentes de sistema operacional como por exemplo pacotes jar da
linguagem java. Neste trabalho trataremos apenas de pacotes Debian.

\subsubsection{Pacotes Debian}

O formato utilizado pelo Debian GNU/LINUX e seus derivados é o formato de pacotes
binátios conhecido como .deb, um pacote .deb é composto de arquivos executáveis,
bibliotecas e documentação associada a um programa ou a um conjunto de programas,
além de todos os dados e procedimentos necessários para instalar, configurar e remover
aplicativos de um sistema \cite{araujo2011apprecommender}. A estrutura dos pacotes
e também os seus requisitos para que sejam distribuidos oficialmente pelo Debian
estão especificados no Manual de Políticas do Debian \cite{debian}.

A estrutura de um pacote debian pode ser observado a partir de qulquer pacote
.deb, onde é possível encontrar três estruutras a seguir:

\begin{itemize}
  \item \textbf{debian-binary:} Este arquivo contém a versão da especificação
  do empacotamento implementado no pacote, em 2015 a versão é a 2.0 nos pacotes.
  \item \textbf{control.tar.gz:} Este arquivamento contém todas as informações
   disponíveis, como o nome e a versão do pacote. Essas informações servem para
   que as ferramentas de gerenciamento de pacotes determinarem se é possível
   instalar e desinstalar o pacote.
   \item \textbf{data.tar.gz:} Este arquivamento contém todos os arquivos para
   serem extraídos do pacote, ou seja, é onde ficam os arquivos executáveis, a documentação,
   os devidos diretórios em que tais arquvios devem ser copiados e etc.
\end{itemize}

Ao instalar um pacote .deb, são feitos todos os procedimentos necessários para a instalação
de uma aplicação, porém, existem ainda casos em que o usuário precise fazer algumas
configurações, um exemplo é quando o usuário precisa configurar um banco de dados
ou configuração da aplicação a partir de arquivos de configuração. Essas tarefas
não são de responsabilidade dos pacotes e sim do usuário que deseja utilizar esse
software, já que o pacote não poderia prever qual é a estrutura e configuração
que o usuário pretende utilizar, logo a execução desses passos são feitas de
acordo com a preferência do usuário, podendo serfeita manualmente ou a partir
de uma implantação automatizada que utilize alguma ferramenta para isso.

Outro fator importante é que um pacote também não pode prever quais são os procedimentos
de segurança que devem ser tomados, como por exemplo a criação de certificados de
segurança. Para entender melhor quais são esses procedimentos, na próxima seção
será abordado os procedimentos de segurança na implantação de aplicações web utilizando
pacotes Debian, no qual trata o uso de protocolos que utilizam criptografia para a segurança de dados,
que também são procedimentos que podem ser feitos manualmente ou utilizando alguma
ferramentas para automatizar o processo.

\section{Segurança na Implantação de Aplicações Web}
\label{sec:seguranca}

%colocar uma introdução

\subsection{Protocolos}

%falar dos protocolos

\subsection{Ceritificados}

%falar dos certificados

\subsection{Múltiplas Instâncias com VirtualHosting}
%falar sobre
