\chapter{Referencial Teórico}
\label{cap-implantacao}

\section{Implantação de Software}
%nesse capitulo eu pretendo falar sobre implantação de software em geral, trazer
%alguns conceitos, falar sobre as técnicas(citar empacotamento!!!) e as ferramentas
%que permitem automatizar a implantação de um software (citar chef, puppet,etc).
%exemplo de citação
%\citeonline{arthur&carlos2014}

A importância da atividade de implantação dentro do ciclo do desenvolvimento
de software vem ganhando forças, principalmente pela necessidade do alinhamento
entre o time de desenvolvimento com o time de operações(conhecido como sysadmin).
Essa necessidade vem em relação a ligação entre as duas equipes no que tange a
processos de implantação de software, ferramentas para automatizar implatações
e responsabilidades dentro do ciclo de desenvolvimento. Com isso surgiu o termo devops, que
traz as práticas de desenvolvimento ágil a implatanção de software e une a equipe de desenvolvimento
com a equipe de opreções a fim de acelerar as entregas do software aumentando
a sua qualidade, reduzindo as lacunas entre essas duas equipes.

De acordo com \cite{deployment1998} as aplicações de software não são mais
sistemas autônomos, são cada vez mais o resultado da integração de coleções de
componentes, e nem sempre existe a garantia de que cada componente será implantado
corretamente. A equipe de devops devem, portanto, encontrar uma maneira de lidar
com uma maior incerteza no ambiente no qual seus sistemas vão operar.
Por exemplo, antes que eles se possa garantir uma instalação bem-sucedida, o time
de devops deve ser capazes de determinar quais os componentes estão disponíveis
 em um dado local, bem como a configuração desses componentes e ser capaz de
antecipar ou reagir a alterações desses componentes.

Essas questões  devem ser levadas em consideração e a atividade de implatação de
software deve receber uma atenção especial, visto que os softwares estão ficando
cada vez mais complexos, podendo ter várias ferramentas integradas em diferentes
linguagens de programação e diferentes configurações. Neste capítulo falaremos
sobre processo de Implantação de Software
para entender melhor como funciona o processo de implantação, depois vamos ver as
ferramentas que buscam facilitar  e resolver os problemas da implantação de software,
e as características típicas que essas ferramentas trazem para tornar a implatanção
mais rápida e com qualidade, por fim faleramos sobre....(se for falar de mais coisas)

\subsection{Processo de Implantação de Software}

A implantação de um software é o processo que vai desde a aquisição desse software
até a sua execução. \cite{leo2014} O processo implatanção consiste em diversas
atividades que devem ser executadas desde o planejamento da implatanção até a
disponibilidade para uso.

A OMG (Object Management Group) é uma organização internacional sem fins lucrativos
que aprova padrões abertos para tecnologias (arrumar fonte, é o site deles, mas
não sei como referenciar: \url{http://www.omg.org/gettingstarted/gettingstartedindex.htm})
e eles definem uma especificação de implantação e configuração de aplicações
distribuidas baseadas em componentes \cite{omg2006} e ela diz que o processo de
implantação é um processo que se inicia desde a aquisição de um componente até
o momento em que esse componente está em plena execução, pronto pra uso.

Segundo \cite{omg2006} os principais termos definidos são:

\begin{itemize}
  \item  \textbf{Implantador:} Pessoa ou equipe responsável pelo processo de
  implantação  de um sistema.
  \item  \textbf{Ambiente alvo:} São o servidor ou conjunto de servidores em
  que os componentes são implantados.
  \item  \textbf{Nó:} É um recurso computacional onde se implanta um componente,
  por exemplo uma máquina virtual dentro do servidor que vai hospedar um serviço
  de banco de dados. Os nós fazem parte do ambiente alvo.
  \item  \textbf{Pacote:} Artefato executável que contém o código binário do componente
  . É por meio de um pacote que um serviço pode ser instalado e executado dentro
  de um sistema operacional, e que são característicos dependendo da distribuição
  do sistema operacional, por exemplo: .deb para debian e .rpm para redhat, ou
  pacotes independentes de sistema operacional como: jar.
\end{itemize}

Além disso, é importante também definir as fases que compõem o processo de
implantação e de acordo com \cite{omg2006} as fases são:

\begin{itemize}
  \item  \textbf{Planejamento:} O planejamento da implantação é uma fase
  para identificar os componentes necessários na implantação e como cada um será
  distribuido entre os nós do ambiente alvo.
  \item  \textbf{Preparação:} São os procedimentos necessários para preparar o
  ambiente alvo para que um determinado componente possa ser executado, isso envolve
  configuração do sistema operacional, instalação e configuração de dependências
  necessárias (por exemplo um servidor web como apache ou nginx) e a transferência
  do componente para o servidor onde ele será executado.
  \item  \textbf{Instalação:} O implantador transfere o componente para a infraestrutura
  alvo, é quando instalamos um dado componente em um servidor, um exemplo disso
  é a instalação de uma aplicação via pacotes (.deb ou .rmp).
  \item  \textbf{Configuração:} Edição de arquivos de configuração para alterar
  determinado comportamento de uma aplicação, o implantador aplica configurações
  específicas a aplicação a partir de arquivos de configuração.
  \item  \textbf{Inicialização:} É quando a aplicação é iniciada e entra em execução,
  pronto para receber chamadas de seus clientes.
\end{itemize}

Dependendo to tamanho da aplicação essas fases podem se tornar tarefas complicadas,
com isso existe a necessidade de automatizar o processo de implantação, assim
diminuindo a possibilidade de erros comparado a uma tarefa manual e consequentemente
tornando o trabalho mais ágil, dando adeus aos longos manuais de instalação.
\cite{humble2010} diz que o objetivo de um processo de implantação automatizado é
proporcionar um processo de implantação reprodutível, confiável e fácil de ser
executado, o que casa perfeitamente no contexto de uma equipe devops.

Para que isso seja possível é necessário utilizar-se de ferramentas para poder
automatizar todo o processo de implantação, mas é extremamente importante intender todo o processo
para que possa ser feito um bom planejamento e execução, pois se não o uso da ferramenta também
trará dificuldades para executar o processo, principalmente se a equipe de devops não tiver conhecimento
da ferramenta utilziada. Por julgar importante o conhecimento prévio sobre ferramentas
para automatizar a implantação de software na próxima seção falaremos sobre algumas
ferramentas conhecidas para automazar implantação e suas maneiras de atuação.

\subsection{Métodos e ferramentas para implantação de software}
\label{subsec:metodoseferramentas}

%empacotamento, instalação via script, tar.gz, exe, jar, etc.

\cite{deployment1998} diz que recentemente, um número de novas tecnologias
começaram a emergir para resolver o problema de implantação. As características
típicas oferecidas por estas tecnologias incluem sistemas para automatizar a
implantação a partir de configurações, pacotes, gerenciamento de rede
e instalação de recursos, com  propósito de entrega das atualizações de forma
automática.


.................... citações do trabalho do léo.... chef... etc

\cite{deployment1998} diz que  apesar da proliferação de tecnologias de
implantação, não temos nenhuma compreensão clara da questões relacionadas com a
implantação em escala de Internet, nem uma maneira de caracterizar a adequação
dessas tecnologias para abordar essas questões, com isso o objetivo deste trabalho
é trazer os aspectos da implantação segura de múltiplas instâncias de aplicações
propondo um modelo de implantação caracterizando o uso de ferramentas e tecnologias
para apoiar as atividades de implantação.

% ------------------------------ NEXT SESSION --------------------------


\section{Aplicações com multiplas instâncias}
\label{cap-multiplas}

nesse capitulo eu pretendo falar das vantagens de se utilizar aplicações com
multiplas instancias, como fazer isso e a arquitetura necessária para isso.

\subsection{Arquitetura de aplicações com multiplas instâncias}

Falar um pouco sobre como deve ser a arquietura, pastas, banco isolado,
configurações idependentes, etc.

\subsection{Hosts Virtuais}

Falar sobre isso, já que é extremamente necessário para poder ser possível a
instalação de multiplas instâncias de aplicações web.

% ------------------------------ NEXT SESSION --------------------------


\section{Aspectos para implatanção segura de aplicações}
\label{cap-seguranca}

nesse capitulo pretendo falar dos aspectos para implantação segura de software.

\subsection{Aspectos principais}

Os principais aspectos para implatanção segura.

\subsection{Técnicas e boas práticas}

Técnicas e boas práticas para implatanção segura de software.

\subsection{Testes ...}

falar sobre os testes que são possíveis de se fazer para averiguar tudo que foi
feito, quero que isso vá para os resultados ao final do tcc2.

%
