\chapter{Referencial Teórico}
\label{cap-referencial}


%isso é bom para introdução

A forma como as empresas entregar software vai
por meio de uma onda de mudanças como o ambiente,
as empresas funcionam, está mudando - as necessidades do mercado
estão mudando continuamente, a tecnologia está mudando
rapidamente, existe mais pressão para se adaptar ao mercado
precisa e entregar rapidamente. As empresas não podem
já não ter recursos para fazer uma torre de menagem do cliente à espera de
6 meses ou 1 ano para uma liberação para vir e, em seguida,
solicitar feedback de cliente sobre como software
se comporta. Os clientes esperam que o engajamento contínuo
de modo que eles podem fornecer feedback contínuo. Dentro
a fim de enfrentar os desafios de hoje, as empresas
precisa ser magra e ágil em todas as fases de
software ciclo de vida de desenvolvimento. Ao longo dos anos
organizações adotaram muitos processos
otimizações em seu desenvolvimento de software (ágeis
transformação) práticas. \cite{7173368}


% aqui ja é bom para introdução do devops, colocar a imagem do gargalo
Como resultado ao longo dos anos
o que aconteceu é que o desenvolvimento de software
equipes são capazes de entregar a um ritmo muito mais rápido do que
o ritmo em que as equipes de operações pode absorver o
constrói. É justamente disse que a força da corrente é
tão forte quanto o elo mais fraco na cadeia
DevOps é o conjunto de práticas que está a tentar colmatar
-operações desenvolvedor lacuna no âmago das coisas [5]
mas ao mesmo tempo não se limita a esta
desenvolvimento e operações de handoff em vez cobre
todos os aspectos que ajudam na rápida, otimizada e
entrega de software de alta qualidade. DevOps é um conjunto de
princípios para a entrega de software onde a chave
foco é a velocidade de entrega, teste contínuo no
produção, como o ambiente, seja no estado shippable
em qualquer dia, feedback contínuo, capacidade de reagir a
mudam mais rapidamente, as equipes de trabalho para realizar
um objetivo em vez de uma tarefa (não há mais limites da equipe
causando um atraso [1]. DevOps estende ágil
princípios para todo o funil de entrega de software.
Embora DevOps princípios se aplicam a todo SDLC
mas motivador chave ou zona focal que desencadeou tudo
isto - está certificando-se equipe de operações pode ser executado ao longo
com equipes de desenvolvimento \cite{7173368}



%mais sobre devops
DevOps não envolve apenas
alterar a processos, mas também alterar a cultura.
Organizações que estão adotando princípios DevOps
vai certamente ter uma vantagem sobre as organizações
que não está montando essa onda de DevOps.
Processos têm de mudar com o tempo como o mercado
ambiente em que operamos é continuamente
mudando. DevOps permite a organização
DevOps apenas define o conjunto de princípios, mas "como
e usando o que de tecnologia "organizações adotam
ou que a abordagem princípio é para ser completamente
avaliadas e decididas pela organização. Até
dentro de uma única organização equipes diferentes podem
têm necessidade de adotar tecnologia ou ferramentas diferentes para
adotar DevOps se aproxima - o que é absolutamente
bem, todo o ser objetivo de otimizar continuamente
e transformar. \cite{7173368}

% vantanges devops
Reduzir o tempo de mercado
Adaptar-se a 3/4 feedback contínuo
3/4 equilibrar eficazmente fora de custos e
qualidade
3/4 Ter mais
lançamentos
Aumento 3/4
A organização de
eficiência como toda \cite{7173368}

% implantação automatizada
A implantação automatizada refere-se à prática de fazer
software disponível para os usuários finais automaticamente; esta prática é
conduzida entre e software de aquisição de software
execução sem esforço manual [19]. A prática de
implantação automatizada facilita na entrega rápida de software
alterações aos utilizadores finais [19]. \cite{7284592}


%fala de ágeis e implantação
Hoje, o software é desenvolvido em rápida mudança e os mercados imprevisíveis,
com os requisitos do cliente complexas e em mudança
com a pressão adicional de tempo de colocação no mercado de mais curto. Para
resolver esta
situação, práticas ágeis têm aumentado a capacidade para software
empresas de desenvolvimento para lidar com as necessidades dos clientes complexos
 e mudando, e mudando as necessidades do mercado [23]. Contudo,
enquanto métodos e práticas ágeis são atraentes para muitos software
empresas de desenvolvimento, existem poucas empresas (por exemplo, Facebook
[21,40], Atlassian [41], a IBM [4], Adobe e Tesla [10], e Microsoft
[26] que tiveram sucesso na implementação de práticas ágeis para um
medida em que software é continuamente implantado para seus clientes
[42]). Ao mesmo tempo, uma série de estudos recentes [16,47]
revelaram que os profissionais de desenvolvimento de software ágil tem
mostrado um interesse crescente no sentido de compreender o princípio de
desenvolvimento ágil de software '' Entregar [ndo] de software que trabalha
freqüentemente '' [7].
Para ser capaz de se mover em direção, e implementar um Continuous
Implantação (CD) processo de software, vários passos precisam ser
tomado [33]. Enquanto implantação contínua de software pode criar
novas oportunidades de negócios, implantação contínua também apresenta
novos desafios. Uma série de blogs organizacionais dos praticantes
[4,22,41,48] manifestaram os desafios que enfrentaram quando
adotar e utilizar o CD. Estes desafios incluem a facilidade de
implantação de bugs de software, estabelecendo uma cultura de toda a organização,
a necessidade de adoção de princípios 'magras', e a colaboração entre equipes.
Além disso, Humble e Farley [27] afirma que um '' objeção intuitiva para
implantação contínua é que é muito arriscado '', e que
é um fato conhecido que mais lançamentos levar a um menor risco de qualquer
lançamento. Assim, implantação contínua reduz os riscos de cada
versão do software.

%mais sobre devops
DevOps é um conjunto de práticas que defendem a colaboração
entre desenvolvedores de software e operações de TI onde o objectivo é
para encurtar o ciclo de feedback, alinhando os objetivos de ambos os
operações de desenvolvimento e departamentos de TI [28]. Conforme
Huttermann [28], DevOps enfatiza a capacitação de pessoas
processos mais, a necessidade de automação no desenvolvimento de novo
software, que institui medidas de qualidade e criar uma cultura
de partilha entre as pessoas.
A ampla adoção de metodologias ágeis melhorou
o desempenho de equipes de desenvolvedores de software [18]. DevOps alarga
o espectro de metodologias ágeis, não só reunindo
programadores, testadores e engenheiros de seguros de qualidade, mas também o
Operações de TI da equipe por promover a comunicação, colaboração e
alinhando objectivos [28]. De acordo com Bass et al. [5], o DevOps
comunidade defende a comunicação entre as operações
equipe ea equipe de desenvolvimento como um meio de assegurar que o
os desenvolvedores a entender os problemas associados com as operações.
Roche [39] assinala que, a influência de DevOps coloca eficiência
e processar em perspectivas. Um dos principais benefícios da DevOps
é a capacidade de quantificar aspectos do desenvolvimento, isto é, a
processo de desenvolvimento pode ser descrito com as figuras [39]. Isto, por sua
por sua vez, levar à melhoria do desenvolvimento do produto, devido à
um foco mais nítido em métricas.


%decidir o que fica aqui e o que vai na introdução
A instalação e configuração de um software é um assunto sério, esta aumenta
vez mais nos afeta principalmente devido desenvolvedores à proliferação e
complexidade do sistemas voltados para a internet. Felizmente, nós podemos
controlar e conquistar este comcomplexidade, adotando IT-sistema de figuração
ferramentas de gerenciamento de implantação. Comece com a complexidade das novas
aplicações na internet, são aplicações de louças foram monolítico contraptions
que podem ser instalados numa ambiente padrão básico. para em aplicações de
prémios e do usuário final, a ambiente seria a operar sis; para sistemas embarcados, seria
o hardware subjacente. Tudo o que desenvolvedores tinha que fazer era o teste de software
procedimentos de implantação e gerenciar configuração do nosso software com uma ver
 sistema de controle de Sion, para que pudéssemos proporcionar uma linha de base
 conhecida para o  para a instalação. \cite{6265084}


%introdução do trabalho
Sistemas voltado para a Internet consistem em muitas partes que não podem mais con-
trolar como um bloco monolítico de software. Disponibilidade e desempenho requisitos
mentos impulsionar a adoção de aplicação, servidores, soluções de balanceamento de
 carga, rela sistemas de gerenciamento de banco de dados adicionais, e configurações de
recuperação de desastres, enquanto interoperability requisitos e complexo normas
nos fazer usar uma infinidade de bibliotecas de terceiros e serviços online. Em
seguida, considere a onipresença da Sistemas voltados para a Internet. Considerando
que, no passado, a maioria dos softwares felizmente correu como uma ilha isolada;
hoje, nós expect todos os nossos pedidos para ser Adesãosível através da Web ou
conectado através da Internet. Sistemas ranging de au-molhando do nosso flowerpot
tomation para motores a jato de um avião telefonar regularmente para casa para
fazer upload de seu instruções mais recentes status e recebem. Todos os apps de
smartphone que se preze exigem conectividade com a Internet. Compa-sas, grandes e
pequenos, oferecem o seu duto dutos e serviços através da Web e cada vez mais usam
redes sociais para o mercado e até mesmo criar seu OFERTA Ings. O ecologicamente
 incontestado simplicidade de software monolítica tinha ido a maneira do dodo. Finalmente, percebemos que a complexidade e
alimentação onipresença na infra mais barato custos ture oferecidos pela solução
baseada em nuvem ções e virtualização. A criação de um datacenter usado para ser apenas algo
uma grande empresa pode pagar. Agora, armado com um cartão de crédito, uma startup pode
usar um provedor de nuvem para configurar separado desenvolvimento, teste e produção Serv
ers, um sistema de gerenciamento de banco de dados, um subsistema de armazenamento, um clus- computação
ter, monitoramento e recuperação de desastres em uma semana. \cite{6265084}


%colocar na introdução de devops
A revolução que eu descrevi nos afeta desenvolvedores, porque como nós construímos
sistemas de software com muitas dependências complexas, estamos já esperado
para fornecer o código de trabalho, mas contribuir
em direção a um sistema de TI funcionando. Por esta,
precisamos de cooperar estreitamente com o
equipe manipulação operações de TI a fim de
coordenar e integrar de- software
senvolvimento, operações de tecnologia, e
garantia de qualidade.
Quando o desenvolvimento e às operações
equipes rações trabalhar juntos em um ambiente DevOps socalled, os desenvolvedores não
atirar entregas de software sobre uma parede
para operações de implantação. Ao invés,
coordenar os dois através de vários processos de desenvolvimento ágil como continuidade
ous implantação e testes automatizados. DevOps pode fazer maravilhas quando
a organização fornece software
como um serviço (como o Google), como um dispositivo de shrinkwrapped (como o iPhone), ou como
um aplicativo personalizado (como o SAP
ERP). Note, no entanto, que a aplicação
os mesmos princípios sobre psiquiatra software não é realista.\cite{6265084}



%funcionamento das ferramentas
Ferramentas de gerenciamento de configuração trabalho, especificando a instalação de um sistema de
através de regras. Por exemplo, faríamos
especificar regras para a configuração do sistema
infra-estrutura básica (armazenamento, da rede
ing), cada um executando um aplicativo ou
serviço, e os papéis dos usuários do sistema. Cada passo da configuração frequentemente
depende dos outros: para construir um aplicativo, é preciso instalar um compilador; para correr
um serviço voltado para o Web, que poderá ser necessário um
aplicação e um servidor de banco de dados. Para
lidar com estas dependências, que normalmente
expressar cada regra da configuração juntamente com o seu pré e pós-condições.
Como um exemplo, o seguinte Fantoche
regra expressa que para executar o Postfix
servidor de correio como um serviço, o sistema deve
ter instalado o arquivo de pacote ea configuração correspondente \cite{6265084}


%valor ao cliente

Software só oferece valor para os clientes quando eles estão
implantado em produção e proporcionar o necessário
funcionalidades para o usuário final. Como et al Humble aponta:
"É duro o suficiente para desenvolvedores de software para escrever código que
trabalha em sua máquina. Mas mesmo quando isso é feito, há uma
longa viagem de lá para software que está produzindo valor -
desde que o software só produz valor quando está em
produção "[2]. É, portanto, vital para assegurar que um país desenvolvido
software acaba sendo deliverable- como que é o máximo
objetivo. \cite{akerele2013system}

%problemas que podem acontecer

Entrega de software sofre como resultado de muitos pós-
questões de desenvolvimento: problemas de gerenciamento de configuração,
falta de testes em um clone de produção e o ambiente
insuficiente colaboração entre as equipes de desenvolvimento e
a equipe de implantação (operações) são os principais problemas que
rejeição software causa nesta fase [2]. Um exemplo é a
falha completa do software na produção
ambiente devido a pressupostos construído em teste anterior
- ambientes que são diferentes em relação às características de
o ambiente de produção. Outro exemplo de tais
problemas que derivam desses fatores é adiantado pela
operações da equipe para perceber que não pode apoiar uma versão do
desenvolveu um software devido à incompatibilidade do software
arquitetura com sua infra-estrutura disponível. O resultado final:
falha de entrega.\cite{akerele2013system}



%ainda sobre ferraments


\section{Implantação de Software}
%nesse capitulo eu pretendo falar sobre implantação de software em geral, trazer
%alguns conceitos, falar sobre as técnicas(citar empacotamento!!!) e as ferramentas
%que permitem automatizar a implantação de um software (citar chef, puppet,etc).
%exemplo de citação
%\citeonline{arthur&carlos2014}

A Implantação de software é um conjunto de atividade cruciais para todos os fornecedores de
software, isto começa desde um pedido de um cliente de uma nova versão de software até todas
as medidas necessárias para essa nova versão está disponível para o cliente \cite{5741269}.
A implantação de software é composta com várias atividades que são essenciais para
disponibilizar um produto a alguém, essas atividades são atividades de configuração
e instalação de softwres.

A importância da atividade de implantação dentro do ciclo do desenvolvimento
de software vem ganhando forças, principalmente pela necessidade do alinhamento
entre o time de desenvolvimento com o time de operações. Essa necessidade vem em
relação a ligação entre as duas equipes em relação ao processo de implantação
de software, a ferramentas para automatizar implatações de software, atividades
de implantação e  responsabilidades dentro do ciclo de desenvolvimento.

De acordo com \cite{deployment1998} as aplicações de software não são mais
sistemas autônomos, são cada vez mais o resultado da integração de coleções de
componentes, e nem sempre existe a garantia de que cada componente será implantado
corretamente. A equipe de desenvolvimento deve, portanto, encontrar uma maneira de lidar
com uma maior incerteza no ambiente no qual seus sistemas vão operar.

Essas questões  devem ser levadas em consideração e a atividade de implatação de
software deve receber uma atenção especial, visto que os softwares estão ficando
cada vez mais complexos, podendo ter várias ferramentas integradas em diferentes
linguagens de programação e diferentes configurações. Neste capítulo falaremos
sobre processo de Implantação de Software para entender melhor como funciona o
processo de implantação, depois vamos ver as ferramentas que buscam facilitar e
resolver os problemas da implantação de software, e as características típicas que
essas ferramentas trazem para tornar a implatanção mais rápida e com qualidade.

\subsection{Processo de Implantação de Software}

A implantação de um software é o processo que vai desde a aquisição desse software
até a sua execução. \cite{leo2014} O processo implatanção consiste em diversas
atividades que devem ser executadas desde o planejamento da implatanção até a
disponibilidade para uso.

A OMG (Object Management Group) é uma organização internacional sem fins lucrativos
que aprova padrões abertos para tecnologias (arrumar fonte, é o site deles, mas
não sei como referenciar: \url{http://www.omg.org/gettingstarted/gettingstartedindex.htm})
e eles definem uma especificação de implantação e configuração de aplicações
distribuidas baseadas em componentes \cite{omg2006} e ela diz que o processo de
implantação é um processo que se inicia desde a aquisição de um componente até
o momento em que esse componente está em plena execução, pronto pra uso.

Segundo \cite{omg2006} os principais termos definidos são:

\begin{itemize}
  \item  \textbf{Implantador:} Pessoa ou equipe responsável pelo processo de
  implantação  de um sistema.
  \item  \textbf{Ambiente alvo:} São o servidor ou conjunto de servidores em
  que os componentes são implantados.
  \item  \textbf{Nó:} É um recurso computacional onde se implanta um componente,
  por exemplo uma máquina virtual dentro do servidor que vai hospedar um serviço
  de banco de dados. Os nós fazem parte do ambiente alvo.
  \item  \textbf{Pacote:} Artefato executável que contém o código binário do componente
  . É por meio de um pacote que um serviço pode ser instalado e executado dentro
  de um sistema operacional, e que são característicos dependendo da distribuição
  do sistema operacional, por exemplo: .deb para debian e .rpm para redhat, ou
  pacotes independentes de sistema operacional como: jar.
\end{itemize}

Além disso, é importante também definir as fases que compõem o processo de
implantação e de acordo com \cite{omg2006} as fases são:

\begin{itemize}
  \item  \textbf{Planejamento:} O planejamento da implantação é uma fase
  para identificar os componentes necessários na implantação e como cada um será
  distribuido entre os nós do ambiente alvo.
  \item  \textbf{Preparação:} São os procedimentos necessários para preparar o
  ambiente alvo para que um determinado componente possa ser executado, isso envolve
  configuração do sistema operacional, instalação e configuração de dependências
  necessárias (por exemplo um servidor web como apache ou nginx) e a transferência
  do componente para o servidor onde ele será executado.
  \item  \textbf{Instalação:} O implantador transfere o componente para a infraestrutura
  alvo, é quando instalamos um dado componente em um servidor, um exemplo disso
  é a instalação de uma aplicação via pacotes (.deb ou .rmp).
  \item  \textbf{Configuração:} Edição de arquivos de configuração para alterar
  determinado comportamento de uma aplicação, o implantador aplica configurações
  específicas a aplicação a partir de arquivos de configuração.
  \item  \textbf{Inicialização:} É quando a aplicação é iniciada e entra em execução,
  pronto para receber chamadas de seus clientes.
\end{itemize}

Dependendo to tamanho da aplicação essas fases podem se tornar tarefas complicadas,
com isso existe a necessidade de automatizar o processo de implantação, assim
diminuindo a possibilidade de erros comparado a uma tarefa manual e consequentemente
tornando o trabalho mais ágil, dando adeus aos longos manuais de instalação.
\cite{humble2010} diz que o objetivo de um processo de implantação automatizado é
proporcionar um processo de implantação reprodutível, confiável e fácil de ser
executado, o que casa perfeitamente no contexto de uma equipe devops.

Para que isso seja possível é necessário utilizar-se de ferramentas para poder
automatizar todo o processo de implantação, mas é extremamente importante intender todo o processo
para que possa ser feito um bom planejamento e execução, pois se não o uso da ferramenta também
trará dificuldades para executar o processo, principalmente se a equipe de devops não tiver conhecimento
da ferramenta utilziada. Por julgar importante o conhecimento prévio sobre ferramentas
para automatizar a implantação de software na próxima seção falaremos sobre algumas
ferramentas conhecidas para automazar implantação e suas maneiras de atuação.

\subsection{Métodos e ferramentas para implantação de software}
\label{subsec:metodoseferramentas}

%empacotamento, instalação via script, tar.gz, exe, jar, etc.

Várias soluções técnicas têm sido feitas para melhorar a implantação de software,
\cite{5741269} no entanto, apesar da sua importância para as empresas de software, as atividades
de implantação de software têm recebido pouca atenção. Em \cite{deployment1998}
é dito que recentemente, um número de novas tecnologias começaram a emergir para
resolver o problema de implantação. As características típicas oferecidas por
estas tecnologias incluem sistemas para automatizar a implantação a partir de
configurações, pacotes, gerenciamento de rede e instalação de recursos, com
propósito de entrega das atualizações de forma automática.

%importancia de automatizar a implantação
%falar das ferramentas
Um grande facilitador da implantação de software são as ferramentas que auxiliam na
automação das atividades de implantação, pois dependendo do software, a implantação
pode se tornar um processo com muitas atividades ou bastante longo, e a automação
vêm para facilitar a implantação, evitando erros de instalações manuais e facilitando a replicação em
vários ambientes diferentes.

De acordo com  \cite{6265084} tais ferramentas permitem-nos controlar e automatizar
a configuração de todos os elementos que compõem um sistema, são eles: hosts, software a serem instalados,
usuários do sistema, serviços em execução, arquivos de configuração, tarefas agendadas,
configuração de rede, armazenamento de arquivos, monitoramento e segurança. Existem
ferramentas populares como Puppet e Chef, tendo como sua principal função principal
de automatizar a instalação e configuração de um sistema. Com essas ferramentas é
possível escrever algumas regras que expressam como um software deve ser configurado,
e assima ferramenta irá configurar o sistema conforme as especificações.


.................... citações do trabalho do léo.... chef... etc


% ------------------------------ NEXT SESSION --------------------------

\section{Aplicações com multiplas instâncias}
\label{cap-multiplas}

nesse capitulo eu pretendo falar das vantagens de se utilizar aplicações com
multiplas instancias, como fazer isso e a arquitetura necessária para isso.

\subsection{Arquitetura de aplicações com multiplas instâncias}

Falar um pouco sobre como deve ser a arquietura, pastas, banco isolado,
configurações idependentes, etc.

\subsection{Hosts Virtuais}

Falar sobre isso, já que é extremamente necessário para poder ser possível a
instalação de multiplas instâncias de aplicações web.

% ------------------------------ NEXT SESSION --------------------------


\section{Aspectos para implatanção segura de aplicações}
\label{cap-seguranca}

nesse capitulo pretendo falar dos aspectos para implantação segura de software.

\subsection{Aspectos principais}

Os principais aspectos para implatanção segura.

\subsection{Técnicas e boas práticas}

Técnicas e boas práticas para implatanção segura de software.

\subsection{Testes ...}

falar sobre os testes que são possíveis de se fazer para averiguar tudo que foi
feito, quero que isso vá para os resultados ao final do tcc2.

%
