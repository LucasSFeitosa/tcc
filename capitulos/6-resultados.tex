\chapter{Resultados obtidos}
\label{cap-resultados}

Este capítulo aborda os resultados encontrados, a partir da proposta de
definição e preparação dos estudos. 

Como dito no capítulo \ref{cap-metodologia}
as aplicações piloto foram Owncloud e Wordpress, como visto na seção 
\label{subsection:validacao}. Cada fase da implantação será tratado como uma 
subseção, seguindo a escolha da ferramenta Shak como arquitetura inicial. Todo 
o trabalho foi organizado na ferramenta Gitlab no projeto Shak disponível no endereço 
\href{https://gitlab.com/Shak/Shak/} nele contém todo o código fonte disponível.

\section{Wordpress}
\label{sub:wordpress}

Wordpress é uma plataforma semântica de vanguarda para publicação pessoal, 
com foco na estética, nos padrões web e na usabilidade, ao mesmo tempo é 
um software livre. Além disso, wordpress é um dos maiores softwares de 
publicação de conteúdo \cite{wordpress}. 

O wordpress foi a primeira ferramenta escolhida para automatizar a instalação, partindo
dessa decisão deu início a construção da solução.

Primeiramente foram seguido os procedimentos para construção da solução definidos 
na seção \ref{section:construcao}, é
necessário definir as fases e procedimentos para a implantação automatizada,
seguindo as fases que compõem o processo de implantação de aplicações e de acordo
com a seção \ref{sec:fases}.

\subsection{Planejamento}

O planejamento da implantação é uma fase para identificar os componentes
necessários na implantação da aplicação. Definir quais são as dependências mínimas
para o funcionamento de uma aplicação, tais como: banco de dados, pacotes
pré-instalados e aplicações pré-configuradas. Para o wordpress foram escolhidas:

\begin{itemize}
   \item \textbf{Pacote wordpress para o Debian:} Pacote Debian com o wordpress.
   \item \textbf{Pacote Nginx para o Debian:} Pacote Debian com o Nginx.
   \item \textbf{Pacote mysql para o Debian:} Pacote Debian com o banco de dados mysql
   que será usado pelo wordpress.
   \item \textbf{Arquivos de configuração:} Criação dos arquivos de configuração
   necessários para configurar o wordpress, o servidor web Nginx e o banco de dados
   mysql.
\end{itemize}

A instalação e execução desses recursos serão feitas nas fases seguintes.

\subsection{Preparação e Instalação de Pacotes}
\label{wordpress:preparacao}

São os procedimentos necessários para preparar o ambiente alvo para que o wordpress
possa ser executado, isso envolve configuração do sistema operacional, instalação
e configuração de dependências necessárias, e a transferência do componente
para o servidor onde ele será executado.

Para o funcionamento correto do wordpress é necessário a instalação do php5, além
disso também é necessário algum servidor web, na arquitetura do Shak o servidor
web padrão é o Nginx, e o banco de dados escolhido foi o mysql.

Para solução desse procedimento bastou apenas instalar os pacotes php5-fpm, o pacote
do banco de dados mysql e o pacote nginx, além disso habilitar o serviço do
php5-fpm, todos esses passos são feitos antes da instalação do wordpress,
pois por default se ele não encontrar essas dependências ele instala o servidor
apache2.

Para executar essas instalações é preciso criar uma receita no livro de receitas
do wordpress. Primeiramente é necessário criar o próprio livro de receitas no Shak,
para isso existe o comando rake cookbook que cria a estrutura básica
de um livro de receitas que deve ser usado pelo Chef, como por exemplo:

\begin{lstlisting}[language=Ruby,label=dice_index,caption={Exemplo de criação de estrutura básica de livro de receitas do wordpress com shak}]
  rake cookbook
  Cookbook name: wordpress
  knife cookbook create wordpress -o cookbooks/
  ** Creating cookbook foo in /home/thiago/Shak/cookbooks
  ** Creating README for cookbook: wordpress
  ** Creating CHANGELOG for cookbook: wordpress
  ** Creating metadata for cookbook: wordpress
\end{lstlisting}

Assim com a estrutura inicial, é possível declarar as dependências do wordpress
dentro do arquivo default.rb dentro da pasta recipes. Nesse arquivo é aonde
será declarado os pacotes que devem ser instalados, os arquivos de configuração,
os serviços que precisar executar, as pastas, permissões de usuários, etc. Um exemplo
de como declarar pacotes e serviços dentro de uma receita Chef é:

\begin{lstlisting}[language=Ruby,label=dice_index,caption={Exemplo de criação de serviço do mysql com o chef}]
  package "mysql-server"
  service "mysql" do
    action :start
  end
\end{lstlisting}


\subsection{Configuração}
\label{wordpress:preparacao}

Como levantado no planejamento é necessário a edição de arquivos de configuração
do wordpress, arquivos de configuração do banco de dados e arquivo de configuração
do nginx.

O primeiro arquivo de configuração é o config.php. É nesse arquivo de
configuração onde ficam as informações de banco de dados como nome do banco de dados,
login do usuário do banco de dados, senha, o endereço do banco de dados e o diretório
onde irá ficar os arquivos estáticos, como temas, galeria de mídias e plugins 
de cada instância do wordpress.

Por padrão ele deve ficar no diretório /etc/wordpress/ e seu nome deve conter
a seguinte estrutura: config-nomehost.php, onde o nomehost deve ser o hostname
desejado. O segundo arquivo é o database.sql que é um pequeno script \textit{SQL} que
cria o banco de dados do Wordpress e dá os privilégios ao usuário desejado. Por fim
o arquivo de configuração do Nginx, com a configuração do servidor web.

Esses arquivos de configuração serão criados dentro da pasta template, com o conteúdo
desejado dos arquivos de configuração. Seguindo a mesma estrutura
inicial que foi criada em \ref{wordpress:preparacao}, e a ação de criação é
declarada dentro do arquivo default.rb, de forma semelhante como foi feito com
a declaração dos pacotes. Um exemplo de como gerenciar templates numa receita Chef
é:

\begin{lstlisting}[language=Ruby,label=dice_index,caption={Exemplo de criação de templates com o chef}]
  template "text\_file.txt" do
  source "text\_file.txt"
  mode "0755"
  owner "root"
  group "root"
\end{lstlisting}

\subsection{Múltiplas Instâncias}

O uso de múltiplas instâncias do wordpress pode ser feito de duas maneiras, e servem
para que o usuário possa ter várias instâncias de wordpress no mesmo servidor. No 
wordpress é possível ter múltiplos sites e múltiplas
instâncias, no caso do Shak é desejado que seja múltiplas instâncias, a partir da
repetição da instalação. O wordpress recomenda que se for necessário o uso de múltiplas
instâncias é necessário realmente instalar cada Wordpress, um por vez.

A arquitetura para isso não é muito complicada, porém algumas pastas do sistema
precisam ser divididas, ou seja, uma para cada aplicação, essa pasta é a pasta wp-content
que é a pasta aonde fica os uploads, os temas, os plugins e etc, portanto para não
existir conflito entre as instâncias essas pastas devem estar devidamente separadas
com seus caminhos referenciados em cada arquivo config.php, o arquivo config.php também
deve ser único para cada instância. 

Outro fator importante é que cada instância
tenha seu banco de dados, para evitar novamente os conflitos. Para
solucionar esse problema o Shak possui um recurso interessante onde cada aplicação
possui um atributo id, que é um atributo único para cada instância executada pelo
Shak, com isso é possível criar pastas personalizadas com o id da aplicação e também
bancos de dados específicos e arquivos config.php específicos.

Por fim basta que seja criado um arquivo de configuração Nginx para cada aplicação,
como visto no capitulo \ref{cap-referencial} todo arquivo de configuração
do Nginx possui um bloco server, e cada bloco server equivale a um virtual hosting,
por isso as aplicações serão também independentes a nível de servidor web sendo assim
possível o acesso entre elas separadamente.

\subsection{Inicialização}

Após a construção da receita do wordpress é necessário testar a receita construída. 
Para isso o Shak precisa de duas informações importantes, a primeira é a aplicação
que será instalada, e segunda o hostname destino. Como é um ambiente de desenvolvimento
não é preciso configurar um ip ou configurar um \textit{DNS}, é preciso apenas adicionar o
hostname desejado no arquivo /etc/hosts. Assim, mesmo que esteja na sua máquina local
é possível acessar um endereço mais familiar em seu navegador. Um exemplo da execução
da instalação via Shak é:

\begin{lstlisting}[language=Ruby,label=dice_index,caption={Exemplo de exexução de instalação do wordpress com shak}]
shak install wordpress hostname=wordpress.dev
\end{lstlisting}

E assim o Chef inicia o processo de instalação dos pacotes, criação dos arquivos
de configuração, incia os serviços desejados, e ao fim do procedimento a aplicação
já estará pronta para uso no hostname escolhido.

\section{Owncloud}
\label{sub:owncloud}

Owncloud é uma ferramenta para compartilhamento de arquivos, é um software 
livre em que é possível compartilhar
um ou mais arquivos e pastas do seu computador na nuvem, e sincronizá-los com o seu
servidor Owncloud. Esses arquivos são imediatamente sincronizados com o servidor
e disponibilizados para outros dispositivos que utilizam o ambiente de trabalho
Owncloud ou app Android ou app IOS \cite{owncloud} .

Também foram seguidos os mesmos passos feitos em \ref{sub:wordpress}, primeiramente
foram seguidos os procedimentos para construção da solução definidos em \ref{section:construcao},
definindo as fases e procedimentos para a implantação automatizada, seguindo as
fases que compõem o processo de implantação de aplicações definidos na seção
\ref{sec:fases}.

\subsection{Planejamento}

Seguindo a mesma linha do wordpress, é necessário definir quais são as dependências
mínimas para o funcionamento do Owncloud tais como: banco de dados, pacotes
pré-instalados e aplicações pré-configuradas. Para o Owncloud foram escolhidas:

\begin{itemize}
   \item \textbf{Pacote Owncloud para o Debian:} Pacote Debian com o Owncloud.
   \item \textbf{Pacote Nginx para o Debian:} Pacote Debian com o nginx.
   \item \textbf{Pacote Postgresql para o Debian:} Pacote Debian com o banco de dados postgresql
   que será usado pelo Owncloud.
   \item \textbf{Arquivos de configuração:} Criação dos arquivos de configuração
   necessários para configurar o Owncloud, o servidor web Nginx e o banco de dados
   postgresql.
\end{itemize}

A diferença da escolha do banco de dados em relação ao wordpress é poder aumentar
o suporte a diferentes bancos de dados, na documentação do wordpress é recomendado
o uso do banco de dados mysql, porém no Owncloud fica a escolha do desenvolvedor.
A escolha do postgresql vêm com a possibilidade de trabalhar com dois bancos de
dados diferentes, aumentando o suporte do Shak, assim quando existirem aplicações
que suportam apenas o postgresql o Shak já terá uma estrutura básica pronta.
A instalação e execução desses recursos serão feitas nas fases seguintes.

\subsection{Preparação e Instalação de Pacotes}

Para o funcionamento correto do Owncloud é necessário a instalação do php5, além
disso também é necessário que algum servidor web, na arquitetura do Shak o servidor
web padrão é o Nginx, e o banco de dados Postgresql.

Para solução desse procedimento foi necessário instalar os pacotes php5-fpm, o banco
de dados postgresql e o pacote nginx, além disso habilitar o serviço do php5-fpm,
também foi necessário instalar o pacote php5-pgsql que é um módulo para
conexões de banco de dados postgresql, necessário para o funcionamento de
aplicações php com o postgresql.

Para executar essas instalações é preciso criar uma receita no livro de receitas
do owncloud, bastando executar novamente o comando rake cookbook que cria a estrutura básica
de um livro de receitas que deve ser usado pelo Chef, como por exemplo:

\begin{lstlisting}[language=Ruby,label=dice_index,caption={Exemplo de criação de estrutura básica de livro de receitas do owncloud com shak}]
  rake cookbook
  Cookbook name: owncloud
  knife cookbook create owncloud -o cookbooks/
  ** Creating cookbook foo in /home/thiago/Shak/cookbooks
  ** Creating README for cookbook: owncloud
  ** Creating CHANGELOG for cookbook: owncloud
  ** Creating metadata for cookbook: owncloud
\end{lstlisting}

Assim com a estrutura inicial, é possível declarar as dependências do owncloud
dentro do arquivo default.rb dentro da pasta recipes. Nesse arquivo é 
declarado os pacotes que devem ser instalados, os arquivos de configuração,
os serviços que precisar executar, as pastas, permissões de usuários, etc. Um exemplo
de como declarar pacotes e serviços dentro de uma receita Chef é:

\begin{lstlisting}[language=Ruby,label=dice_index,caption={Exemplo de como habilitar serviço do postgresql com chef}]
  package "postgresql"
  service "postgresql" do
    action :start
  end
\end{lstlisting}

\subsection{Configuração}

Como levantado no planejamento é necessário a edição de arquivos de configuração
do owncloud, arquivos de configuração do banco de dados e arquivo de configuração
do Nginx.

O primeiro arquivo de configuração é o autoconfig.php, é nesse arquivo de
configuração onde ficam as informações, como nome do banco de dados,
login do usuário do banco de dados, senha, o host do banco de dados e o diretório
onde irá ficar os arquivos de configuração do owncloud, o login e senha
do administrador. Isso é necessário apenas para o primeiro acesso, ou seja, o usuário
fará o login automaticamente quando o owncloud terminar a instalação, após isso
a recomendação é trocar a senha do administrador.

Além disso é necessário criar a pasta de conteúdos públicos do owncloud, que por
padrão devem ficar em /etc/owncloud. O segundo arquivo é o postgresql-conf.sql
que é um pequeno script \textit{SQL} que cria o banco de dados do owncloud e dá os
privilégios ao usuário desejado. Por fim o arquivo de configuração do Nginx
configuração do servidor web.

\subsection{Múltiplas Instâncias}

Diferentemente do wordpress o owncloud não suporta múltiplas instâncias, porém isso
não foi um impeditivo na execução do trabalho, com uma busca encontrou-se uma 
discussão no repositório oficial do owncloud relacionado a implementação dessa 
funcionalidade. 

A discussão está disponibilizada em 
\href{https://github.com/owncloud/core/pull/16424}, nela existia
uma proposta de solução ao fato de não existir a opção de múltiplas instâncias. O
resultado desta discussão foi que os desenvolvedores do owncloud não acharam relevante
a funcionalidade, mas caso o desenvolvedor ache necessário, ele poderia fazer essa
alteração diretamente no código fonte.

Porém manter isso no Shak seria enviável, uma solução a esse problema seria enviar
uma contribuição ao pacote do owncloud no Debian, onde assim a contribuição feita
poderia ser facilmente utilizada por mais pessoas que queiram essa funcionalidade,
bastando utilizar a versão disponibilizada nos servidores do Debian. 

Com isso, foi feito a contribuição que adicionaria a funcionalidade de múltiplas 
instâncias para o owncloud, e enviado ao mantenedor do pacote do owncloud no Debian. 
A discussão está disponível em \href{https://bugs.debian.org/cgi-bin/bugreport.cgi?bug=789726},
a contribuição foi bem vista pelo mantenedor do pacote, e será incorporada na nova
versão do owncloud no Debian.

Em paralelo ao processo da contribuição estar disponível no novo pacote Debian,
foi necessário utilizar apenas instâncias únicas do owncloud, porém também foi
adicionado na receita do owncloud o suporte a múltiplas instâncias, portanto
assim que o owncloud suportar múltiplas instâncias via pacote Debian, será possível
também criar múltiplas instâncias do owncloud.

A solução feita na receita do owncloud foi baseada na contribuição enviada, portanto
os testes foram feitos gerando um novo pacote Debian do owncloud porém incorporando
a contribuição feita ao Debian. Os procedimentos feitos para suportar múltiplas instâncias
no owncloud são bem semelhantes aos do wordpress, criando um diretório de dados
para cada aplicação, um arquivo de configuração para cada aplicação e um banco de
dados para cada aplicação. Seguindo também a mesma abordagem do wordpress também
será criado uma hospedagem virtual para cada instância do owncloud.

\subsection{Inicialização}

Após a construção da receita do owncloud é necessário testar a receita construída,
para isso o Shak precisa de duas informações importantes, a primeira é a aplicação
que será instalada, e segunda o hostname destino, para executar a instalação
do owncloud basta:

\begin{lstlisting}[language=Ruby,label=dice_index,caption={Exemplo de execução de instalação do owncloud com shak}]
shak install owncloud hostname=owncloud.dev
\end{lstlisting}

E assim o Chef inicia o processo de instalação dos pacotes, criação dos arquivos
de configuração, incia os serviços desejados, e ao fim do procedimento a aplicação
já estará pronta para uso no hostname escolhido.


\section{Servidor de e-mail}
\label{sub:e-mail}

Servidores de e-mail formam a infraestrutura do e-mail, sendo o \textit{SMTP} o protocolo
mais importante, pois é o responsável por transferir as mensagens de servidores
de e-mail remetentes para servidores de e-mail destinatários \cite{kurose2010redes}, 

Porém o protocolo\textit{SMTP} é um protocolo do tipo de envio de informações, 
diferente do protocolo \textit{HTTP}
que é um protocolo que recupera informações, o que acarretaria um problema em que
o usuário não conseguiria recuperar suas informações como e-mails recebidos, já
que o \textit{SMTP} apenas envia informações.

Para solucionar esse problema, deve-se usar os protocolos de acesso a servidores de
e-mail, que recuperam as informações do servidor de e-mail até ao usuário final.
Como visto no capítulo \ref{cap-referencial} existem protocolos de acesso ao correio
como \textit{POP3}, \textit{IMAP} ou usar até mesmo o protocolo \textit{HTTP} 
servindo de protocolo para recuperar informações.

Tanto owncloud como wordpress possuem funcionalidades que utilizam serviços 
de e-mail, portanto
é necessário uma configuração mínima de e-mail para que funcione tais funcionalidades.
Assim, a construção de um servidor de e-mail possui bastante relevância, para que
 qualquer aplicação que precisar de um servidor de e-mail, possa utilizar 
o servidor de e-mail que o Shak fornece.

Para a construção do servidor de e-mail também foram seguidos os mesmos passos
feitos no wordpress na seção  \ref{sub:wordpress} e no owncloud na seção\ref{sub:owncloud}. 

\subsection{Planejamento}

No planejamento, é necessário definir
quais são as dependências mínimas para o funcionamento do servidor de e-mail, é 
necessário configurar um servidor que utiliza o procolo \textit{IMAP} ou \textit{POP3},
também é necessário a configuração de um agente de transferência de e-mails, e também
alguma ferramenta para controle de spam.

Primeiramente ficou definido que o protocolo escolhido seria o protocolo \textit{IMAP}, já
que o protocolo \textit{IMAP} é um protocolo online, ou seja, ele se conecta ao servidor
e realiza o download das mensagens, depois ainda  mantém a conexão para que
as alterações e mensagens novas sejam atualizadas em tempo real, diferentemente do
protocolo \textit{POP3} que é um protocolo offline, onde após o download das mensagens encerra
a conexão. Outra vantagem do \textit{IMAP} em relação ao \textit{POP3} é que o 
\textit{IMAP} mantém uma cópia de mensagens no servidor, ideal para quem precisa 
acessar os e-mails de mais de um local.

O servidor \textit{IMAP} escolhido foi o Dovecot, que é um servidor de e-mail
\textit{IMAP}, além disso, dovecot é um software livre simples de configurar, requer nenhuma
administração especial e ele usa muito pouca memória \cite{dovecot}. 

O agente de transferência de e-mails escolhido foi o Postfix, que é um software
livre para envio e entrega de e-mails. Sua escolha foi feita pela facilidade de
integração com o Dovecot. Postfix é quem cuidará do método de entrega de e-mail
utilizando \textit{SMTP} como protocolo de transferência de e-mails. 

Por fim a aplicação que servirá como anti-spam será o Apache SpamAssassin, que 
é uma plataforma anti-spam que que dá aos administradores de servidor de e-mail, 
um filtro para classificar e-mails e bloquear os e-mails que julgarem como spam \cite{spam}. 
As aplicações são:

\begin{itemize}
   \item \textbf{Pacote dovecot para o Debian:} Pacote Debian com o dovecot.
   \item \textbf{Pacote postfix para o Debian:} Pacote Debian com o postfix.
   \item \textbf{Pacote spamassasin para o Debian:} Pacote Debian com o spamassasin.
   \item \textbf{Arquivos de configuração:} Criação dos arquivos de configuração
   necessários para configurar o spamassasin, dovecot e postfix.
\end{itemize}

\subsection{Preparação e Instalação de Pacotes }

Para o funcionamento correto do servidor de e-mail, são necessários vários pacotes, 
é possível separar as dependências dos pacotes necessários para cada ferramenta
escolhida, para compor o servidor de e-mail. Primeiramente para o postfix são necessários
os pacotes postfix e bsd-mailx, além de habilitar o serviço do postfix, para o dovecot
é necessário instalar o pacote dovecot-imadp além de habilitar o serviço do dovecot,
para o spamassassin são necessários os pacotes spamassassin e o pacote spamc e ativar
o serviço do spamassassin.

\subsection{Configuração}

Como levantado no planejamento, é necessário a edição de arquivos de configuração
de todas as ferramentas, existem arquivos de configuração para cada uma delas, inclusive
o dovecot que precisa saber que estamos utilizando o postfix como agência de transferência
de e-mails.

Primeiramente, para essa configuração específica do dovecot, foi utilizado 5 arquivos
de configuração, o primeiro é o 10-mail.conf que é um arquivo de configuração para
indicar onde será o local no qual os e-mails ficarão. O segundo arquivo é o 10-ssl.conf
que é o arquivo em que possui os caminhos do certificado \textit{SSL} e da chave ssl, sendo o
certificado pode ser lido por todos e o arquivo de chave apenas para um usuário root
ou que esteja no grupo de usuários ssl-cert. 

O terceiro arquivo é o arquivo
11-postfix-auth.conf, arquivo responsável por indicar o caminho do arquivo de autenticação
do postfix. O quarto arquivo é o arquivo 20-imap.conf onde será indicado o tamanho
máximo de conexões por ip permitidas no servidor. Por fim o arquivo
20-disable-imap-non-ssl.conf que permitirá que o imap utilize sempre uma conexão
segura utilizando o \textit{IMAPS} (\textit{IMAP} + \textit{SSL}), assim bloqueando 
qualquer conexão ou tentativa de conexão insegura.

Para o posfix são apenas dois arquivos de configuração, o main.cf e o master.cf,
o main.cf é aonde se configura os parâmetros mínimos de configuração, esse arquivo
também contém as configurações que filtram os e-mails com spamassassin e 
algumas configurações do \textit{SMTP}. 

No master.cf definimos como um programa cliente se conecta a um serviço, e qual o
programa que é executado quando um serviço é solicitado. Neste caso habilitamos
o recurso do servidor de e-mail utilizar \textit{SSL} e autenticação segura, 
apontando o caminho dos arquivos de chave e certificados.

Por fim, a configuração do spamassasin. São necessários dois arquivos de configuração
o spamassassin que é o arquivo de configurações padrão para o spamassasin e o
local.cf que são as configurações locais. No spamassassin você habilita o spamassasin
e configura o caminho do arquivo de log, opções e habilitar a cronjob que será
executada de tempos em tempos. 

Já o local.cf possui um parâmetro importante, o
parâmetro required\_score. Esse parâmetro possui um nível de 0 a 10, em que 
são classificados os e-mails como spam, por padrão esse valor é cinco, porém se 
o usuário quiser aumentar o filtro pode aumentar para valores como 6 ou 7 até chegar em um nível de confiança em que o spamassasin cuide dos casos falsos-positivos, ou seja, 
e-mails que não são spam porém foram interpretados como spam.

\subsection{Inicialização}

Após a construção da receita do servidor de e-mail é necessário testar a receita construída,
para isso o Shak precisa de duas informações importantes, a primeira é a aplicação
que será instalada, e segunda o hostname destino, para executar a instalação
do servidor de e-mail basta:

\begin{lstlisting}[language=Ruby,label=dice_index,caption={Exemplo de execução de instalação do servidor de e-mailcom shak}]
Shak install e-mail hostname=owncloud.dev
\end{lstlisting}


E assim o Chef inicia o processo de instalação dos pacotes, criação dos arquivos
de configuração, incia os serviços desejados, e ao fim do procedimento a aplicação
já estará pronta para uso no hostname escolhido.

Para testar o funcionamento foram feitos dois procedimentos, o primeiro era utilizar
um cliente de e-mail para testes, o cliente de e-mail escolhido foi o mutt, utilizando
suas funções básicas de enviar e receber e-mails. O segundo passo foi realizar conexões
com o servidor de e-mail via telnet, com os seguintes comandos:

\begin{lstlisting}[language=Ruby,label=dice_index,caption={Exemplo de teste de conexão telnet no servidor imap}]
  telnet localhost imap
  telnet localhost imaps
\end{lstlisting}

Pela configuração do servidor de e-mail não será possível abrir uma conexão telnet
com o parâmetro imap, porém em imaps será possível, isso é útil para testar 
que o servidor de e-mail está impedindo conexões que não 
utilizem os protocolos criptografados.

\section{MoinMoin}
\label{sub:owncloud}

MoinMoin é uma ferramenta de construção
de wikis, com uma grande comunidade de usuários, podendo criar páginas web
facilmente editáveis. MoinMoin é software livre, alguns exemplos
de wiki que utilizam o MoinMoin são: Wiki do debian, Wiki do python e Wiki do apache,
contendo várias documentações relacionado aos seus softwares \cite{moin}. 

Para a implantação automatizada do MoinMon, foram seguidos os mesmos passos
feitos em wordpress \ref{sub:wordpress} e owncloud \ref{sub:owncloud}.

\subsection{Planejamento}

Primeiramente é necessário definir quais são as dependências
mínimas para o funcionamento do MoinMoin tais como: banco de dados, pacotes
pré-instalados e aplicações pré-configuradas. Para o MoinMoin são elas:

\begin{itemize}
   \item \textbf{Pacote python-moinmoin para o Debian:} Pacote Debian com o MoinMoin.
   \item \textbf{Pacote nginx para o Debian:} Pacote Debian com o nginx.
   \item \textbf{Pacote uwsgi para o Debian:} Pacote Debian com o servidor de aplicação web
uwsgi, com suporte para aplicações escritas em python, utilizando o plugin uwsgi-plugin-python,
também disponível como um pacote debian.
   \item \textbf{Arquivos de configuração:} Criação dos arquivos de configuração
   necessários para configurar o Uwsgi, o servidor web Nginx e o servidor web
uwsgi.
\end{itemize}

Diferentemente das outras aplicações web, moinmoin não utiliza banco de dados.

\subsection{Preparação e Instalação de Pacotes}

Para o funcionamento correto do moinmoin é necessário a instalação do python, além
disso também é necessário que algum servidor web, no caso, utilizaremos o nginx
junto ao uwsgi, também é necessário a configuração do moinmoin a partir de seus
arquivos de configuração, mywiki.py e farmconfig.py.

Para solução desse procedimento foi necessário instalar os pacotes uwsgi,
uwsgi-plugin-python, e o pacote do moinmoin, chamado python-moinmoin.
Para executar essas instalações é preciso criar uma receita no livro de receitas
do moinmoin, bastando executar novamente o comando rake cookbook que cria a
estrutura básica de um livro de receitas que deve ser usado pelo Chef, como por exemplo:

\begin{lstlisting}[language=Ruby,label=dice_index,caption={Exemplo de criação de estrutura básica de livro de receitas do moinmoin com shak}]
  rake cookbook
  Cookbook name: moinmoin
  knife cookbook create moinmoin -o cookbooks/
  ** Creating cookbook foo in /home/thiago/Shak/cookbooks
  ** Creating README for cookbook: moinmoin
  ** Creating CHANGELOG for cookbook: moinmoin
  ** Creating metadata for cookbook: moinmoin
\end{lstlisting}

Assim com a estrutura inicial, é possível declarar as dependências do moinmoin
dentro do arquivo default.rb dentro da pasta recipes. Nesse arquivo é aonde
será declarado os pacotes que devem ser instalados, os arquivos de configuração,
os serviços que precisar executar, as pastas, permissões de usuários, dentre outros.

\subsection{Configuração}

Como levantado no planejamento é necessário a edição de arquivos de configuração
do moinmoin, arquivos de configuração do uwsgi e arquivo de configuração
do Nginx.

O primeiro arquivo de configuração é o moin.wsgi, arquivo onde é necessário
informar, por exemplo, o caminho dos arquivos de configuração do moinmoin. No caso
de um pacote debian, esse caminho por padrão é  /etc/moin. Ainda relacionado
ao uwsgi, também é necessário editar o arquivo uwsgi.ini, que é o arquivo de configuração
do uwsgi para sua aplicação, nele é possível adicionar informações importantes, como
por exemplo aonde será o arquivo de log do uwsgi, aonde será criado seu socket,
e a quantidade máxima de requisições.

O primeiro arquivo de configuração é o mywiki.py, onde é necessário
informar informações básicas de sua wiki,  como o nome da sua wiki e seu diretório de dados. 
O segundo arquivo de configuração é o farmconfig.py, que é o responsável pelos 
links da barra de navegação e responsável por configurar múltiplas instâncias.

Além disso é necessário criar a pasta de conteúdos do moinmoin, que por
escolha devem ficar em /var/lib/moin, além disso para que o nginx funcione corretamente
com o uwsgi é necessário indicar o socket do uwsgi, que foi indicado no uwsgi.ini.

\subsection{Múltiplas Instâncias}

MoinMoin possui nativamente a funcionalidade
que permite a criação de múltiplas instâncias de wiki no mesmo servidor,
que é chamado de farmconfig, no qual você adiciona em um arquivo de configuração,
chamado farmconfig.py, com todas as wikis com seus respectivos identificadores 
e domínios. Assim
o MoinMoin consegue gerenciar as wikis separadamente. 

O grande desafio é fazer
com que o Shak entenda que existem outras wikis já configuradas, já que para
utilizar esta funcionalidade,
é necessário alterar o arquivo de configuração, ou seja, a cada nova instância
o arquivo deve ser atualizado com as novas informações de wiki, sem remover as
informações das outras wikis anteriormente inseridas. Para isso, é adicionado
numa lista de wikis, para que a cada vez que surja uma wiki nova apenas as
informações novas nessa lista são inseridas no arquivo farmconfig.py.

\subsection{Inicialização}

Após a construção da receita do owncloud, é necessário testar a receita construída,
para isso o Shak precisa de duas informações importantes, a primeira é a aplicação
que será instalada, e segunda o hostname destino, para executar a instalação
do owncloud basta:

\begin{lstlisting}[language=Ruby,label=dice_index,caption={Exemplo de execução de instalação do owncloud com shak}]
shak install moinmoin hostname=moinmoin.dev
\end{lstlisting}

E assim o Chef inicia o processo de instalação dos pacotes, criação dos arquivos
de configuração, incia os serviços desejados, e ao fim do procedimento a aplicação
já estará pronta para uso no hostname escolhido.

\section{Segurança}
\label{sub:seguranca}

Na implantação das aplicações web foram feitos dois procedimentos de segurança, o primeiro
é forçar as aplicações web a sempre utilizarem o protocolo \textit{HTTP} e a segunda foi forçar o
servidor de e-mail a não permitir a conexão via protocolos sem criptografia. Neste
caso sendo o protocolo imap utilizando o protocolo \textit{imaps}. Para que isso fosse possível
foi necessário gerar um certificado \textit{SSL}, certificados \textit{SSL} são necessários para
que um determinado serviço opere com suporte a conexão segura por meio de criptografia.
É de conhecimento do autor que a melhor forma é obter um certificado assinado
por uma certificadora registrada, porém inicialmente foi trabalhado apenas com certificados
autoassinados.

Para suprir a necessidade das aplicações, cada aplicação terá um certificado para
si, além disso o servidor também terá o seu certificado autoassinado, isso foi necessário
para que as aplicações possam ser instaladas em diferentes servidores, por
isso é necessário garantir que cada servidor possua o seu certificado. 

Para adicionar esse novo suporte ao Shak, foi necessário criar uma receita Chef,
para que possa gerenciar os certificados \textit{SSL}. Optou-se por utilizar a ferramenta
openssl para geração das chaves e certificados, o openssl é uma ferramenta de
implementação do Transport Layer Security (TLS) e Secure Sockets Layer (\textit{SSL}),
além de ser uma biblioteca de propósito geral de criptografia \cite{openssl}.

Esse novo componente no Shak é composto basicamente do pacote openssl, além da criação
de certificados autoassinados utilizando o openssl. Os certificados são gerados 
para cada aplicação e para o servidor, o padrão do caminho onde os certificados são
gerados é /etc/ssl/certs/hostname.pem onde hostname é o endereço do servidor
e o caminho onde as chaves são geradas é /etc/ssl/private/hostname.key.

Com as chaves geradas basta configurar as aplicações indicando os caminhos dos certificados
e das chaves, para forçar as aplicações web a utilizarem o \textit{HTTP} foi necessário fazer
uma configuração específica no servidor web Nginx:


\begin{lstlisting}[language=Ruby,label=dice_index,caption={Exemplo de arquivo de configuração do Nginx para aplicações web no shak}]
  server {
      server_name <%= @hostname %>;
      rewrite     \^https://\$server\_name\$request\_uri? permanent;
  }

  server {
      server_name <%= @hostname %>;

      listen 443 ssl;
      ssl_certificate       /etc/ssl/certs/<%= @hostname %>.pem;
      ssl_certificate_key   /etc/ssl/private/<%= @hostname %>.key;

      access_log            /var/log/nginx/<%= @hostname %>.access.log;
      error_log             /var/log/nginx/<%= @hostname %>.error.log;

      include /var/lib/Shak/etc/nginx/<%= @hostname %>/*.conf;
  }
\end{lstlisting}

Com isso, todas as requisições serão forçadas a utilizar a porta 443 que é a porta
\textit{TCP} padrão para sites que utilizam \textit{HTTP} e \textit{SSL}, assim as aplicações utilizarão
um protocolo mais seguro utilizando criptografia dos dados, utilizando o
certificado autoassinado que foi gerado para o servidor específico.

\section{Protótipo da aplicação web}
\label{sub:prototipo}

Outro resultado obtido neste trabalho foi a construção de um protótipo para a interface
web da aplicação, como um dos objetivos do Shak é possuir uma interface web para que
os usuários não precisem utilizar um terminal para realizar as ações, também foi
feito um protótipo funcional, para ser aplicado no shak. O protótipo foi feito
na ferramenta chamada Pingendo, que é um software livre utilizado para construir
protótipos funcionais. Além de ter um protótipo funcional, a ferramenta Pingendo
também gera o código \textit{HTML} e \textit{CSS}, que pode ser aproveitado para 
a construção do layout do Shak. A sugestão inicial foi feita utilizando \textit{HTML} e \textit{CSS} utilizando
bootstrap 3, que é um framework para criar aplicações responsivas na web. A versão
inicial está disponível em \href{https://gitlab.com/Thiagovsk/shak_frontend/tree/master}.


