\chapter{Resultados parciais}
\label{cap-resultados}

Este capítulo aborda sobre os resultados parciais que foram obtidos, de acordo
com a execução da metodologia proposta, como dito no capítulo\ref{metodologia}
as aplicações escolhidas para terem sua instalação automatizada foram Owncloud e
wordpress, seguindo a escolha da ferramenta Shak como arquitetura inicial. O
desenvolvimento do trabalho também seguiu a metodologia da programação extrema
(XP) para o desenvolvimento. Todo o trabalho foi organizado na ferramenta Gitlab
no projeto Shak disponível no endereço \href{https://gitlab.com/shak/shak/} nele
contém todo o código fonte e as milestones e issues. As issues são as atividades
definidas nas reuniões do jogo do planjemento e as milestones são o conjunto de
atividades definidas para serem executadas dentro do período de uma semana, todas
essas informações também estão disponíveis no endereço do projeto.

\section{Planejamento das Atividades}

O planejamento das atividades foi feito a partir das aplicações worpress e owncloud,
que foram escolhidas na seção\ref{subsection:exemplos} para servirem como exemplos
de uso, logo a meta inicial planejada foi conter a instalação as duas aplicações
de forma automatizada, levando em consideração todos os apectos definidos no capítulo
\ref{cap-metodologia}. As atividades iniciais levantadas foram:

 \begin{itemize}
   \item \textbf{Issue #1} Livro de Receitas para instalação do Wordpress.
   \item \textbf{Issue #7} Livro de Receitas para instalação do Owncloud.
   \item \textbf{Issue #8} Livro de Receitas para instalação do servidor de email.
   \item \textbf{Issue #9} Forçar Wordpress a utilizar https.
   \item \textbf{Issue #10} Forçar Owncloud a utilizar https.
   \item \textbf{Issue #11} Prototipação do front-end para a ferramenta Shak.
 \end{itemize}

Ao verificar que tanto a ferramenta owncloud como a ferramenta wordpress precisam
de configurações servidor de email para algumas funcionalidades, também foi adicionado
a atividade de criação de uma receita para a automatizar a configuração de
servidor de email. Como aspecto de segurança, ficou definido que tanto a aplicação
owncloud como wordpress deveriam usar sempre https por padrão, isso envolve também
uma extratégia de como serão gerenciados os certificados de segurança para aplicar
o protocolo https.Além dessas atividades previstas, também foram feitas evoluções
no Shak para que seja possível a implementação dessas atividades.Nas seções a
seguir serão apresentados os resutlados das atividades levantadas, a partir da
proposta da construção da solução vista no capítudo\ref{cap-metodoologia}.

\subsection{Wordpress}
\label{sub:wordpress}

De acordo com a documentação oficial\cite{wordpress} WordPress é uma plataforma
semântica de vanguarda para publicação pessoal, com foco na estética, nos
Padrões Web e na usabilidade e ao mesmo tempo é um software livre. WordPress é
um dos maiores software de publicação de conteúdo, sendo hoje a maior
plataforma de Gerenciamento de Conteúdo do mundo,com quase 70\% do mercado. O
wordpress foi a primeira ferramenta escolhida para automatizar a instalação, partindo
dessa decisão deu início a construção da solução.

Primeiramente foi seguido os procedimentos para construção da solução definidos em\ref{section:construcao}, é
necessário definir as fases e procedimentos paara a implantação automatizada,
seguindo as fases que compõem o processo de implantação de aplicações e de acordo
 com\cite{omg2006}, e isso será o que o Shak irá automatizar, cada fase desse processo
 será  tratada como uma subseção. No wordpress as fases e procedimentos são:

\begin{itemize}
  \item  \textbf{Planejamento}
  \item  \textbf{Preparação}
  \item  \textbf{Instalação de pacotes}
  \item  \textbf{Configuração}
  \item  \textbf{Inicialização}
\end{itemize}


\subsubsection{Planejamento}

O planejamento da implantação é uma fase para identificar os componentes
necessários na implantação da aplicação. Definir quais são as dependências mínimas
para o funcionamento de uma aplicação tais como: banco de dados, pacotes
pré-instalados e aplicações pré-configuradas. Para o wordpress foram escolhidas:

\begin{itemize}
   \item \textbf{Pacote wordpress para o Debian:} Pacote Debian com o wordpress.
   \item \textbf{Pacote Nginx para o Debian:} Pacote Debian com o nginx.
   \item \textbf{Pacote mysql para o Debian:} Pacote debian com o anco de dados mysql
   que será usado pelo wordpress.
   \item \textbf{Arquivos de configuração:} Criação dos arquivos de configuração
   necessários para configurar o wordpress, o servidor web Nginx e o banco de dados
   mysql.
\end{itemize}

A instalação e execução desses recursos serão feitas nas fases seguintes.

\subsubsection{Preparação}

São os procedimentos necessários para preparar o ambiente alvo para que o wordpress
possa ser executado, isso envolve configuração do sistema operacional, instalação
e configuração de dependências necessárias, e a transferência do componente
para o servidor onde ele será executado.

Para o funcionamento correto do wordpress é necessário a instalação do php5, além
disso também é necessário que algum servidor web, na arquitetura do Shak o servidor
web padrão é o Nginx.

Para solução desse procedimento bastou apenas instalar os pacotes php5-fpm e o
pacote nginx, além disso habilitar o serviço do php5-fpm, todos esses passos são
feitos antes da instalação do wordpress, pois por default se ele não encontrar
essas dependências ele instala o servidor apache2.

% TODO explicar como faz isso no shak...

\subsubsection{Instalação de pacotes}

Os únicos pacotes necessários para o instalar nessa fase é o próprio pacote do
wordpress e o pacote do mysql.

% TODO explicar como faz isso no shak...

\subsubsection{Arquivos de configuração}

Como levantado no planjeamento é necessário a edição de arquivos de configuração
do wordpress, arquivos de configuração do banco de dados e aqruivo de configuração
do nginx.

O primeiro aqruivo de configuração é o config.php, é nesse arquivo de
configuração onde ficam as informações de banco de dados como nome do banco de dados,
login do usuário do banco de dados, senha, o host do banco de dados e o diretório
onde irá ficar os temas, galeria de mídas e plugins de cada instância do wordpress.
Por padrão ele deve ficar no diretório /etc/wordpress/ e seu nome deve conter
a seguinte estrutura: config-nomehost.php, onde o nomehost deve ser o hostname
desejado. O segundo arquivo é o database.sql que é um pequeno script sql que
cria o banco de dados do wrodpress e dá os privilégios ao usuário desejado. Por fim
o arquivo de configuração do Nginx configuração do servidor web.

% TODO explicar como faz isso no shak...

\subsubsection{Inicialização}

% TODO explicar como faz isso no shak...

\subsection{Owncloud}
\label{sub:owncloud}

\subsubsection{Planejamento}
\subsubsection{Preparação}
\subsubsection{Instalação de Pacotes}
\subsubsection{Configuração}
\subsubsection{Inicialização}



\subsection{Servidor de Email}
\label{sub:email}

\subsubsection{Planejamento}
\subsubsection{Preparação}
\subsubsection{Instalação de Pacotes}
\subsubsection{Configuração}
\subsubsection{Inicialização}


\subsection{Protótipo de Front-End}
\label{sub:prototipo}

% TODO falar do protótipo de alta fidelidade, o que é, como fiz.
