\chapter{Metodologia}
\label{cap-metodologia}

#TODO
================= Precisa falar da pesquisa? ex: qualitativa, descritiva.. etc ============/============

Este capítulo aborda sobre tudo que compõe a meotodologia de pesquisa utilizada
durante o trabalho, a fim de explciar como o trabalho foi realizado para atingir os
objetivos, servindo como base para a sua reprodução em trabalhos futuros.

A questão de pesquisa a ser respondida neste trabalho é:

\begin{center}
  \textit{
    Como implantar multiplas aplicações a partir de um pacote único
    de forma centralizada e segura?
}
\end{center}

Tendo a questão de pesquisa como insumo, foram levantadas algumas hipóteses que devem
ser estudadas e avalidas, como:

\begin{itemize}
  \item  \textbf{H1 :} É possível criar varias instâncias de aplicações
  web a partir de um pacote único no mesmo servidor.
   \item  \textbf{H2 :} Existe uma arquitetura definida para implantação de
   múltiplas instâncias de aplicações a partir de um pacote único que levam em consideração
   os aspectos de segurança em uma implantação automatizada de sofware.
\end{itemize}

Para responder H1 seŕa necessário fazer uma busca nas aplicações web que são
empacotadas no debian e que possuem suporte a configuração de múltiplas instâncias,
essa busca deve levar em consideração também a documentação para realizar tal
configuração.

Para responder H2 será necessário fazer uma pesquisa em ferramentas ou tecnologias que
permitem a configuração de múltiplas instâncias de aplicações no mesmo servidor e com
isso ver a proposta de arquitetura feita por tais ferramentas ou tecnologias. É
possível que também possam existeir propostas de arquiteura mas sem possuir uma
ferramenta ou tecnologia que execute tal arquitetura.

Também é importante para responder H2 definir quais são os aspectos de segurança em
uma implantação de software dentro do contexto de implantação automatizada de múltiplas instâncias de aplicações no mesmo servidor,
com isso ver o impacto causado na arquitetura de implantação, levando em consideração os
aspectos comuns entre as aplicações, lidar com diferentes linguagens e suas características
dentro de uma implantação segura e centralizada.

#TODO ver como está ficando e tirar duvidas

\section{Trabalhos Relacionados}

Nesta sessão falaremos sobre os trabalhos relacionados, antes de prosseguir no
trabalho é importante intender o que acontece no mundo sobre o assunto no qual
está sendo abordado, por isso essa seção é dividida em duas subsseções em que uma
contém alguns trabalhos acadêmicos relacionados ao tema e outra que contém
ferramentas disponíveis no mercado.

\subsection{Trabalhos Acadêmicos Relacionados}

\begin{itemize}
  \item  \textbf{Implantação Automatizada de Composições de Serviços Web de Grande Escala :}
  Este trabalho feito por \cite{leo2014} foi desenvolvido um sistema de middleware
  chamado CHOReOS Enactment Engine que é um sistema que possibilita a implantação
  distribuída e automatizada de composições de serviços web em uma estrutura virtualizada,
  no qual opera no modelo computacional conhecido como plataforma como serviço,
  comparando esse sistema com abordagens ah-hoc de implnatação levando em consideração
  a escalabilidade em relaçaõ ao tempo de implantação das composições do serviços.
\end{itemize}

\subsection{Ferramentas do Mercado Relacionadas}

\begin{itemize}
  \item  \textbf{Bitnami :}
  De acordo com \cite{bitnami} BitNami é uma biblioteca de aplicativos de
  servidor populares e ambientes de desenvolvimento que pode ser instalado
  com um clique. Ele automatiza todo o processo de  compilar e configurar os
   aplicativos e todas as suas dependências (bibliotecas de terceiros,
  linguagem de programação, bases de dados) para que o usuário comum não
  se preocupe com questões mais técnicas. O software empacotado resultante é
  então disponibilizado como instaladores nativos, máquinas virtuais e imagens de nuvem.
  Estes pacotes de aplicativos BitNami proporcionar uma experiência ao usuário final
  de forma consistente, seguro e otimizado ao implantar qualquer aplicativo, em qualquer plataforma.

  \item  \textbf{Sandstorm.io :}
  De acordo com \cite{standstormio} Sandstorm é uma plataforma de código
  aberto para servidores pessoais. Sandstorm permite que você execute o seu próprio
  servidor e instalar aplicativos como EtherCalc, GitLab, Etherpad, Wekan, tão
  facilmente como você instalar aplicativos em seu telefone.
  Sandstorm lida com o login e controle de acesso, e pode executar qualquer aplicação
   web que roda em Linux,  embora alguns aplicativos podem precisar de mudanças para trabalhar
  dentro da sandbox Sandstorm.

 \item  \textbf{Sandstorm.io :}
 De acordo com \citeJuju Juju permite-lhe implementar,
 configurar, gerenciar, manter e serviços de forma rápida e eficiente escala nuvem
 em nuvens públicas, bem como em servidores físicos, OpenStack e contêineres.
 O mecanismo central por trás Juju é chamado charms ("encantos"). Charms podem
 ser escritos em qualquer linguagem de programação que pode ser executado
 a partir da linha de comando. Os charms são feitos pela comunidade do software juju
 que é quem cuida dos processos de instalação e configuração dos serviços sendo orquestrados.

\end{itemize}

\subsection{Ferramentas do Mercado Relacionadas}

Uma desvantagem do bitnami é que ele é uma ferramenta de código fechado, o que
dificulta para uma análise sobre sua arquitetura.

Uma desvantagem do standstorm.io é que ele não utiliza pacotes debian.


\section{Teste das Hipóteses}

#TODO

Para o teste de H1 foram levantados algumas aplicações web empacotadas no debian
da seguinte forma....  .....

Como podemos ver na documentação das aplicações XYZ...

Logo podemos ver que nem todas as aplicações empacotadas no debian possuem
suporte a múltiplas instâncias.....


Para o teste de H2 foi feito primeiramente uma pesquisa dentro das ferramentas
disponíveis e que são software livre....

A pesquisa foi feita de seguinte maneira....

Logo ao realizar a análise apresentada a cima, chegou-se a conclusão de que
é necessário a definição de uma arquitetura para implantação de múltiplas
instâncias de aplicações a partir de um pacote único. Com isso resolveu-se
realizar uma proposta de arquitetura para implantação dessas apliacções, levando
em consideração os aspectos de segurança no que tangem essa arquitetura.

\section{Detalhes da Proposta}

A proposta deste trabalho é trazer uma solução para a implantação automatizada
de múltiplas aplicações a partir de um pacote único de forma centralizada e segura.
Para isso será primeiramente necessário definir uma arquitetura padrão para a
implantação dessas aplicações, ou seja, definir:

\begin{itemize}
  \item  \textbf{Tecnologia para automatizar implantação :}  Definir qual será a
  tecnologia escolhida dentre as citadas no referencial teórico para dar suporte
  a implantação.
  \item  \textbf{Conjunto mínimo de dependências:} Definir quais são as dependências
  mínimas para o funcionamento das aplicações escolhidas, tais como: sistema operacional,
  pacotes pré-instalados e aplicações pré-configuradas.
  \item  \textbf{Procedimentos para implantação:} Definir quais são etapas da implantação,
  definir a ordem necessária para a execução de cada fase da implantação.
  \item  \textbf{Aspectos de segurança} Definir quais são os aspectos de segurança
  e automatizar os que forem possíveis de serem aplicados dentro do contexto desejado.
\end{itemize}

Com a arquitetura definida o próximo passo é definir um ambiente de desenvolvimento e de
testes, dado o contexto de implantação de software é importante possuir um abiente fexível para
testar facilmente a instalação e configuração das aplicações e podendo facilmente
reinicializar esse ambiente de forma que ele fique limpo sem resquícios da instalação
anterior.

Com a arquitetura definida e com um ambiente de desenvolvimento e de testes configurados, é
necessário definir quais são as ferramentas que servirão como caso de testes
para a implementação da solução. Elas serão importantes pois a partir delas
e de suas necessidades que será implementado a primeira versão da proposta para
a solução. Por isso também é importante definir aspectos que servirão como parâmetro
para ajudar no desenvolvimento da solução. Esses aspectos são:

\begin{itemize}
  \item  \textbf{Aplicações empacotadas no debian :}  Como o intuito do trabalho
  é realizar implantações múltiplas a partir de um pacote único, tais aplicações
  devem estar empacotadas e disponíveis para instalação nos servidores do debian.
  Isso impacta na escolha da ferramenta, visto que não será necessário ter o trabalho
  de empacotar aplicações que ainda não estão empacotadas no debian.
  \item  \textbf{Servidor web compatível:} As ferramentas escolhidas devem no
  mínimo possuir o servidor web compatível, por exemplo: as aplicações juntas
  devem possuir suporte para nginx ou apache ou similares.
  \item  \textbf{Aplicações com comunidades ativas:} Como estamos trabalhando
  com software livre, é importante que os softwares escolhidos possuem comunidades
  ativas, isso pode ajudar na resolução de  possíveis problemas, logo aplicações
  desconhecidas devem ser evitadas, e aplicações conhecidas devem ser priorizadas.
  E isso também pode ser um fator importante caso durante os testes sejam descobertos
  bugs ou melhorias dentro dessas ferramentas.
  \item  \textbf{Documentação do software:} A documentação do software também deve
  ser levado em consideração, principalmente a documentação da instalação e configuração
  dentro da arquitetura definida.
\end{itemize}

#TODO
Por fim, com a arquitetura definida, ambientes definidos e ferramentas escolhidas
é mão na massa... como escrever isso?

\section{Ferramentas e Dados}

#TODO
vou falar aqui sobre o shak, sobre o summer of code? sobre a participação do terceiro
nas decisões técnicas.

\section{Atividades}

#TODO
Com intuito de ...., foi definido um fluxo das atividades necessárias para organizar
o trabalho. .....

%
