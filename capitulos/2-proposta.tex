\chapter{Definição e Preparação dos estudos}

\label{cap-metodologia}
Este capítulo aborda sobre todo o pĺanejamento para a execução do projeto,
contendo os procedimentos e técnicas utilizados no trabalho, a fim de
explicar como o desenvolvimento foi realizado para atingir os seus objetivos,
servindo como base para a sua reprodução em trabalhos futuros.

O trabalho é fundamentado na construção de uma solução utilizando conceitos e práticas
da engenharia de software, com a definição de um objetivo, a definição de um panejamento,
e a execução das atividades planejadas, e posteriormente a validação dos resultados.
O objetivo deste trabalho é resolver a seguinte questão:

\begin{center}
  \textit{
  Como implantar aplicações web em sistema Debian GNU/Linux de forma automatizada e
  e segura?
}
\end{center}

Dado objetivo e a questão problema, este trabalho consiste em contribuir
com a construção de uma solução de engenharia de software que permita implantar
aplicações web em sistemas Debian GNU/Linux de forma automatizada e segura, essa
construção será feita a partir dos conhecimentos adquiridos durante
o curso de engenharia de software, além disso a solução proposta deverá passar
por uma validação dos resultados, baseada na observação
da execução da solução em exemplos de uso, onde cada exemplo conterá uma aplicação
real que deverá ser implantada com sucesso. Essa validação tem seus procedimentos
definidos na seção \ref{subsection:validacao} e as definições dos exemplos de uso
estão na seção \ref{subsection:exemplos}.

Inicialmente será feito uma pesquisa nos trabalhos relacionados para entender
melhor o problema, essa pesquisa é uma pesquisa aplicada pois tem como objetivo
gerar conhecimentos para aplicação prática \cite{gerhardt2009metodos},
relacionada a aplicações que auxiliam na implantação automatizada de software.

A partir dessa pesquisa será possível encontrar os trabalhos relacionados, os
trabalhos encontrados também serão de insumo para iniciar
a construção da solução a partir da análise dos resultados que foram encontrados,
as próximas seções tratam dos trabalhos relacionados encontrados e dos métodos e
procedimentos para construção da solução, além da validação da solução.

\section{Trabalhos Relacionados}
\label{section:trabalhos_relacionados}
Para responder a questão problema primeiramente deve ser feito um levantamento
de trabalhos relacionados, para entender melhor como a automação da implantação
de software está sendo utilizada como solução na academia e nas empresas, e
quais são os procedimentos e técnicas utilizados.

Para entender o que acontece na comunidade da computação em relação a implantação
automatizada de aplicações, foi feito uma busca de alguns trabalhos relacionados, o primeiro
trabalho relacionado é um trabalho acadêmico feito por \cite{leo2014} no qual foi
desenvolvido um sistema de middleware chamado CHOReOS Enactment Engine que é um
sistema que possibilita a implantação distribuída e automatizada de composições
de serviços web em uma estrutura virtualizada, no qual opera no modelo
computacional conhecido como plataforma como serviço, comparando esse sistema
com abordagens ad-hoc de implantação levando em consideração a escalabilidade
em relação ao tempo de implantação das composições dos serviços.

Existem também ferramentas disponíveis no mercado que também trazem a proposta
de automação de instalação de aplicações alguns exemplos são: \cite{bitnami}
BitNami que é uma biblioteca de aplicativos populares e ambientes de
desenvolvimento que podem ser instalados com apenas um clique, através de uma interface web
amigável. Ele automatiza todo o
processo de compilar e configurar os aplicativos e todas as suas dependências
(bibliotecas de terceiros, linguagem de programação, bases de dados) para que o
usuário comum não se preocupe com questões técnicas. Também temos o \cite{sandstormio}
Sandstorm é uma plataforma de código aberto para servidores
pessoais. Sandstorm permite que você execute o seu próprio servidor e instalar
facilmente as aplicações em que dão suporte, e por fim a ferramenta \cite{juju}
Juju que lhe permite implementar, configurar, gerenciar, e manter serviços de forma
rápida e eficiente automatizando a instalação e configuração de aplicações na nuvem.

O estudo dessas aplicações serve como base para entender como funcionam as aplicações
já existentes, e como elas são feitas, para compreender as ferramentas que são
utilizadas para solucionar problemas semelhantes ao problema deste
trabalho, por exemplo a ferramenta JuJu utiliza uma abstração conhecida como charm,
que é basicamente um código, e esse código define um serviço, como por exemplo,
instale um banco de dados ou instale a aplicação wordpress. Um charm
contém toda a lógica de que você precisa para implementar e integrar uma aplicação,
contendo todo o processo de download de ferramentas, instalação e configuração de
aplicações, podendo ser por provisionadores como chef, puppet ou docker \cite{juju},
é possível reaproveitar os charms feitos pela comunidade do JuJu assim
reaproveitando os charms prontos ou não precisando escrever um charm do zero.

Já bitnami já trás as aplicações
prontas pra uso, assim o usuário interage no formato clique para instalar, porém
o usuário só tem a disposição as aplicações disponibilizadas pelo bitnami, isso
também é a forma em que o sandstorm.io trabalha, com a diferença de que o standstorm é
software livre, já Bitnami tem o código fechado, o que dificulta a análise da arquitetura
que é utilizada por eles. Por fim o trabalho feito por \cite{leo2014} é voltado
para serviços web de grande escala, com foco em implantação de aplicações na
nuvem e escalar infraestrutura, utilizando o middleware construído em seu trabalho,
 assim podendo gerenciar ambientes de computação em nuvem
 como Amazon EC2 e Openstack, sua engine utiliza o chef solo como seu agente de configuração.

Com esse levantamento, as soluções encontradas não resolvem totalmente
o problema, a grande diferença é que as ferramentas vistas não utilizam os pacotes
disponibilizados pelo Debian, mas elas servem como motivação para o desenvolvimento
deste trabalho. Como conclusão deste levantamento, vimos que tais ferramentas buscam
abstrair todo processo de instalação e configuração das ferramentas, para que o
usuário não tenha dificuldades na instalação de aplicações web, as ferramentas também
trazem uma interface web para que o usuário também não precise executar tarefas via terminal.
Outra informação importante é que as ferramentas também utilizam do conceito de
infraestrutura como código, como visto no capítulo \ref{cap-referencial}, e para
isso utilizam provisionadores como chef ou puppet, tornando mais simples a
construção de scripts para a instalação e configuração de aplicações.

Tendo em vista que pelo levantamento dos trabalhos relacionados, o problema inicialmente
proposto não foi resolvido, justifica-se a construção de uma solução de engenharia
de software para implantação automatizada de aplicações em sistemas Debian GNU/Linux.

\section{Construção da solução}
\label{section:construcao}

A partir da análise econtrada nos trabalhos relacionados, é necessário definir
como será feito a construção da solução, primeiramente será definido as ferramentas
de apoio ao desenvolvimento da solução, após isso, a escolha da ferramenta que será
utilizada para solução do problema, essa será a ferramenta que será evoluida
durante este trabalho. Após a escolha das ferramentas, será descrito como será
feito a validação da solução nos exemplos de uso com aplicações web escolhidas que possuem
seus pacotes disponibilizados pelos Debian.

\subsection{Ferramentas de apoio}

As ferramentas de apoio ao desenvolvimento de software são importantes para a
organização do trabalho, auxiliando nas atividades típicas dentro de um projeto
de engenharia de software. Algumas ferramentas são importantes pois tornam mais fácil
a execução de algumas atividades, como por exemplo: o controle de versão do software,
ferramenta para documentação do software e ferramenta para gerenciar as tarefas.
Os critérios para escolha das ferramentas são:

\begin{itemize}
  \item \textbf{Forge para sistema de controle de versão git:} Uma ferramenta que
  possa gerenciar diversas versões do desenvolvimento do código fonte utilizando
  o sistema de controle de versão git, que seja de preferência um software livre
  ou que não cobre o serviço de hospedagem. Algumas ferramentas disponíveis para
  isso são github, gitlab e bitbucket.
  \item \textbf{Ferramenta para documentação de código:} Uma ferramenta que possa
  documentar o projeto, os forges como gitlab e github possuem uma wiki já disponível
  para o projeto.
  \item \textbf{Ferramenta de gerenciamento de tarefas:} Uma ferramenta que possa
  gerenciar as tarefas que serão executadas, que estão em execução ou que vão ser executadas
  durante o desenvolvimento do projeto. Essa ferramenta será importante para a
  organização do trabalho, dando visibilidade do trabalho que está sendo realizado.
  \item \textbf{Ferramenta de comunicação:} A ferramenta de comunicação será
  importante para tirar dúvidas rápidas em relação a dificuldades e desafios, as
  comunidades de software livre costumam usar listas de e-mail e canais no IRC
  para a comunicação de seus desenvolvedores e usuários.
\end{itemize}

As ferramentas escolhidas para o apoio ao desenvolvimento foram:
\begin{itemize}
  \item \textbf{Forge Git:} O Gitlab foi forge escolhido para o armazenar o
  repositório git, é semelhante ao github, porém é software livre, distribuído pela
  licensa mit\cite{gitlab}.
  \item \textbf{Documentação de Código:} Toda documentação será disponibilizada
  na wiki do projeto no gitlab.
  \item \textbf{Gerenciamento de Tarefas:} O gerenciamento de tarefas também será
  feito pelo projeto do gitlab, onde é possível criar atividades para serem feitas
  e criar marcos podendo adicionar as atividades a esses marcos. Como o ciclo de
  desenvolvimento será de uma semana, serão criados marcos para cada semana e atividades
  agrupadas em cada marco.
  \item \textbf{Comunicação:} A comunicação será feita via IRC, é uma ferramenta
  de comunicação no formato de chat com canais e usuários dentro dos canais,
  apesar de ser uma ferramenta antiga ainda é muito usada pelas comunidades de software
  livre e grandes empresas como facebook \cite{artigofacebook}.
\end{itemize}

\subsection{Ferramenta Shak}

De acordo com a motivação da contribuição em software livre
e a participação do google summer of code, como dito na seção de motivação
\ref{sec:motivacao}, e todo o contexto definido na seção \ref{section:construcao},
a ferramenta para apoio a implantação de software escolhida
é a ferramenta Shak (Self Hosting Applications Kit).

A ferramenta Shak tem o objetivo de facilitar ao máximo que usuários sem conhecimento
técnico possam ter os seus próprios serviços de internet, garantindo sua privacidade
e segurança. Esta plataforma está concebida de acordo com os seguintes princípios:

\textbf{Base: pacotes Debian}

Sistemas operacionais livres, como o Debian
GNU/Linux , permitem que usuários com conhecimento
técnico para tal possam hospedar os seus próprios serviços de internet,
utilizando exatamente a mesma tecnologia que é empregada nos maiores sites
profissionais. Todo tipo de serviço pode ser hospedado com sistemas
operacionais livres, mas esta realidade ainda está longe do usuário final.

De acordo com \cite{shak2015}, a plataforma Shak é baseada no sistema de pacotes
do Debian, uma das distribuições GNU/Linux mais consolidadas e com maior comunidade internacional. Os
pacotes do Debian fornecem atualizações consistentes de correção e de segurança,
e são utilizados por uma grande quantidade de usuários que reportam problemas a
serem resolvidos. Porém, para garantirem uma flexibilidade e suportar diferentes
casos de uso, normalmente os pacotes não realizam a configuração completa do ambiente,
deixando os detalhes finais para o administrador do sistema.

Além disso, normalmente aplicações são compostas de diversos pacotes, e
requerem configurações que dizem respeitos a mais de um deles e que por consequencia não poderiam ser
feitas automaticamente de uma forma sustentável em nenhum dos pacotes.

\textbf{Nova abstração: Aplicação}

Para resolver as questões que não podem ser resolvidas a nível de pacotes, o
Shak introduz uma nova abstração: a aplicação. Uma aplicação geralmente
contém mais de um pacote, inclusive necessita de configurações em vários pacotes.

Uma vez que o usuário seleciona uma determinada aplicação para
ser instalada, os pacotes necessários são instalados e as configurações
necessárias são feitas de forma automática, fornecendo de fato um "instalador
de um clique" para aplicações suportadas. Todas as configurações são feitas de
forma consistente utilizando um framework comum, buscando o maior nível de
segurança possível, e assumindo responsabilidades que não podem ser tomadas por
um pacote.

\textbf{Diferencial para outras soluções existentes}

Existem outras soluções disponíveis para instalação de aplicações por um clique no
mercado, como por exemplo \cite{bitnami} e \cite{sandstormio}, mas o shak se
diferencia delas nos seguintes aspectos:

\begin{itemize}
  \item O shak pode ser implantando em um VPS na nuvem ou em servidores físicos
    próprios, a critério do usuário.

  \item O shak reutiliza o trabalho dos mantenedores Debian, que fornecem pacotes
    de alta qualidade.

  \item O shak reutiliza a infraestrutura do projeto Debian.

\end{itemize}

Por outro lado, apenas software que esteja empacotado no Debian estará
disponível, mas isso pode ser visto de forma positiva também: o shak
fornecerá um incentivo para que ainda mais software esteja disponível no
Debian, o que é um benefício coletivo para o ecossistema do software livre, e
para os seus usuários. Logo, quanto mais aplicações disponíveis no Debian,
mais aplicações serão disponibilizadas para os usuários do Debian e seus derivados,
como ubuntu e outros.

O Debian possui espelhos do seu repositório espalhados por todo o mundo,
fornecendo links para downloads mais rápidos aonde quer que o usuário esteja.
Atualizações de defeitos e correções de segurança realizadas no Debian estarão
automaticamente disponíveis para usuários do shak de forma transparente. Além disso,
o próprio shak estará disponível como um pacote dentro do próprio repositório oficial do Debian.

A arquitetura da ferramenta Shak é compsota de:

\begin{itemize}
  \item  \textbf{Livro de Receitas Chef:} Shak contém alguns livros de receitas
  para poder organizar a instalação de cada componente, como por, exemplo um livro
  de receitas para cada aplicação que for instalada.
  \item  \textbf{Código Ruby} A arquitetura do Shak foi desenvolvida na linguagem
  ruby, com programação orientado a objetos.
  \item  \textbf{Servidor Web} O Shak utiliza o software livre Nginx, que é
  servidor HTTP de alto desempenho \cite{nginx}.
  \item  \textbf{Pacotes Debian} O Shak utiliza pacotes incluídos na distribuição
  oficial do Debian.
  \item  \textbf{Gems} O shak também utiliza algumas gems para seu funcionamento,
  uma gem nada mais é do que uma biblioteca, que provê um formato padrão para
  a distribuição de programas Ruby \cite{gem}.
  \item  \textbf{Código Shellscript} O shak também utiliza algumas gems para seu funcionamento,
  uma gem nada mais é do que uma biblioteca, que provê um formato padrão para
  a distribuição de programas Ruby \cite{gem}.
\end{itemize}

\subsection{Criação de Ambientes de Desenvolvimento}

Uma das necessidades para o desenvolvimento do projeto é ter um ambiente de desenvolvimento
flexível que possa ser rapidamente construído e destruído, onde possa testar as
implantações feitas e verificar erros caso aconteçam. Para isso é necessário utilizar
alguma ferramenta que automatize esse processo, construir ambientes manualmente pode
ser um processo longo e demorado, além disso evitar que ocorra erros por
desatenção ao executar vários passos manuais, assim para automatizar a implantação
das aplicações é necessário o uso de uma ferramenta para auxiliar na criação das
máquinas virtuais onde é possível gerenciá-las de forma automatizada servindo de
suporte para a criação dos ambientes de desenvolvimento, facilitando a execução e teste da implantação das
aplicações escolhidas.

A ferramenta escolhida para auxiliar a criação de um ambiente de desenvolvimento é
a ferramenta Vagrant, com o Vagrant pode-se gerenciar a criação de máquinas
virtuais para os ambientes de desenvolvimento do projeto, também é possível usar
ferramentas de automação para criar o servidor de implantação, definindo tudo
em arquivos que farão parte do repositório git do Shak. Dessa maneira,
qualquer pessoa que clonar o repositório poderá rodar uma instância desse servidor, que será exatamente
idêntico para todos os usuários que forem criar uma máquina virtual, reduzindo ao
máximo os problemas que surgem por causa das diferenças entre sistemas operacionais e configurações de sistema.

Quando executamos o comando vagrant up o Vagrant cria uma máquina virtual com as
configurações que estão no arquivo Vagrantfile na raiz do projeto Shak, nele
estão todas as configurações e definições da máquina virtual em questão. Para o
Vagrant criar uma máquina virtual era necessário escolher um software que gerencie
esse procedimento, a ferramenta escolhida foi o virtualbox para poder emular o sistema
operacional em uma máquina virtual, assim o Vagrant inicia uma máquina virtual
utilizando o virtualbox, com a distribuição linux e as configurações que estiverem
definida no arquivo de configuração Vagrantfile.

Para executar o Vagrant basta instalar o pacote disponível pelo Debian, além disso
o repositório do projeto Shak já possui um Vagrantfile padrão para que todos os
desenvolvedores tenham o mesmo ambiente, passa acessar a máquina virtual criada
basta acessá-la via ssh, o Vagrant já gerencia as chaves ssh, portanto basta
dar o comando vagrant ssh que estará conectado a máquia virtual via ssh.
O ambiente de desenvolvimento levantado é um ambiente com Debian na sua versão
sid 64 bits(que é a versão que contém os pacotes mais atuais, conhecida como versão instável),
apesar de a versão ser a versão instável isso é importante para que possamos
utilizar as versões mais novas dos pacotes das ferramentas escolhidas, também
será importante para reportar algum bug caso seja encontrado.

Com o ambiente de desenvolvimento definido, agora é possível iniciar a criação dos
livros de receitas das aplicações, o repositório do Shak possui alguns scripts
para instalar algumas dependências ao iniciar uma máquina virtual, dentro
desses scripts contém as instalações das dependências para o Shak funcionar perfeitamente
e também o próprio Chef, que é utilizado pela ferramenta Shak para instalar aplicações
e suas dependências.


\section{Planejamento das Atividades}

As atividades planejadas são baseadas nas aplicações que foram escolhidas na seção
\ref{subsection:exemplos} serão as aplicações piloto, que ajudarão na evolução
da solução proposta, o planejamento inicial é que a ferramenta Shak possa concluir com sucesso
a instalação do Owncloud e Wordpress de forma automatizada,
levando em consideração todos os aspectos definidos no capítulo
\ref{cap-metodologia}. As atividades iniciais levantadas foram:

 \begin{itemize}
   \item \textbf{Atividade 1} Suporte a instalação automatizada do Wordpress.
   \item \textbf{Atividade 2} Suporte a instalação automatizada do Owncloud.
   \item \textbf{Atividade 3} Forçar as aplicações Wordpress e Owncloud a
   utilizar o protocolo https.
   \item \textbf{Atividade 4} Suporte a instalação automatizada do servidor de e-mail,
   utilizando os protocolos: IMAPS e SMTPS.
 \end{itemize}

Ao verificar que tanto a ferramenta owncloud como a ferramenta wordpress precisam
de configurações servidor de e-mail para algumas funcionalidades, também foi adicionado
a atividade de criação de uma receita para a automatizar a configuração de
servidor de e-mail. Como aspecto de segurança, ficou definido que tanto a aplicação
owncloud como wordpress deveriam usar sempre https por padrão, isso envolve também
uma estratégia de como serão gerenciados os certificados de segurança para aplicar
o protocolo https. Além dessas atividades previstas, também foram feitas evoluções
no Shak para que seja possível a implementação dessas atividades. Nas seções a
seguir serão apresentados os resultados das atividades levantadas, a partir da
proposta da construção da solução vista no capítudo \ref{cap-metodologia}.

Dentro da implantação automatizada de software Debian GNU/Linux temos várias
configurações possíveis, porém neste trabalho alguns aspectos que deverão ser levados em
consideração, esses aspectos serão importantes para guiar a construção da solução.

\begin{itemize}
  \item  \textbf{A1:} Uso dos aspectos de segurança na implantação de aplicações web.
  \item  \textbf{A2:} Uso da implantação automatizada de múltiplas instâncias de
   aplicações no mesmo servidor.
  \item  \textbf{A4:} Uso de aplicações que são empacotadas no debian.
  \item  \textbf{A5:} Uso de boas práticas de implantação de software.
\end{itemize}

Por fim é necessário definir procedimentos que serão
feitos para que aconteça a implantação automatizada das aplicações, esses procedimentos são
importantes para a construção da arquitetura inicial considerando
o processo de implantação de software visto no capítulo \ref{cap-introducao}
e a referência dos trabalhos relacionados. Os procedimentos são os seguintes:

\begin{itemize}
  \item  \textbf{Definir as fases e os procedimentos para implantação automatizada:}
   Definir quais são etapas da implantação, definir a ordem necessária para a execução de
  cada fase da implantação, dado a importância de definir as fases que compõem o processo de
  implantação e de acordo com \cite{omg2006}.
  \item  \textbf{Definir os procedimentos de segurança na implantação:} Definir
  quais são os procedimentos de segurança e automatizar os que forem possíveis
  de serem aplicados dentro do contexto da arquitetura proposta. Um exemplo
  é a configuração das aplicações sempre usarem protocolos seguros, como o https
  que possui uma camada adicional de segurança que utilizando o protocolo ssl/tls.
\end{itemize}

\subsection{Fases e Procedimentos para implantação}

Primeiramente é necessário definir as fases e os procedimentos para a implantação automatizada,
seguindo as fases que compõem o processo de implantação de aplicações e de acordo
com \cite{omg2006}, e isso será o que o Shak automatizará fases e procedimentos são:

\begin{itemize}
  \item  \textbf{Planejamento:} O planejamento da implantação é uma fase para
  identificar os componentes necessários na implantação da aplicação. Definir
  quais são as dependências mínimas
  para o funcionamento de uma aplicação, tais como: banco de dados, pacotes
  pré-instalados e aplicações pré-configuradas
  \item  \textbf{Preparação e Instalação de Pacotes:} São os procedimentos necessários
   para preparar o ambiente alvo para que a aplicação
  possa ser executada, isso envolve configuração do sistema operacional, instalação
  e configuração de dependências necessárias, e a transferência do componente
  para o servidor onde ele será executado.
  \item  \textbf{Configuração:} Como levantado no planejamento é necessário a
  edição de arquivos de configuração
  de cada aplicação, arquivos de configuração do banco de dados e arquivo de configuração
  do servidor web.
  \item  \textbf{Configuração de múltiplas instâncias:} Habilitar a configuração
  de múltiplas instâncias de uma aplicação, permitindo que seja possível que
  tenha várias instâncias da mesma aplição sem duplicação de recursos, ou seja,
  utilizando os mesmos recursos de um único servidor.
  \item  \textbf{Inicialização/Execução:} Executar a instalação da aplicação pela
  ferramenta Shak e validar os resultados obtidos.
\end{itemize}

\section{Validação da solução}
\label{subsection:validacao}

Para validar a arquitetura proposta no trabalho serão feitos exemplos de uso
com aplicações que possam servir para a execução da arquitetura construída, ou seja,
aplicações que tenham todo o seu processo de instalação e configuração automatizado, a
fim de refinar e evoluir a solução conforme problemas forem surgindo, a escolha
desses exemplos de uso devem ser feitas a partir de aplicações reais e
conhecidas na comunidade de software livre. Outros aspectos também são importantes e
devem ser levados em consideração para a escolha das
aplicações piloto, as seguintes características foram levadas em consideração
para a escolha das aplicações:

\begin{itemize}
  \item  \textbf{Aplicações empacotadas no debian:} Como o intuito do trabalho
  é realizar implantações múltiplas a partir de um pacote único, tais aplicações
  devem estar empacotadas e disponíveis para instalação nos servidores do debian.
  Isso impacta na escolha da ferramenta, visto que não será necessário ter o trabalho
  de empacotar aplicações que ainda não estão empacotadas no debian.
  \item  \textbf{Servidor web compatível:} As ferramentas escolhidas devem no
  mínimo possuir o servidor web compatível, por exemplo: as aplicações juntas
  devem possuir suporte para nginx ou apache ou similares.
  \item  \textbf{Aplicações com comunidades ativas:} Como estamos trabalhando
  com software livre, é importante que os softwares escolhidos possuem comunidades
  ativas, isso pode ajudar na resolução de possíveis problemas, logo aplicações
  em que sejam difíceis de comunicar com sua comunidade devem ser evitadas, e
  aplicações com a comunidade de desenvolvedores e usuários ativa devem ser priorizadas.
  E isso também pode ser um fator importante caso durante os testes sejam descobertos
  bugs ou melhorias dentro dessas ferramentas e tais bugs e melhorias possam ser
  reportados para a comunidade, ou até mesmo bugs solucionados e devolvidos aos mantenedores
  das ferramentas.
  \item  \textbf{Documentação do software:} A documentação do software também deve
  ser levado em consideração, principalmente a documentação da instalação e configuração
  dentro da ferramenta, ferramentas que não possuem documentação de instalação e
  configuração devem ser evitadas.
  \item  \textbf{Aplicações com suporte a federação:} %TODO
\end{itemize}

A partir dessas características definidas, é necessário encontrar os exemplos de uso
que possam servir para a execução da solução, todos as aplicações que servirão como
exemplos de uso devem possuir pacotes incluídos na distribuição oficial do Debian.

\subsection{Exemplos de uso}
\label{subsection:exemplos}

Para encontrar as aplicações que possam se encaixar dentro desses parâmetros
devemos buscar por alguns exemplos de uso, para a escolha das aplicações que
serão utilizadas como exemplos de uso é necessário fazer uma busca nas aplicações
web que são empacotadas no debian e que possuem suporte a configuração de múltiplas
instâncias, essa busca deve levar em consideração também a documentação para realizar
tal configuração. Foram levantados algumas aplicações web empacotadas no debian da
seguinte forma:

\begin{center}
apt-cache search web | wc -l
\end{center}

O resultado obtido com pacotes que contenham a palavra web recebe o resultado de 3470
pacotes de diversas aplicações ou módulos de aplicações, como:
wordpress, owncloud, drupal, mailman, chromium, etc. Logo dentro dessa rápida busca
encontramos alguns pacotes de aplicações conhecidas, para facilitar a busca basta
aplicar para alguns nomes de aplicações conhecidos como:

\begin{center}
apt-cache search web | grep wordpress
\end{center}

Para avaliar se as aplicações possuem suporte a múltiplas instâncias é necessário
analisar a documentação das aplicações, as aplicações costumam disponibilizar a sua
documentação na sua página ou em uma wiki, também é possível checar a documentação
quando se instala uma aplicação nos arquivos de documentação da aplicação em
/usr/share/doc/ ou também pode-se utilizar os comandos:

\begin{center}
man nomeaplicação

info nomeaplicação
\end{center}

Analisando a documentação de algumas aplicações como wordpress, redmine,
owncloud, mailman, wikimédia, foram escolhidas as aplicações wordpress e owncloud por
comtemplarem as características definidas anteriormente. Primeiro
foi feito uma análise da documentação dessas ferramentas para encontrar a
possibilidade da configuração de múltiplas instâncias no mesmo servidor, além
disso também foi visto se as aplicações tinham suporte ao servidor Nginx que é
o servidor utilizado pelo Shak, por fim as duas aplicações também foram escolhidas
pela grande comunidade de usuários ativa e as duas aplicações também possuírem
canais de IRC para tirar dúvidas com desenvolvedores e usuários, também foi levado
em consideração o fato da  aplicação Owncloud dar suporte a federação,
com  o objetivo de compartilhar arquivos em servidores na nuvem de forma simples,
como por exemplo: enviar um e-mail de um servidor para outro, sendo assim,
também é possível migrar os seus arquivos pessoais de um servidor Owncloud a outro.

Com as aplicações que serão exemplos de uso escolhidas a proposta de arquitetura
pode ser testada e validada, com o objetivo de criar múltiplas instâncias
no mesmo servidor de forma automatizada a partir de um pacote único, logo os
exemplos de uso também servirão para refinar e evoluir a arquitetura proposta.
Para realizar tais validações é importante que as instalações e configurações
sejam em um ambiente limpo, ou seja, os testes devem ser criados de preferência
em máquinas virtuais com a configuração conhecida como minimal, que contém
instalado apenas as aplicações necessárias para o funcionamento do sistema operacional.

A partir da execução dos exemplos de uso será possível a coleta de informações
necessárias para a validação da implantação correta das aplicações, essa
validação a princípio será feita a partir da execução da implantação e
verificação das funcionalidades da aplicação implantada, ou seja,
após a execução da implantação o testador deverá verificar o perfeito
funcionamento de algumas funcionalidades básicas da aplicação escolhida,
e principalmente verificar a implantação de várias instâncias da mesma
aplicação no mesmo servidor destino, observando o perfeito funcionamento de todas as
instâncias implantadas assim validando a implantação de múltiplas instâncias.
