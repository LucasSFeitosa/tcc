\chapter{Metodologia}
\label{cap-metodologia}
Este capítulo aborda sobre tudo que compõe a metodologia de pesquisa,
procedimentos e técnicas utilizados no trabalho, a fim de
explicar como o desenvolvimento do trabalho foi realizado para atingir os seus objetivos,
servindo como base para a sua reprodução em trabalhos futuros.

A pesquisa é fundamentada e metodologicamente construída objetivando a resolução
ou o esclarecimento de um problema. O problema é o ponto de partida da pesquisa.
Da sua formulação dependerá o desenvolvimento da sua pesquisa
\cite{moresi2003metodologia}, neste trabalho a questão problema a ser resolvida é:

\begin{center}
  \textit{
  Como implantar múltiplas aplicações a partir de um pacote único
  de forma centralizada e segura?
}
\end{center}

Dado a questão problema, e o objetivo que é implantação automatizada
de aplicações em sistema Debian GNU/Linux, este trabalho consiste em contribuir
com a construção de uma solução de engenharia de software que resolva a questão
problema, essa construção será feita a partir dos conhecimentos adquiridos durante
o curso de engenharia de software, além disso a solução proposta deverá passar
por uma validação dos resultados, utilizando uma validação baseada em exemplos
de uso, onde cada exemplo de uso terá um cenário diferente para testar a solução
proposta. Essa validação tem seus procedimentos definidos na seção\ref{section:validacao} e
e as definições dos exemplo de uso estão na seção\ref{section:exemplos}.

Inicialmente será feito uma pesquisa nos trabalhos relacionados para entender
melhor o problema, e para guiar essa pesquisa será utilizado a definição de uma
metodologia de pesquisa. De acordo com\cite{gerhardt2009metodos} existem diferentes
tipos de pesquisa, elas podem ser classificados quanto sua abordagem, sua natureza,
seus objetivos e seus procedimentos. Por isso é importante selecionar a modalidade
de pesquisa adequada ao objeto de pesquisa.

Assim a pesquisa inicial será classificada quanto à natureza como pesquisa aplicada
pois tem como objetivo gerar conhecimentos para aplicação prática\cite{gerhardt2009metodos},
relacionada a aplicações que auxiliam na implantação automatizada de software.
Quanto a abordagem, a pesquisa é classificada como qualitativa, para buscar entender
o uso da automação da implantação de aplicações, exprimindo o que convém
a ser feito.

Quanto aos objetivos, esta pesquisa é classificada como pesquisa descritiva, que
exige do investigador uma série de informações sobre o que deseja pesquisar
\cite{trivinos1987introduccao}. Quanto aos procedimentos técnicos, serão utilizados
neste trabalho a pesquisa bibliográfica baseada no levantamento de referências
teóricas em livros e artigos científicos, permitindo conhecer os trabalhos
relacionados. A figura\ref{fig:metodologia1} apresenta de forma resumida a classficação
da pesquisa:

\begin{figure}[h]
  \centering
  \includegraphics[width=1.0\textwidth]
      {figuras/met1.eps}
  \caption{Resumo da classificação de pesquisa}
\label{fig:metodologia1}
\end{figure}

% agora aqui falar do desenvolvimento do software em si, como vou fazer o desenvolvimento
% depois eu vou dizer o que eu escolhi pra "executar" o desenvolvimento
% ja tem uma seção de validação, trabalhos relacionados, Exemplos de uso, construcao
% na validação entra a observação simples e não aqui
% colocar uma outra figura que represente tudo
% aqui ja morreu pesquisa bibliográfica
%
A partir da metodologia definida é possível encontrar os trabalhos relacionados
pela pesquisa científica, os trabalhos encontrados serão de insumo para iniciar
a construção da solução proposta a partir da análise dos resultados encontrados,
as próximas seções tratam dos trabalhos relacionados encontrados e dos métodos e
procedimentos para construção da solução.

\section{Trabalhos Relacionados}
\label{section:trabalhos_relacionados}
Para responder a questão problema primeiramente deve ser feito um levantamento
de trabalhos relacionados, para entender melhor como a automação da implantação
de software está sendo utilizada como solução na academia e nas empresas, e
quais são os procedimentos e técnicas utilizados, e se os aspectos de segurança
são levados em consideração.

Para entender o que acontece na comunidade da computação em relação a implantação
automatizada de aplicações, foi feito uma busca de alguns trabalhos relacionados, o primeiro
trabalho relacionado é um trabalho acadêmico feito por\cite{leo2014} no qual foi
desenvolvido um sistema de middleware chamado CHOReOS Enactment Engine que é um
sistema que possibilita a implantação distribuída e automatizada de composições
de serviços web em uma estrutura virtualizada, no qual opera no modelo
computacional conhecido como plataforma como serviço, comparando esse sistema
com abordagens ad-hoc de implantação levando em consideração a escalabilidade
em relação ao tempo de implantação das composições dos serviços.

Existem também ferramentas disponíveis no mercado que também trazem a proposta
de automação de instalação de aplicações alguns exemplos são:\cite{bitnami}
BitNami que é uma biblioteca de aplicativos de servidor populares e ambientes de
desenvolvimento que pode ser instalado com um clique. Ele automatiza todo o
processo de compilar e configurar os aplicativos e todas as suas dependências
(bibliotecas de terceiros, linguagem de programação, bases de dados) para que o
usuário comum não se preocupe com questões mais técnicas. Também temos o\cite{sandstormio} Sandstorm é uma plataforma de código aberto para servidores
pessoais. Sandstorm permite que você execute o seu próprio servidor e instalar
facilmente as aplicações em que dão suporte, e por fim a ferramenta\cite{juju}
Juju que lhe permite implementar,configurar, gerenciar, manter e serviços de forma
rápida e eficiente automatizando a instalação e configuração de aplicações na nuvem.

O estudo dessas aplicações serve como base para entender como são feitas as
arquiteturas que são utilizadas pela comunidade da computação, por exemplo o JuJu
utiliza uma abstração conhecida como charm, que são o códigos que define um serviço,
um charm contém toda a lógica de que você precisa para implementar e integrar uma
aplicação, contendo todo o processo de download de ferramentas, instalação e configuração
de aplicações, podendo ser escrito
em várias linguagens ou até provisionadores como chef, puppet ou docker\cite{juju}, é
possível reaproveitar os charms feitos pela comunidade do JuJu assim reaproveitando
os charms prontos ou não precisando escrever um charm do zero.

Já bitnami ja tráz as aplicações
prontas pra uso, assim o usuário interage no formato clique para instalar, porém o usuário
só tem a disposição as aplicações disponibilizadas pelo bitnami, isso também é
a forma em que o sandstorm.io trabalha, com a diferença de que o standstorm é
software livre. Já o trabalho feito por\cite{leo2014} é voltado para serviços web
de grande escala, com foco em implantação de aplicações na nuvem e escalar infraestrutura, utilizando o middleware
construido em seu trabalho, assim podendo gerenciar ambientes de computação em nuvem
como Amazon EC2 e Openstack, sua engine utiliza o chef solo como seu agente de configuração.

Com esse levantamento vemos que existem algumas soluções  que seguem a mesma
linha do que este trabalho pretende atingir como objetivo, servindo como motivação
o desenvolvimento do trabalho. Assim visto que a utilização do empacotamento
como técnica de implantação, traz inúmeras vantagens no processo de implantação
de aplicações como citado no capítulo\ref{cap-introducao}, tendo o levantamento
dos trabalhos relacionados e a questão de pesquisa como insumo, justifica-se a
construção de uma solução de engenharia de software para implantação automatizada
de aplicações  em sistemas Debian GNU/Linux.

\section{Construção da solução}
\label{section:construcao}

A engenharia de software é a área da computação que busca evoluir de forma contínua
todos os aspectos da produção de um software, procurando sempre a garantia da
qualidade do produto desenvolvido. Segundo\cite{pressman2011engenharia}, a engenharia
de software é  composta por um conjunto de três elementos fundamentais:

\begin{itemize}
  \item \textbf{Métodos:} Como fazer a implementação do software, com modelos,
  especificações, critérios para qualidade.
  \item \textbf{Ferramentas:} Objetivo de apoio automatizado para auxiliar as atividades
  de engenharia de software.
  \item \textbf{Processos:} Definem a sequência de práticas que serão utilizadas no
  desenvolvimento.
\end{itemize}

Seguindo esses três elementos, é possível definir como será construido a solução.
Primeiramente, é necessário definir o método de desenvolvimento de software, o
método escolhido para desenvolvimento de software é o método ágil chamado programação
extrema (XP), de acordo com\cite{796139} essa metodologia contém um conjunto de
práticas que auxiliam um engenheiro de software. Dentro das práticas que estão
em\cite{796139} foram escolhidas as que poderiam ser aplicadas no contexto deste
 trabalho, e são elas:

\begin{itemize}
  \item \textbf{Pequenas versões:} Implementar pequenas versões para conseguir
  um feedback mais rápido, atualizando o software frequentemente.
lançamentos serão feitos a partir
  \item \textbf{Design simples:} O código desenvolvido deve ser o mais simples possível,
  facilitando o entendimento de outros desenvolvedores que venham a ler o código.
  Além disso o código desenvolvido não deve fazer mais do que foi definido como
  requisito de software.
  \item \textbf{Refatoração:} O sistema pode ser resstruturado e seu comportamento
  continuar o mesmo, é muito utilizado para remover duplicações de código, deixar
  o código mais simples e fácil de dar manutenção.
  \item \textbf{Programação em pares:} A produção do software deve ser feita por duas
  pessoas utilizando a mesma máquina, no contexto deste trabalho a programação
  em pares será feita com o orientando pareando com o orientador ou co-orientador,
  sempre que possível.
  \item \textbf{Revisão de código:} Todo o código produzido deverá ser revisado
  por um progamador mais experiente, para que assim evite que erros sejam
  adicionados no código, no contexto deste trabalho a revisão de código será
  feita pelo co-orientador.
  \item \textbf{Jogo de planejameto:} A equipe de desenvolvimento deve decidir
  as atividades que serão desenvolvidas baseada em uma estimativa de duração,
  prioridade das atividades e compelxidade, e também se tais atividades cabem
  no período do ciclo de desenvolvimento.
\end{itemize}

O ciclo de desenvolvimento terá iterações com duração de uma semana, ou seja,
o jogo do planejamento das atividades deverá levar em consideração que a
estimativa de todas as atividades escolhidas para a a iteração não poderá
ultrapassar a duração de uma semana, dentro dessa semana além do desenvolvimento
das funcionalidaes também deve levar em consideração a próprioa atividade de jogo
de planejamento, a programação em pares e a revisão de cóido. A revisão de código
pode acarretar em pequenas correções que serão necessárias antes do código novo
ser incorporado, e isso pode levar algum tempo para ser feito, por isso a importância
de não deixar a revisão de cóidigo para o último dia da interação.

%imagem

As ferramentas de apoio ao desenvolvilmento de software são importantes para a
organização do trabalho, auxiliando nas atividades típicas dentro de um projeto
de engenharia de software. Algumas ferramentas são importantes pois tornan mais fácil
a execução de algumas atividades, como por exemplo: o controle de versão do software,
ferramenta para documentação do software e ferramenta para gerenciar as tarefas.
Os critérios para escolha das ferramentas são:

\begin{itemize}
  \item \textbf{Forge para sistema de controle de versão git:} Uma ferramenta que
  possa gerenciar diversas versões do desenvolvimento do código fonte utilizando
  o sistema de controle de versão git, que seja de preferência um software livre
  ou que não cobre o serviço de hospedagem. Algumas ferramentas ja são
  conhecidas como github, gitlab e bitbucket.
  \item \textbf{Ferramenta para documentação de código:} Uma ferramenta que possa
  documentar o software, os forges como gitlab e github possuem uma wiki já disponível
  para o projeto.
  \item \textbf{Ferramenta de gerenciamento de tarefas:} Ferramenta para gerenciamento
  das tarefas que serão executadas, que estão em execução ou que vão ser executadas
  durante o desenvolvimento do projeto. Essa ferramenta será importante para a
  organização do trabalho, dando visibilidade ao trabalho que está sendo realizado.
  \item \textbf{Ferramenta de comunicação:} A ferramenta de comunicação será
  importante para tirar dúvidas rápidas em relação a dificuldades e desafios, as
  comunidades de software livre costumam usar listas de emails e canais no IRC
  para a comunicação de seus desenvolvedores e usuários.
\end{itemize}

%colocar imagem

Por fim é necessário definir a  sequência de práticas  e atividades que serão
utilizadas no desenvolvimento, já que partir do referencial teórico é possível
definir uma proposta de arquitetura básica para a solução de implantação automatizada
de aplicações. A primeira prática é a definição da arquitetura inicial, nela é importante levar em consideração os
seguintes aspectos:

\begin{itemize}
  \item  \textbf{A1:} Aspectos de segurança na implantação de aplicações web.
  \item  \textbf{A2:} Aspectos para implantação automatizada de múltiplas instâncias de
   aplicações no mesmo servidor.
  \item  \textbf{A3:} Possibilidade de lidar com aplicações desenvolvidas em
  diferentes linguagens.
  \item  \textbf{A4:} Aplicações que são empacotadas no debian.
  \item  \textbf{A5:} Boas práticas de implantação de software.
\end{itemize}

Esses aspectos são importantes para definir o que vai compor a arquitetura inicial,
para que se possa validar a proposta com os exemplos de uso, podendo assim evoluir
a arquitetura proposta. Além disso também é preciso definir alguns procedimentos
importantes para a construção da arquitetura inicial considerando o processo de
implantação de software visto no capítulo\ref{cap-introducao} e a referência dos
trabalhos relacionados. Os procedimentos são os seguintes:

\begin{itemize}
  \item  \textbf{Definir tecnologia para automatizar implantação:}  Definir qual será a
  tecnologia escolhida dentre as citadas no referencial teórico para dar suporte
  a implantação.
  \item  \textbf{Definir conjunto mínimo de dependências:} Definir quais são as dependências
  mínimas para o funcionamento de uma aplicação independente das aplicações escolhidas para
  servir de exemplos de uso, tais como: sistema operacional, pacotes pré-instalados
  e aplicações pré-configuradas, como por exemplo: servidor web e banco de dados.
  \item  \textbf{Definir as fases e os procedimentos para implantação automatizada:}
   Definir quais são etapas da implantação, definir a ordem necessária para a execução de
  cada fase da implantação, dado a importância de definir as fases que compõem o processo de
  implantação e de acordo com\cite{omg2006}.
  \item  \textbf{Definir os aspectos de segurança} Definir quais são os aspectos de segurança
  e automatizar os que forem possíveis de serem aplicados dentro do contexto da arquitetura
  proposta.
\end{itemize}

Com essa base definida o próximo passo é definir um ambiente de desenvolvimento
e um ambiente de testes, já que dado o contexto de implantação de software,
é importante possuir um ambiente flexível para testar facilmente a instalação
e configuração das aplicações, podendo facilmente reinicializar esse ambiente de
forma que ele fique limpo sem resquícios da instalação anterior. Com a base definida
e com um ambiente de desenvolvimento e de testes configurados, é necessário definir
quais são as aplicações que servirão como caso de testes para a implementação da
solução. Elas serão importantes pois a partir delas e de suas necessidades que será
feito a validação da proposta e seus refinamentos. Após a escolha das aplicações
que servirão como exemplos de uso e os ambientes definidos pode-se iniciar o
ciclo de desenvolvimento de software para a construção da solução.

%imagem aqui
\subsection{Exemplos de uso}
\label{section:validacao}

Para validar a arquitetura proposta no trabalho serão feitos exemplos de uso
com aplicações que possam servir para a execução da arquitetura construída, a
fim de refinar e evoluir a solução, a escolha desses exemplos de uso devem ser
feitas a partir de aplicações reais e conhecidas na comunidade de software livre.
 Para isso devem ser levados as seguintes características para a escolha das aplicações:

\begin{itemize}
  \item  \textbf{Aplicações empacotadas no debian:}  Como o intuito do trabalho
  é realizar implantações múltiplas a partir de um pacote único, tais aplicações
  devem estar empacotadas e disponíveis para instalação nos servidores do debian.
  Isso impacta na escolha da ferramenta, visto que não será necessário ter o trabalho
  de empacotar aplicações que ainda não estão empacotadas no debian.
  \item  \textbf{Servidor web compatível:} As ferramentas escolhidas devem no
  mínimo possuir o servidor web compatível, por exemplo: as aplicações juntas
  devem possuir suporte para nginx ou apache ou similares.
  \item  \textbf{Aplicações com comunidades ativas:} Como estamos trabalhando
  com software livre, é importante que os softwares escolhidos possuem comunidades
  ativas, isso pode ajudar na resolução de  possíveis problemas, logo aplicações
  em que sejam difíceis de comunicar com sua comunidade devem ser evitadas, e
  aplicações com a comunidade de desenvolvedores e usuários ativa devem ser priorizadas.
  E isso também pode ser um fator importante caso durante os testes sejam descobertos
  bugs ou melhorias dentro dessas ferramentas e tais bugs e melhorias possam ser
  reportados para a comunidade, ou até mesmo solucionados e devolvidos aos mantenedores
  das ferramentas.
  \item  \textbf{Documentação do software:} A documentação do software também deve
  ser levado em consideração, principalmente a documentação da instalação e configuração
  dentro da ferramenta, ferramentas que não possuem documentação de instalação e
  configuração devem ser evitadas.
\end{itemize}

Para encontrar as aplicações que possam se encaixar dentro desses parâmetros devemos
alguns exemplos de uso, com a finalidade de verificar se com tais métodos e
procedimentos atingirá o objetivo estabelecido. Para a escolha das aplicações que
serão utilizadas como exemplos de uso é necessário fazer uma busca nas aplicações
web que são empacotadas no debian e que possuem suporte a configuração de múltiplas
instâncias, essa busca deve levar em consideração também a documentação para realizar
tal configuração. Foram levantados algumas aplicações web empacotadas no debian da
seguinte forma:

\begin{center}
apt-cache search web | wc -l
\end{center}

O resultado obtido com pacotes que contenham a palavra web recebe o resultado de 3470
pacotes de diversas aplicações ou módulos de aplicações, como:
wordpress, owncloud, drupal, mailman, chromium, etc. Logo dentro dessa rápida busca
encontramos alguns pacotes de aplicações conhecidas, para facilitar a busca basta
aplicar para alguns nomes de aplicações conhecidos como:

\begin{center}
apt-cache search web | grep wordpress
\end{center}

Para avaliar se as aplicações possuem suporte a múltiplas instâncias é necessário
analisar a documentação das aplicações, as aplicações costumam disponibilizar a sua
documentação na sua página ou em uma wiki, também é possível checar a documentação
quando se instala uma aplicação nos arquivos de documentação da aplicação em
/usr/share/doc/ ou também pode-se utilizar os comandos:

\begin{center}
man nomeaplicação

info nomeaplicação
\end{center}

Com as aplicações que serão exemplos de uso escolhidas a proposta de arquitetura
pode ser testada e validada, com o objetivo de criar múltiplas instâncias
no mesmo servidor de forma automatizada a partir de um pacote único, logo os
exemplos de uso também servirão para refinar e evoluir a arquitetura proposta.
Para realizar tais validações é importante que as instalações e configurações
sejam em um ambiente limpo, ou seja, os testes devem ser criados de preferência
em máquinas virtuais com a configuração conhecida como minimal, que contém
instalado apenas as aplicações necessárias para o funcionamento do sistema operacional.

A partir da execução dos exemplos de uso será possível a coleta de informações
necessárias para a validação da solução proposta, essa validação a princípio
será feita a partir da execução da implantação e verificação das funcionalidades
da aplicação implantada, ou seja, após a execução da implantação o testador deverá
verificar o perfeito funcionamento de algumas funcionalidades básicas da aplicação
escolhida, e principalmente verificar a implantação de várias instâncias da mesma
aplicação no mesmo servidor destino, observando o perfeito funcionamento de todas as
instâncias implantadas assim validando a implantação de múltiplas instâncias.
