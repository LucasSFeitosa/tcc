\chapter{Definição e Preparação dos Estudos}

\label{cap-metodologia}
Este capítulo aborda o planejamento para a execução do projeto,
contendo os procedimentos e técnicas utilizados, a fim de
explicar como o desenvolvimento foi realizado para atingir os seus objetivos,
servindo como base para a sua reprodução em trabalhos futuros.

O trabalho é fundamentado na proposta de uma solução, utilizando conceitos e práticas
da engenharia de software, com a definição de um objetivo, a definição de um planejamento,
e a execução das atividades planejadas, e posteriormente a validação dos resultados. 
A principal contribuição deste trabalho está em ajudar a responder a seguinte questão:

\begin{center}
 \textbf{QP:}
  \textit{
  Como implantar aplicações web em sistema Debian GNU/Linux de forma automatizada e
  e segura?
}
\end{center}

Dado objetivo e a questão problema, este trabalho consiste em contribuir
com a proposta de uma solução de gerência de configuração de software, que permita implantar
aplicações web em sistemas Debian GNU/Linux de forma automatizada e segura.

Além disso, a solução proposta deverá passar
por uma validação dos resultados, baseada na observação
da execução da solução em exemplos de uso, onde cada exemplo conterá uma aplicação
real que deverá ser implantada com sucesso. Essa validação tem seus procedimentos
definidos na Seção \ref{subsection:validacao} e as definições dos exemplos de uso
estão na Seção \ref{subsection:exemplos}.

Inicialmente, foi realizado uma pesquisa dos trabalhos relacionados, para compreender
quais eram as soluções na engenharia de software que buscam a automação da implantação
de aplicações web. Esta pesquisa foi uma pesquisa aplicada, pois tem como objetivo
gerar conhecimentos para aplicação prática \cite{gerhardt2009metodos},
relacionada a aplicações que auxiliam na implantação automatizada de software.

Os trabalhos encontrados serviram de insumos para auxiliar na construção do trabalho, 
a partir da análise dos resultados que foram encontrados. 

\section{Trabalhos Relacionados}
\label{section:trabalhos_relacionados}

Para compreender o que acontece na área relacionada com a implantação
automatizada de aplicações, foi feito uma busca de alguns trabalhos relacionados 
recentes. \citeonline{leo2014} em seu mestrado,
desenvolveu um sistema de middleware chamado CHOReOS Enactment Engine, que possibilita a implantação distribuída e automatizada de composições
de serviços web numa estrutura virtualizada, no qual opera no modelo
computacional conhecido como plataforma como serviço. Seu foco é na automação da 
implantação automatizada de aplicações aliado com a gerência de recursos de 
hardware.

Existem ferramentas disponíveis no mercado que também trazem a proposta
de automação de instalação de aplicações, BitNami \footnote{https://bitnami.com/} é uma 
uma biblioteca de aplicativos populares e ambientes de desenvolvimento, 
que podem ser instalados com apenas um clique, 
através de uma interface web amigável. Ela automatiza todo o processo de 
compilar e configurar os aplicativos, 
e todas as suas dependências (bibliotecas de terceiros, linguagem de programação, 
bases de dados) para que o usuário comum não se preocupe com questões técnicas \citeonline{bitnami}. 

Sandstom.io \footnote{https://sandstorm.io/} é uma plataforma de código aberto para servidores
pessoais. Sandstorm permite instalar
facilmente as aplicações em que o Sandstorm 
suporta \citeonline{sandstormio}. Sandstorm trás aplicações prontas para uso, 
em que o usuário escolhe a aplicação desejada  por meio de uma interface web, 
e logo após isso, a aplicação já está disponível para uso.

JuJu \footnote{https://jujucharms.com/} é uma ferramenta que permite 
implementar, configurar, gerenciar, 
e manter serviços de forma rápida e eficiente, automatizando a instalação e 
configuração de aplicações na nuvem \cite{juju}. É 
é uma ferramenta mantida pela Canonical \footnote{http://www.canonical.com/}. 

O estudo dessas aplicações serve como base para entender como funcionam as aplicações
já existentes, e como elas são feitas, para que assim seja possível identificar
as ferramentas que são utilizadas para solucionar problemas semelhantes ao problema deste
trabalho. 

A ferramenta \citeonline{juju} utiliza uma abstração conhecida como charm,
que é basicamente um arquivo com trechos de código. Um charm
contém toda a lógica de que é preciso para implementar e integrar uma aplicação,
contendo todo o processo de download de ferramentas, instalação e configuração de
aplicações. É possível utilizar provisionadores como Chef, Puppet ou Docker. Também 
é possível reaproveitar os charms feitos pela comunidade do JuJu. Assim,
reaproveitando os charms prontos ou não precisando escrever um charm do zero.

Já \citeonline{bitnami} trás as aplicações prontas pra uso. Assim, o usuário interage com 
uma interface web, no formato clique para instalar. Porém,
o usuário só tem a disposição as aplicações disponibilizadas pelo Bitnami. Isso
também é a forma em que o Sandstorm trabalha, com a diferença de que o Sandstorm é
software livre, já Bitnami tem o código fechado, o que dificulta a análise da arquitetura
que é utilizada por eles. 

Por último, o trabalho feito por \citeonline{leo2014} é voltado
para serviços web de grande escala, com foco em implantação de aplicações na
nuvem. Também é possível escalar infraestrutura, utilizando o middleware construído 
em seu trabalho, podendo gerenciar ambientes de computação em nuvem como Amazon EC2 e 
Openstack. A sua aplicação utiliza o Chef como seu agente de configuração.

Com esse levantamento de trabalhos recentes, as soluções encontradas não resolvem totalmente
o problema. A grande diferença é que as ferramentas encontradas não utilizam os pacotes
disponibilizados pelo Debian. De toda forma, as ferramentas encontradas buscam
abstrair todo processo de instalação e configuração das aplicações, para que o
usuário não tenha dificuldades na instalação de aplicações. As ferramentas também
trazem uma interface web para que o usuário não precise executar tarefas via terminal.

Em resumo, por um lado, as ferramentas encontradas também utilizam do conceito de
infraestrutura como código, como visto no Capítulo \ref{cap-referencial}. Para
isso, utilizam os provisionadores Chef ou Puppet, tornando mais simples a
construção de códigos para a instalação e configuração de aplicações. Por outro lado, 
observa-se que o problema tratado neste trabalho não está totalmente resolvido. Assim, 
a proposta de solução de implantação automatizada de software em sistemas debian, é válida
, já que se diferencia das demais ferramentas encontradas.

\section{Proposta da solução}
\label{section:construcao}

Para a proposta da solução, foram definidas as ferramentas
de apoio ao desenvolvimento da solução. Posteriormente, a escolha da ferramenta
que foi utilizada para solução do problema.

\subsection{Ferramentas de apoio}

As ferramentas de apoio ao desenvolvimento de software são importantes para a
organização do trabalho, auxiliando nas atividades típicas dentro de um projeto
de engenharia de software. Algumas ferramentas são importantes, pois tornam mais fácil
a execução de algumas atividades. Os critérios para escolha das ferramentas são:

\begin{itemize}
  \item \textbf{ Sistema de controle de versão git:} Uma ferramenta que
  possa gerenciar diversas versões do desenvolvimento do código fonte, utilizando
  o sistema de controle de versão git, e que seja de preferência um software livre
  ou que não cobre o serviço de hospedagem. A ferramenta escolhida foi o Gitlab 
  \footnote{http://gitlab.com/}, por ser um software livre.
  \item \textbf{Ferramenta para documentação do projeto:} Uma ferramenta que possa
  documentar o projeto, o gitlab já possui uma wiki disponível para documentar 
  cada projeto.
  \item \textbf{Ferramenta de gerenciamento de tarefas:} Uma ferramenta que possa
  gerenciar as tarefas que serão executadas, que estão em execução ou que vão ser executadas
  durante o desenvolvimento do projeto. O gerenciamento de tarefas também pode ser
  feito no gitlab, onde é possível criar atividades para serem feitas, e criar marcos
  com datas de início e fim das atividades.
  \item \textbf{Ferramenta de comunicação:} A ferramenta de comunicação será
  importante para tirar dúvidas rápidas em relação a dificuldades e desafios do trabalho, 
  as
  comunidades de software livre costumam usar listas de e-mail e canais no \textit{IRC}
  para a comunicação de seus desenvolvedores. Logo, toda comunicação, é feita
  pelos canais de \textit{IRC} das ferramentas e suas respectivas listas de e-mail.
\end{itemize}

\subsection{Ferramenta Shak}

A solução proposta neste trabalho será baseada na evolução da ferramenta Shak
(Self Hosting Applications Kit). Ela tem o objetivo de facilitar 
ao máximo que usuários sem conhecimento técnico possam ter os seus próprios 
serviços de internet, garantindo a sua privacidade e segurança. 

Durante o Google Summer of Code 2015 \footnote{https://www.google-melange.com/gsoc/project/details/google/gsoc2015/thiagovsk/5757334940811264}, o autor deste trabalho realizou 
várias atividades relacionadas à evolução da ferramenta Shak. Tornando a ferramenta uma
 solução para implantação automatizada de aplicações web em sistemas Debian, 
justificando assim a sua escolha.

De acordo com \citeonline{shak2015}, esta plataforma está concebida de acordo 
com os seguintes princípios:

\textbf{Base: Pacotes Debian}

A plataforma Shak utiliza aplicações que são distribuídas oficialmente pelo 
sistema de pacotes do Debian. Essas aplicações utilizam os
pacotes do Debian, que fornecem atualizações consistentes de correção e de segurança,
e são utilizados por uma grande quantidade de usuários que reportam problemas a
serem resolvidos. 

\textbf{Nova abstração: Aplicação}

Para resolver as questões que não podem ser resolvidas a nível de pacotes, o
Shak introduz uma nova abstração: a aplicação. Uma aplicação geralmente
contém mais de um pacote, inclusive necessita de configurações em vários pacotes.

Uma vez que o usuário seleciona uma determinada aplicação para
ser instalada, os pacotes necessários são instalados e as configurações
necessárias são feitas de forma automática, fornecendo de fato um instalador
de um clique para aplicações suportadas. 

\textbf{Diferencial para outras soluções existentes}

Existem outras soluções disponíveis para instalação de aplicações por um clique,
como Bitnami e Sandstorm, mas o Shak se
diferencia delas nas seguintes características:

\begin{itemize}
  \item O Shak pode ser implantando numa máquina virtual na nuvem ou em servidores físicos
    próprios, a critério do usuário.

  \item O Shak reutiliza o trabalho dos mantenedores Debian, que fornecem pacotes
    de alta qualidade.

  \item O Shak reutiliza a infraestrutura do Debian, que é utilizada por várias
outras distribuições Linux baseadas no Debian, como por exemplo: Ubuntu e Linux Mint.

\end{itemize}

Por outro lado, apenas os softwares que estejam empacotados no Debian estarão
disponíveis, mas isso pode ser visto de forma positiva também. O Shak
fornecerá um incentivo para que ainda mais softwares estejam disponíveis no
Debian, o que é um benefício coletivo para o ecossistema do software livre, e
para os seus usuários. Logo, quanto mais aplicações disponíveis no Debian,
mais aplicações serão disponibilizadas para os usuários do Debian e seus derivados.

A arquitetura da ferramenta Shak é composta de:

\begin{itemize}
  \item  \textbf{Livro de Receitas Chef:} Shak contém livros de receitas chef
  para poder organizar a instalação de cada componente, contendo um livro
  de receitas para cada aplicação que for instalada.
  \item  \textbf{Código Ruby} A arquitetura do Shak foi desenvolvida na linguagem
  Ruby, com programação orientado a objetos.
  \item  \textbf{Servidor web} O Shak utiliza o software livre Nginx, que é
  servidor \textit{HTTP} de alto desempenho \cite{nginx}.
  \item  \textbf{Pacotes Debian} O Shak utiliza pacotes incluídos na distribuição
  oficial do Debian.
  \item  \textbf{Gems} O Shak também utiliza algumas gems para seu funcionamento,
  uma gem nada mais é do que uma biblioteca Ruby, que provê um formato padrão para
  a distribuição de programas Ruby \cite{gem}.
  \item  \textbf{Código Shellscript} O Shak também utiliza códigos ShellScript,
  principalmente para configurações de ambiente de desenvolvimento. 
\end{itemize}

Para adicionar uma aplicação que esteja disponível nos repositórios oficiais
do Debian na ferramenta Shak, é necessário construir o seu respectiovo 
livro de receitas, utilizando chef, para a
aplicação desejada. Essa atividade será uma das atividades deste trabalho, sendo assim
adicionando novas aplicações web a ferramenta Shak.

\subsection{Criação de Ambientes de Desenvolvimento}

Uma das necessidades do desenvolvimento do projeto é ter um ambiente de desenvolvimento
flexível, que possa ser rapidamente construído e destruído. Onde seja possível 
testar as implantações feitas pela ferramenta Shak e verificar os erros da implantação, caso 
aconteçam. 

Para isso, foi necessário utilizar alguma ferramenta que automatize processo de 
construir ambientes de desenvolvimento. Construir ambientes manualmente pode
ser um processo longo e demorado, além disso, é possível que ocorra erros por
desatenção ao executar vários passos manuais. A ferramenta escolhida serviu para
auxiliar na criação das máquinas virtuais, que eram os nós alvo 
das implantações.

A ferramenta que foi escolhida para auxiliar a criação de um ambiente de desenvolvimento é
a ferramenta Vagrant \footnote{https://www.vagrantup.com/}. Com o Vagrant é 
possível gerenciar a criação de máquinas
virtuais para os ambientes de desenvolvimento do projeto.

O ambiente de desenvolvimento escolhido, foi um ambiente com o sistema Debian na sua versão
sid 64 bits(que é a versão que contém os pacotes mais atuais, conhecida como versão instável),
apesar de a versão ser a versão instável, isso foi importante para que tenha disponível 
as versões mais novas dos pacotes das ferramentas escolhidas.

O repositório do Shak possui alguns scripts
para instalar algumas dependências. Nesses scripts contém as instalações das 
dependências para o Shak funcionar perfeitamente
e também o próprio Chef, que é utilizado pela ferramenta Shak para instalar aplicações. Esses
scripts podem são referenciados no vagrant, no arquivo Vagrantfile. 

\section{Planejamento das Atividades}

As atividades planejadas para a evolução da ferramenta Shak são baseadas nas 
aplicações que foram escolhidas na Seção
\ref{subsection:exemplos}. Essas aplicações foram as aplicações que tiveram
sua implantação automatizadas pela ferramenta. 

No trabalho, foi feito uma fase exploratória, no qual o planejamento foi
suportar a instalação do Owncloud e Wordpress de forma 
automatizada. A escolha das duas ferramentas foi feita pelo autor e pelo co-orientador
deste trabalho, levando em consideração a popularidade das duas ferramentas nas 
comunidades de software livre. 

Essa fase exploratória foi necessária para evoluir a ferramenta
, visto que a ferramenta Shak só suportava aplicações estáticas, ou seja, apenas 
aplicações que contenham 
código \textbf{HTML}, \textbf{CSS} e \textbf{Javascript}. 

A fase exploratória foi importante para apoiar na definição das fases e 
procedimentos para implantação
automatizada. Além de que, a partir dela, foram definidos características importantes 
que devem ser levados em consideração para a escolha das próximas aplicações.

As atividades iniciais levantadas foram:

 \begin{enumerate}
   \item  Suporte a instalação automatizada do Wordpress.
   \item  Suporte a instalação automatizada do Owncloud.
   \item  Forçar as aplicações Wordpress e Owncloud a
   utilizar o protocolo HTTPS.
   \item  Suporte a instalação automatizada do servidor de e-mail,
   \item  Suporte a instalação automatizada do MoinMoin.
   \item  Suporte a instalação automatizada do Roundcube.
   \item  Suporte a instalação automatizada do Noosfero.
 \end{enumerate}

Ao verificar que tanto a ferramenta Owncloud como a ferramenta Wordpress precisam
de configurações servidor de e-mail para algumas funcionalidades, também foi adicionado
a atividade de criação de uma receita para a automatizar a configuração de
servidor de e-mail. Como aspecto de segurança, ficou definido que tanto a aplicação
Owncloud como Wordpress deveriam usar sempre HTTPS por padrão, isso envolve também
uma estratégia de como serão gerenciados os certificados de segurança para aplicar
o protocolo HTTPS. 

Após a fase exploratória, outras aplicações foram escolhidas, são elas o MoinMoin, 
Roundcube e Noosfero, de acordo com as características desejadas, 
definidas na Seção \ref{subsection:validacao}

Dentro da implantação automatizada de software Debian GNU/Linux, existem várias
configurações possíveis. Neste trabalho, existem algumas características foram 
levadas em consideração, são elas:

\begin{enumerate}
  \item  Segurança na implantação de aplicações web,
   como por exemplo utilizar protocolos como \textit{HTTPS}.
  \item  Configuração de múltiplas instâncias de
   aplicações, utilizando hospedagem virtual.
\end{enumerate}

Por fim, foi necessário definir procedimentos para a execução da implantação 
automatizada das aplicações. Considerando
o processo de implantação de software visto no Capítulo \ref{cap-referencial}
e a referência dos trabalhos relacionados. Foi necessário definir as fases e os
procedimentos para implantação automatizada, definindo as etapas
da implantação.

\subsection{Fases e Procedimentos para implantação}
\label{sec:fases}

Primeiramente, era necessário definir as fases e os procedimentos para a implantação 
automatizada, de acordo com as fases que compõem o processo de 
implantação de aplicações, e de acordo com a Seção \ref{sub:processoimplantaaco}. 

Neste trabalho, foi adicionado a fase de configuração de múltiplas
instâncias, que é a fase responsável por habilitar a implantação de múltiplas 
instâncias de uma aplicação. Isso possibilita a configuração 
de várias instâncias da mesma aplicação sem duplicação de recursos. As fases 
definidas são:

%TODO ilustrar:
\begin{itemize}
  \item  \textbf{Planejamento}: Identificar os componentes mínimos necessários na implantação
 da aplicação.  
  \item  \textbf{Preparação e Instalação de Pacotes}: Preparar o ambiente alvo, 
isso envolve configuração do sistema operacional e instalação de dependências necessárias.
  \item  \textbf{Configuração}: Editar os arquivos de configuração necessários, tanto 
da aplicação como de suas dependências.
  \item  \textbf{Configuração de múltiplas instâncias}: Configurar o suporte de múltiplas instâncias
da aplicação.   
  \item  \textbf{Inicialização/Execução}: Testar as receitas construídas, utilizando
a ferramenta Shak. 
\end{itemize}

Também, foi definido as caraterísticas de segurança na implantação, e
automatizar os que forem possíveis de serem aplicados dentro do contexto da
arquitetura proposta. Foi definido que as aplicações sempre usem protocolos 
seguros, como o \textit{HTTPS}. 
Para as aplicações web é necessário utilizar o protocolo HTTPS, para aplicações
de e-mail é necessário utilizar os protocolos \textit{SMTP} e \textit{IMAPS}.

\subsection{Validação da solução}
\label{subsection:validacao}

Para validar a evolução da ferramenta, foram feitos exemplos de uso,
com as aplicações defindas, ou seja, aplicações que tenham todo o seu 
processo de instalação e configuração automatizado, a
fim de refinar e evoluir a ferramenta, conforme problemas forem surgindo. A escolha
desses exemplos de uso devem ser feitas a partir de aplicações reais e
conhecidas na comunidade de software livre. 

Depois da fase exploratória, observou-se outras características importantes, e que 
devem ser levados em consideração para a escolha das aplicações. além do fato 
popularidade, utilizado anteriormente. As seguintes 
características foram levadas em consideração para a escolha das aplicações:

%TODO definir se vai usar primeira ou terceira pessoa
\begin{description}
  \item  [Aplicações empacotadas no Debian:] Como o intuito do trabalho
  é realizar implantações múltiplas a partir de um pacote único, tais aplicações
  devem estar empacotadas e disponíveis para instalação nos servidores do Debian. Caso
a aplicação não esteja disponível no Debian, é necessário empacotá-la e distribuir
nos servidores oficiais do Debian.
 \item  [Servidor web compatível:] As ferramentas escolhidas possuem, no
  mínimo, o servidor web compatível. Por exemplo, todas as aplicações devem 
possuir suporte ao servidor Nginx ou Apache.
  \item  [Aplicações com comunidades ativas:] É importante que os softwares 
escolhidos possuem comunidades ativas, isso pode ajudar na resolução de possíveis 
problemas. Logo, aplicações abandonadas pela sua comunidade foram evitadas, e
  aplicações com comunidade de desenvolvedores e usuários ativas foram priorizadas.
  Por utilizar a distribuição instável, é possivel que sejam descobertos
  erros ou melhorias dentro dessas ferramentas, e tais erros e melhorias são
  reportados para a comunidade, ou até mesmo problemas solucionados, e 
  devolvidos aos mantenedores das ferramentas.
  \item  [Documentação do software:] A documentação do software também foi
  levado em consideração, principalmente a documentação da instalação e configuração
  dentro da ferramenta. As ferramentas que não possuem documentação de instalação e
  configuração foram evitadas.
  \item  [Aplicações com suporte a federação:] Aplicações federadas permitem
 utilização de métodos de autenticação única para cada instância da aplicação,
mantendo a compatibilidade entre elas, um exemplo é a aplicação Owncloud,
que permite compartilhar seus arquivos na nuvem independente do Owncloud que estiver usando,
isto é possível utilizando apenas um identificador único, chamado ID Federated Cloud, assim
possibilitando que os arquivos do usuário sejam compartilhados entre suas instâncias do
Owncloud na nuvem.
\end{description}

A partir dessas características definidas, também era necessário encontrar os exemplos de uso
utilizados para a execução da solução.

\subsection{Exemplos de uso: busca dos pacotes das aplicações}
\label{subsection:exemplos}

Para encontrar as aplicações que possam se encaixar dentro desses parâmetros
devemos buscar por alguns exemplos de uso. Para a escolha das aplicações que
foram utilizadas como exemplos de uso, foi necessário fazer uma busca nas aplicações
web  que possuem suporte a configuração de múltiplas
instâncias, essa busca levou em consideração também, a documentação para realizar
tal configuração. Foram levantados algumas aplicações web da seguinte forma:

\begin{center}
apt-cache search web | wc -l
\end{center}

O resultado obtido com pacotes que contenham a palavra web recebe o resultado de 3470
pacotes de diversas aplicações ou módulos de aplicações, como
Wordpress; Owncloud; Drupal; Mailman e Chromium. Logo, dentro dessa busca
encontramos alguns pacotes de aplicações conhecidas, para facilitar a busca basta
aplicar para alguns nomes de aplicações conhecidos como:

\begin{center}
apt-cache search web | grep wordpress
\end{center}

Para avaliar se as aplicações possuem suporte a múltiplas instâncias foi necessário
analisar a documentação das aplicações, as aplicações costumam disponibilizar a sua
documentação na sua página ou numa wiki, também é possível checar a documentação
quando se instala uma aplicação nos arquivos de documentação da aplicação em
/usr/share/doc/ ou também pode-se utilizar os comandos:

\begin{center}
man "nome aplicação"

info "nome aplicação"
\end{center}

Analisando a documentação de algumas aplicações como Wordpress, Redmine,
Owncloud, Mailman, Wikimédia, foram escolhidas as aplicações Wordpress, Owncloud, MoinMoin
e Roundcube, por conter as características definidas anteriormente. Além disso,
também foi escolhido a aplicação Noosfero, que é uma aplicação que ainda não
está no Debian, porém neste trabalho será feito um esforço inicial para que isso
seja possível.

Foi feito uma análise da documentação dessas ferramentas, para encontrar a
possibilidade da configuração de múltiplas instâncias no mesmo servidor. Além
disso, também foi visto se as aplicações tinham suporte ao servidor Nginx que é
o servidor utilizado pelo Shak. Por fim, as aplicações também foram escolhidas
pela grande comunidade de usuários ativa e as duas aplicações também possuírem
canais de \textit{IRC} para tirar dúvidas com desenvolvedores e usuários. 

Também foi levado em consideração o fato da  aplicação Owncloud dar suporte a federação, e 
até esta fase do trabalho, a aplicação Noosfero também está com suporte a 
federação, porém ainda em desenvolvimento.

%TODO ilustrar a arquitetura
Para realizar a implantação das aplicações, foi importante que as 
instalações e configurações
fossem executadas num ambiente limpo, os testes foram criados
em máquinas virtuais com a configuração conhecida como minimal, que contém
instalado apenas as aplicações necessárias para o funcionamento do sistema operacional.

A partir da execução dos exemplos de uso, foi possível a coleta de informações
necessárias para a validação da implantação correta das aplicações. Essa
validação, a princípio, foi feita a partir da execução da implantação e
verificação das funcionalidades da aplicação implantada.

Após a execução da implantação, o testador verificou o perfeito
funcionamento de algumas funcionalidades básicas das aplicações escolhidas,
e principalmente verificar a implantação de várias instâncias da mesma
aplicação no mesmo servidor destino, observando o perfeito funcionamento de todas as
instâncias implantadas, para assim, validar a implantação de múltiplas instâncias, conforme
apresentado nos resultados do próximo capítulo.
