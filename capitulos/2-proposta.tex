\chapter{Metodologia}
\label{cap-metodologia}

================= Precisa falar da pesquisa? ex: qualitativa, descritiva.. etc ============/============

Este capítulo aborda sobre tudo que compõe a meotodologia de pesquisa utilizada
durante o trabalho, a fim de explciar como o trabalho foi realizado para atingir os
objetivos, servindo como base para a sua reprodução em trabalhos futuros.

A questão de pesquisa a ser respondida neste trabalho é:

\begin{center}
  \textit{
    Como implantar multiplas aplicações a partir de um pacote único
    de forma centralizada e segura?
}
\end{center}

Tendo a questão de pesquisa como isumo, foram levantadas algumas hipóteses  como:

\begin{itemize}
  \item  \textbf{H1 :} É possível criar varias instâncias de aplicações
  web a partir de um pacote único no mesmo servidor.
  \item  \textbf{H2 :} Os aspectos de segurança em uma implantação
   automatizada de sofware impacta na configuração de múltiplas aplicações no
   mesmo servidor.
\end{itemize}

Para responder h1 será necessário definir uma forma de configurar várias instâncias
de aplicações web utilizando um único pacote, com isso poderemos automatizar
essa configuração e testar para pacotes debian disponíveis. Com isso poderemos
mapear os pacotes que são possíveis de realizar tal configuração.

Para responder h2 será necessário definir quais são os aspectos de segurança em
uma implantação de software e aplicá-los dentro do contexto de implantação de
automatizada de  múltiplas instâncias de aplicações no mesmo servidor,
com isso ver o impacto causado na arquitetura de implantação, ver quais são os
aspectos comuns entre as aplicações, lidar com diferentes linguagens e aspectos
de implantação segura específicos de cada aplicação.

================= ver como está ficando e tirar duvidas ============/============

\section{Trabalhos Relacionados}

trabalhos academicos:

citar trabalho do rafael
procurar outros trabalhos acadêmicos

ferramentas do mercado:

citar o bitnami
citar o storm.io

\section{Detalhes da Proposta}

A proposta deste trabalho é trazer uma solução para a implantação automatizada
de múltiplas aplicações a partir de um pacote único de forma centralizada e segura.
Para isso será primeiramente necessário definir uma arquitetura padrão para a
implantação dessas aplicações, ou seja, definir:

\begin{itemize}
  \item  \textbf{Tecnologia para automatizar implantação :}  Definir qual será a
  tecnologia escolhida dentre as citadas no referencial teórico para dar suporte
  a implantação.
  \item  \textbf{Conjunto mínimo de dependências:} Definir quais são as dependências
  mínimas para o funcionamento das aplicações escolhidas, tais como: sistema operacional,
  pacotes pré-instalados e aplicações pré-configuradas.
  \item  \textbf{Procedimentos para implantação:} Definir quais são etapas da implantação,
  definir a ordem necessária para a execução de cada fase da implantação.
  \item  \textbf{Aspectos de segurança} Definir quais são os aspectos de segurança
  e automatizar os que forem possíveis de serem aplicados dentro do contexto desejado.
\end{itemize}

Com a arquitetura definida o próximo passo é definir um ambiente de desenvolvimento e de
testes, dado o contexto de implantação de software é importante possuir um abiente fexível para
testar facilmente a instalação e configuração das aplicações e podendo facilmente
reinicializar esse ambiente de forma que ele fique limpo sem resquícios da instalação
anterior.

Com a arquitetura definida e com um ambiente de desenvolvimento e de testes configurados, é
necessário definir quais são as ferramentas que servirão como caso de testes
para a implementação da solução. Elas serão importantes pois a partir delas
e de suas necessidades que será implementado a primeira versão da proposta para
a solução. Por isso também é importante definir aspectos que servirão como parâmetro
para ajudar no desenvolvimento da solução. Esses aspectos são:

\begin{itemize}
  \item  \textbf{Aplicações empacotadas no debian :}  Como o intuito do trabalho
  é realizar implantações múltiplas a partir de um pacote único, tais aplicações
  devem estar empacotadas e disponíveis para instalação nos servidores do debian.
  Isso impacta na escolha da ferramenta, visto que não será necessário ter o trabalho
  de empacotar aplicações que ainda não estão empacotadas no debian.
  \item  \textbf{Servidor web compatível:} As ferramentas escolhidas devem no
  mínimo possuir o servidor web compatível, por exemplo: as aplicações juntas
  devem possuir suporte para nginx ou apache ou similares.
  \item  \textbf{Aplicações com comunidades ativas:} Como estamos trabalhando
  com software livre, é importante que os softwares escolhidos possuem comunidades
  ativas, isso pode ajudar na resolução de  possíveis problemas, logo aplicações
  desconhecidas devem ser evitadas, e aplicações conhecidas devem ser priorizadas.
  E isso também pode ser um fator importante caso durante os testes sejam descobertos
  bugs ou melhorias dentro dessas ferramentas.
  \item  \textbf{Doscumentação do software:} A documentação do software também deve
  ser levado em consideração, principalmente a documentação da instalação e configuração
  dentro da arquitetura definida.
\end{itemize}

FIXME
Por fim, com a arquitetura definida, ambientes definidos e ferramentas escolhidas
é mão na massa... como escrever isso?

\section{Ferramentas e Dados}

vou falar aqui sobre o shak, sobre o summer of code? sobre a participação do terceiro
nas decisões técnicas.

\section{Atividades}

Com intuito de ...., foi definido um fluxo das atividades necessárias para organizar
o trabalho. .....

%
